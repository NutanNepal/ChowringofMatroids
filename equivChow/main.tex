\pdfoutput=1 
\documentclass[10pt, a4paper, english]{amsart}
\usepackage{amsmath} % utilities for mathematics
\usepackage{amsthm} % utilities for theorem environment
\usepackage{amssymb} % loads mathematical fonts
\usepackage[dvipsnames]{xcolor}
\usepackage{amscd} % utilities for commutative diagrams
\usepackage{mathtools}

\usepackage{subcaption}% <-- added

\usepackage{array} % core package
\usepackage{newtxtext,newtxmath} %fuente

\usepackage[cal=boondoxo,scr=euler]{mathalfa}
\usepackage[backref=page,linktocpage]{hyperref} % permits navigating on the pdf
\usepackage{cleveref} % smart referencing
\usepackage{caption} %
\usepackage{graphics,graphicx} % permits putting images
\usepackage{tikz,tikz-cd} % tool for diagrams

%\usetikzlibrary{graphdrawing.force}% quotes library is for the [""] edges
\usetikzlibrary{graphs,graphs.standard,calc}

\usetikzlibrary{decorations.pathreplacing,}%angles,quotes

\usepackage{enumerate} % permits enumerating using letters or other symbols

\DeclareMathAlphabet{\mathsf}{OT1}{\sfdefault}{m}{n}

\newcommand{\nocontentsline}[3]{}
\newcommand{\tocless}[2]{\bgroup\let\addcontentsline=\nocontentsline#1{#2}\egroup}


\usepackage[margin=1.20in]{geometry}
%\usepackage[textwidth=360pt,textheight=615pt]{geometry}
\linespread{1.08}
%\setlength\parindent{0.3in}

\usepackage{verbatim}


\usepackage{scalerel}

\makeatletter
\def\dual#1{\expandafter\dual@aux#1\@nil}
\def\dual@aux#1/#2\@nil{\begin{tabular}{@{}c@{}}#1\\#2\end{tabular}}
\makeatother

\makeatletter
\@namedef{subjclassname@2020}{\textup{2020} Mathematics Subject Classification}
\makeatother


\newcommand{\stirlingtwo}[2]{\biggl\{\genfrac{}{}{0pt}{}{#1}{#2}\biggr\}}
\newcommand{\tstirlingtwo}[2]{\left\{\genfrac{}{}{0pt}{}{#1}{#2}\right\}}
%\renewcommand{\tilde}{\widetilde}
%\renewcommand{\widetilde}{\wwtilde}

\DeclareMathAlphabet{\amathbb}{U}{bbold}{m}{n}


\newcommand{\mfrak}[1]{\scaleobj{0.93}{\mathfrak{#1}}}

\hypersetup{
    colorlinks = true,
    linkbordercolor = {white},
    linkcolor = {BrickRed},
    anchorcolor = {black},
    citecolor = {BrickRed},
    filecolor = {cyan},
    menucolor = {BrickRed},
    runcolor = {cyan},
    urlcolor = {black}
}

\usetikzlibrary{automata}


\newtheoremstyle{teoremas}% <name>
{11pt}% <Space above>
{11pt}% <Space below>
{\itshape}% <Body font>
{}% <Indent amount>
{\bfseries}% <Theorem head font>
{}% <Punctuation after theorem head>
{.5em}% <Space after theorem headi>
{}% <Theorem head spec (can be left empty, meaning `normal')>



\theoremstyle{teoremas}
\newtheorem{theorem}{Theorem}[section]
\newtheorem{corollary}[theorem]{Corollary}
\newtheorem{lemma}[theorem]{Lemma}
\newtheorem{proposition}[theorem]{Proposition}

\newtheoremstyle{definition}% <name>
{11pt}% <Space above>
{11pt}% <Space below>
{}% <Body font>
{}% <Indent amount>
{\bfseries}% <Theorem head font>
{}% <Punctuation after theorem head>
{.5em}% <Space after theorem headi>
{}% <Theorem head spec (can be left empty, meaning `normal')>

\theoremstyle{definition}
\newtheorem{definition}[theorem]{Definition}
\newtheorem{conjecture}[theorem]{Conjecture}
\newtheorem{openproblem}[theorem]{Open Problem}
\newtheorem{problem}[theorem]{Problem}
\newtheorem{question}[theorem]{Question}
\newtheorem{example}[theorem]{Example}
\newtheorem{remark}[theorem]{Remark}

\crefname{theorem}{theorem}{theorems}
\Crefname{theorem}{Theorem}{Theorems}
\crefname{lemma}{lemma}{lemmas}
\Crefname{lemma}{Lemma}{Lemmas}
\crefname{proposition}{proposition}{propositions}
\Crefname{proposition}{Proposition}{Propositions}

\DeclareMathOperator{\rk}{rk}
\DeclareMathOperator{\cusp}{cusp}
\DeclareMathOperator{\Rel}{Rel}
\DeclareMathOperator{\Poin}{Poin}
\DeclareMathOperator{\Tor}{Tor}
\DeclareMathOperator{\Ext}{Ext}

\newcommand{\M}{\mathsf{M}}
\newcommand{\grp}{\mathsf{G}}
\newcommand{\stab}{\mathsf{Stab}}
%\newcommand{\B}{\mathsf{B}} let us only write \U_{n,n} for boolean matroids
\newcommand{\N}{\mathsf{N}}
\newcommand{\U}{\mathsf{U}}
\newcommand{\cI}{\mathcal{I}}
\newcommand{\symm}{\mathfrak{S}}
\newcommand{\A}{\mathrm{A}}
\newcommand{\Q}{\mathbb{Q}}
\newcommand{\R}{\mathbb{R}}
\newcommand{\Z}{\mathbb{Z}}

\newcommand{\LL}{\mathsf{\Lambda}}
\newcommand{\Hilb}{\operatorname{Hilb}}
\newcommand{\rank}{\operatorname{rk}}
\newcommand{\cl}{\operatorname{cl}}
\renewcommand{\H}{\mathrm{H}}
\newcommand{\CH}{\mathrm{CH}}
\newcommand{\aug}{\operatorname{aug}}
\newcommand{\IH}{\mathrm{IH}}
\newcommand{\uH}{\underline{\mathrm{H}}}
\newcommand{\uCH}{\underline{\mathrm{CH}}}
\newcommand\size[1]{|#1|}
\DeclareMathOperator{\Pal}{\mathrm{Pal}}
\DeclareMathOperator{\Poly}{\mathrm{Poly}}

\newcommand{\cL}{\mathcal{L}}
\newcommand{\Rep}{\operatorname{Rep}}
\newcommand{\gr}{\operatorname{gr}}
\newcommand{\VRep}{\operatorname{VRep}}
\newcommand{\Ind}{\operatorname{Ind}}
\newcommand{\Res}{\operatorname{Res}}
\newcommand{\Aut}{\operatorname{Aut}}
\newcommand{\col}{\operatorname{col}}
\newcommand{\asc}{\operatorname{asc}}
\newcommand{\bad}{\operatorname{bad}}
\newcommand{\des}{\operatorname{des}}

\AtBeginDocument{%
   \def\MR#1{}
}


%% preliminary version only
\begin{comment}
 \usepackage{prelim2e}
 \renewcommand{\PrelimText}{\sf\scriptsize --- INCOMPLETE DRAFT of \today ---}
 \marginparwidth=0.75in
 \usepackage[notcite,notref]{showkeys}
 \usepackage[colorinlistoftodos,bordercolor=orange,backgroundcolor=orange!20,linecolor=orange,textsize=scriptsize]{todonotes}

 \usepackage{prelim2e}
 \usepackage[us,24hr]{datetime}
 \renewcommand{\PrelimText}{\sf\scriptsize ---\,Version of \today \,-\,\currenttime\,---}
\end{comment}

\title[Equivariant Chow Polynomials of Matroids]{Equivariant Chow Polynomials of Matroids}

\author[N. Nepal]{Nutan Nepal}

\address{(N. Nepal)
    North Carolina State University
}
\email{nnepal2@ncsu.edu}


\begin{document}

\begin{abstract}
    beeboop beeboop\dots
\end{abstract}

\maketitle

\section{Introduction}\label{sec:introduction}

\subsection{Overview}
Given a matroid $\M = (E, \cI)$, we can define its Chow ring $\uCH$ and augmented Chow ring $\CH$ for which the bases are given by:
    $$\text{FY}=\{x_{F_1}^{m_1}x_{F_2}^{m_2}\cdots x_{F_k}^{m_k}:\ (\varnothing = F_0)\subset
        F_1\subset F_2\subset\cdots\subset
        F_k,\ \text{and}\ m_i\leq\rank(F_i)-\rank(F_{i-1})-1 \},$$
and
    $$\widetilde{FY} = \{x_{F_1}^{a_1}x_{F_2}^{a_2}\cdots x_{F_m}^{a_m}\ 
    |\ \varnothing\subset F_1\subset\cdots\subset F_m,\ 1\leq a_1\leq \rank(F_1)
    ,\ a_i\leq\rank(F_i)-\rank(F_{i-1})-1\ \text{for } i>1\}$$
respectively. Here $\varnothing\subset F_1\subset\cdots\subset F_m$ is a strictly increasing chain of flats of the matroid $\M$.
The following theorems were proved in~\cite{fmsv24}:.

\begin{theorem}\label{thm:fmsv-main-recursion-defi-H-and-uH}
    There is a unique way to assign to each loopless matroid $\M$ a palindromic polynomial $\uH_{\M}(x) \in \Z[x]$ such that the following properties hold:
    \begin{enumerate}[\normalfont(i)]
        \item If $\rk(\M) = 0$, then $\uH_{\M}(x) = 1$.\label{it:fmsv-mainfirst}\
        \item If $\rk(\M) > 0$, then $\deg \uH_{\M}(x) = \rk(\M) - 1$.\label{it:fmsv-mainsecond}
        \item For every matroid $\M$, the polynomial
            \[ \H_{\M}(x) := \sum_{F\in \mathcal{L}(\M)} x^{\rk(F)}\, \uH_{\M/F}(x)\]
        is palindromic.\label{it:fmsv-mainthird}
    \end{enumerate}
\end{theorem}

\begin{theorem}\label{thm:fmsv-intro-main0}
    There is a unique way to assign to each loopless matroid $\M$  a polynomial $\uH_{\M}(x)\in \mathbb{Z}[x]$ such that the following conditions hold:
    \begin{enumerate}[\normalfont(i)]
        \item If $\rk(\M) = 0$, then $\uH_{\M}(x) = 1$.
        \item For every matroid $\M$, the following recursion holds:
            \[ \uH_{\M}(x) = \sum_{\substack{F\in\mathcal{L}(\M)\\ F\neq\varnothing}} \overline{\chi}_{\M|_F}(x)\, \uH_{\M/F}(x).\]
    \end{enumerate}
\end{theorem}

It was shown in \cite{fmsv24} that these polynomials $\H_\M$ and $\uH_\M$ are the Hilbert-Poincare series
for the augmented Chow ring $\CH$ and the Chow ring $\uCH$ respectively. In other words,
$$\uH_\M(x) = |FY^0| + |FY^1|x +\cdots + |FY^{r-1}|x^{r-1}$$
where $|FY^i|$ denotes the number of fy-monomials of degree $i$ (which equals the dimension of the degree $i$ piece of the Chow ring).

\section{Matroids}\label{sec:matroids}
\subsection{Action}
Given a finite set $E$ with $n$ elements, the symmetric group $\symm_n$ can always act on $E$ by permutation and this induces an
action on the power set $2^E$. If $\M = (E, \cI)$ is a matroid, let $G$ be the stabilizer subgroup $\stab_{\symm_n}(\cI)$ of $\symm_n$ that
stabiizes the set $\cI\subseteq 2^E$. Then, we say that the group $G$ acts on the matroid $\M$.

The action of $G$ on $\M$ induces an action on the Chow ring and the augmented Chow ring of $\M$ by permuting the fy-monomials. Since the action doesn't
affect the degree of the monomials, it is clear that $G$ acts on the Chow ring by acting separately on each graded piece of $\CH$.

Let $V^i$ denote the permutation representation of $G$ on the set $FY^i$. Then $\dim(V^i) = |FY^i|$, the dimension of the $i$-th piece.
We define the polynomial $V^0+V^1x+\cdots V^{r-1}x^{r-1}\in\VRep_G[x]$ to be the equivariant Chow polynomial of the matroid $\M$. In this paper,
we will prove the following theorems:

\begin{theorem}\label{thm:main-perm-rep}
    Let $\M$ be a loopless matroid and $\uH_{\M}^G$ be its equivariant Chow polynomial. Then $\uH_{\M}^G$ is given by
    \begin{equation}
        \uH_{\M}^G(x) = \sum_{\varnothing = F_0\subset F_1\subset\cdots\subset F_m}{\frac{|G_{F_0\cdots F_m|}}{|G|}
        \left(\prod_{i=1}^{m}{\frac{x(1-x^{\rk(F_i)-\rk(F_{i-1})-1})}{1-x}}\right)
        \Ind^G(1_{G_{F_0\cdots F_m}})}.
    \end{equation}
    Here, $G_{F_0\cdots F_m} = G_{F_0}\cap\cdots\cap G_{F_m}$ denotes the stabilizer of the chain $(F_0\subset F_1\subset\cdots\subset F_m)$
    and the sum is taken over all nonempty chains of flats starting at $\varnothing$.
\end{theorem}

\begin{theorem}\label{thm:main-perm-rep-augmented}
    Let $\M$ be a loopless matroid and $\H_{\M}^G$ be its equivariant augmented Chow polynomial. Then $\H_{\M}^G$ is given by
    \begin{equation}
        \H_{\M}^G(x) = 1_G + \sum_{F_0\subset F_1\subset\cdots\subset F_m}{\frac{|G_{F_0\cdots F_m|}}{|G|}
        \frac{x(1-x^{\rk(F_0))})}{1-x}
        \left(\prod_{i=1}^{m}{\frac{x(1-x^{\rk(F_i)-\rk(F_{i-1})-1})}{1-x}}\right)
        \Ind^G(1_{G_{F_0\cdots F_m}})}.
    \end{equation}
    Here, $G_{F_0\cdots F_m} = G_{F_0}\cap\cdots\cap G_{F_m}$ denotes the stabilizer of the chain $(F_0\subset F_1\subset\cdots\subset F_m)$
    and the sum is taken over all nonempty chains of flats.
\end{theorem}

\begin{theorem}\label{thm:equiv-main-recursion-defi-H-and-uH}
    There is a unique way to assign to each loopless matroid $\M$ a palindromic polynomial $\uH_{\M}^G(x) \in \VRep_G[x]$ such that the following properties hold:
    \begin{enumerate}[\normalfont(i)]
        \item If $\rk(\M) = 0$, then $\uH_{\M}^G(x) = 1_G$.\label{it:mainfirst}\
        \item If $\rk(\M) > 0$, then $\deg \uH_{\M}^G(x) = \rk(\M) - 1$.\label{it:mainsecond}
        \item For every matroid $\M$, the polynomial
            \[ \H_{\M}^G(x) := \sum_{F\in \mathcal{L}(\M)} x^{\rk(F)}\, \frac{|G_F|}{|G|}\Ind^G\left(\uH_{\M/F}^{G_F}(x)\right)\]
        is palindromic.\label{it:main-equiv-third}
    \end{enumerate}
\end{theorem}

\begin{theorem}\label{thm:equiv-main}
    There is a unique way to assign to each loopless matroid $\M$  a polynomial $\uH_{\M}^G(x)\in \VRep_G[x]$ such that the following conditions hold:
    \begin{enumerate}[\normalfont(i)]
        \item If $\rk(\M) = 0$, then $\uH_{\M}^G(x) = 1_G$.
        \item For every matroid $\M$, the following recursion holds:
            \[ \uH_{\M}^G(x) = \sum_{\substack{F\in\mathcal{L}(\M)\\ F\neq\varnothing}} \frac{|G_F|}{|G|}\Ind^G\left(\overline{\chi}_{\M|_F}^{G_F}(x)\otimes \uH_{\M/F}^{G_F}(x)\right).\]
    \end{enumerate}
\end{theorem}

\section{Scratch}

\begin{proposition}\label{prop:fy-main}
    Let $\M$ be a loopless matroid. The Hilbert--Poincare series of the Chow ring $\uCH(M)$ is given by
    \begin{equation}\label{eqn:fy-main}
        \uH_{\M}(x) = \sum_{\varnothing = F_0\subset F_1\subset\cdots\subset F_m}
        \left(\prod_{i=1}^{m}{\frac{x(1-x^{\rk(F_i)-\rk(F_{i-1})-1})}{1-x}}\right).
    \end{equation}
    Here, the sum is taken over all nonempty chains of flats starting at $\varnothing$.
\end{proposition}
We notice a few things about the formula in \ref*{prop:fy-main}:
\begin{enumerate}[(i)]
    \item The chain corresponding to just the empty flat gives an empty product which equals 1.
    \item If $\rk(F_i)-\rk(F_{i-1})-1 = 0$ for some $i$ in the chain $F_0\subset F_1\subset\cdots\subset F_m$, then the product is 0.
    \item So, given a chain $F_0\subset F_1\subset\cdots\subset F_m$ with $\rk(F_i)-\rk(F_{i-1})-1 > 0$ for all $i$, we can write the product as
        $a_1x+\cdots+a_{r-1}x^{r-1}$ for some positive integers $a_j$'s.
\end{enumerate}

In particular, we can restate the equation \ref*{eqn:fy-main} as
    $$\uH_{\M}(x) = 1 + \sum_{P_\varnothing}
    {a_1(P_\varnothing)x+\cdots+a_{r-1}(P_\varnothing)x^{r-1}},$$
where the sum is taken over all chains $P_\varnothing = (F_0\subset F_1\subset\cdots\subset F_m)$ starting at $\varnothing$ with $\rk(F_i)-\rk(F_{i-1})-1 > 0$ for all $i$ and
$a_j(P_\varnothing)$ are some integers depending on the chain.

$$S_0+S_1x+\cdots+S_rx^r = \sum_{i\in I}{\sum_{j=0}^{r}{T_{ij}x^j}}$$
$$S_k = \sum_{i\in I}{T_{ik}}$$
$$|S_0|+|S_1|x+\cdots+|S_r|x^r =\sum_{i\in I}{p_i(x)}$$

\section{Going on a tangent}



\bibliographystyle{amsalpha}
\bibliography{bibliography}
\end{document}