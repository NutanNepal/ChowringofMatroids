%%%%%%%%%%%%%%%%%%%%%%%%%%%%%%%%%%%%%%%%%
% Jacobs Landscape Poster
% LaTeX Template
% Version 1.1 (14/06/14)
%
% Created by:
% Computational Physics and Biophysics Group, Jacobs University
% https://teamwork.jacobs-university.de:8443/confluence/display/CoPandBiG/LaTeX+Poster
% 
% Further modified by:
% Nathaniel Johnston (nathaniel@njohnston.ca)
%
% This template has been downloaded from:
% http://www.LaTeXTemplates.com
%
% License:
% CC BY-NC-SA 3.0 (http://creativecommons.org/licenses/by-nc-sa/3.0/)
%
%%%%%%%%%%%%%%%%%%%%%%%%%%%%%%%%%%%%%%%%%

%----------------------------------------------------------------------------------------
%	PACKAGES AND OTHER DOCUMENT CONFIGURATIONS
%----------------------------------------------------------------------------------------

\documentclass[final]{beamer}

\usepackage[scale=1.24]{beamerposter} % Use the beamerposter package for laying out the poster


\usetheme{confposter} % Use the confposter theme supplied with this template

\setbeamercolor{block title}{fg=ngreen,bg=white} % Colors of the block titles
\setbeamercolor{block body}{fg=black,bg=white} % Colors of the body of blocks
\setbeamercolor{block alerted title}{fg=white,bg=dblue!70} % Colors of the highlighted block titles
\setbeamercolor{block alerted body}{fg=black,bg=dblue!10} % Colors of the body of highlighted blocks
% Many more colors are available for use in beamerthemeconfposter.sty

%-----------------------------------------------------------
% Define the column widths and overall poster size
% To set effective sepwid, onecolwid and twocolwid values, first choose how many columns you want and how much separation you want between columns
% In this template, the separation width chosen is 0.024 of the paper width and a 4-column layout
% onecolwid should therefore be (1-(# of columns+1)*sepwid)/# of columns e.g. (1-(4+1)*0.024)/4 = 0.22
% Set twocolwid to be (2*onecolwid)+sepwid = 0.464
% Set threecolwid to be (3*onecolwid)+2*sepwid = 0.708

\newlength{\sepwid}
\newlength{\onecolwid}
\newlength{\twocolwid}
\newlength{\threecolwid}
\setlength{\paperwidth}{48in} % A0 width: 46.8in
\setlength{\paperheight}{36in} % A0 height: 33.1in
\setlength{\sepwid}{0.003\paperwidth} % Separation width (white space) between columns
\setlength{\onecolwid}{0.31\paperwidth} % Width of one column
\setlength{\twocolwid}{0.28\paperwidth} % Width of two columns
\setlength{\threecolwid}{0.31\paperwidth} % Width of three columns
\setlength{\topmargin}{-0.6in} % Reduce the top margin size
%-----------------------------------------------------------

\usepackage{graphicx}  % Required for including images

\usepackage{booktabs} % Top and bottom rules for tables

\usepackage{tikz}
\usepackage{tikz-cd}

\usepackage{biblatex}
\addbibresource{biblio.bib}

%\usepackage[utf8]{inputenc}

\usetikzlibrary{matrix,shapes,arrows,positioning,chains,decorations.pathmorphing,snakes,calc}

\DeclareMathOperator{\rk}{rk}
\DeclareMathOperator{\cusp}{cusp}
\DeclareMathOperator{\Rel}{Rel}
\DeclareMathOperator{\Poin}{Poin}
\DeclareMathOperator{\Tor}{Tor}
\DeclareMathOperator{\Ext}{Ext}

\newcommand{\M}{\mathsf{M}}
\newcommand{\FY}{\mathsf{FY}}
\newcommand{\grp}{\mathsf{G}}
\newcommand{\stab}{\mathsf{Stab}}
%\newcommand{\B}{\mathsf{B}} let us only write \U_{n,n} for boolean matroids
\newcommand{\N}{\mathsf{N}}
\newcommand{\U}{\mathsf{U}}
\newcommand{\cI}{\mathcal{I}}
\newcommand{\symm}{\mathfrak{S}}
\newcommand{\A}{\mathrm{A}}
\newcommand{\Q}{\mathbb{Q}}
\newcommand{\R}{\mathbb{R}}
\newcommand{\Z}{\mathbb{Z}}

\newcommand{\LL}{\mathsf{\Lambda}}
\newcommand{\Hilb}{\operatorname{Hilb}}
\newcommand{\rank}{\operatorname{rk}}
\newcommand{\cl}{\operatorname{cl}}
\renewcommand{\H}{\mathrm{H}}
\newcommand{\CH}{\mathrm{CH}}
\newcommand{\aug}{\operatorname{aug}}
\newcommand{\IH}{\mathrm{IH}}
\newcommand{\uH}{\underline{\mathrm{H}}}
\newcommand{\uCH}{\underline{\mathrm{CH}}}
\newcommand\size[1]{|#1|}
\DeclareMathOperator{\Pal}{\mathrm{Pal}}
\DeclareMathOperator{\Poly}{\mathrm{Poly}}

\newcommand{\cL}{\mathcal{L}}
\newcommand{\Rep}{\operatorname{Rep}}
\newcommand{\gr}{\operatorname{gr}}
\newcommand{\VRep}{\operatorname{VRep}}
\newcommand{\Ind}{\operatorname{Ind}}
\newcommand{\Res}{\operatorname{Res}}
\newcommand{\Aut}{\operatorname{Aut}}
\newcommand{\col}{\operatorname{col}}
\newcommand{\asc}{\operatorname{asc}}
\newcommand{\bad}{\operatorname{bad}}
\newcommand{\des}{\operatorname{des}}

\newcommand{\Ver}{7}
\newcommand{\Vera}{2}
\newcommand{\Hor}{15}
\newcommand{\Hora}{20}
\newcommand{\Horb}{3}
\newcommand{\Horc}{4.5}
\newcommand{\Hord}{6.3}
\newcommand{\Hore}{11}
\newcommand{\Horf}{7.7}
\newcommand{\Tone}{1.85}

\newcommand{\CC}{\mathbb{C}}
\newcommand{\TT}{\mathbb{T}}

\newcommand{\fg}{\mathfrak{g}}

\newcommand{\cF}{\mathcal{F}}
\newcommand{\cU}{\mathcal{U}}
\newcommand{\cO}{\mathcal{O}}

\newcommand{\bw}{\mathbf{w}}
\newcommand{\bP}{\mathbf{P}}
\newcommand{\be}{\mathbf{e}}

\newcommand{\sT}{\mathsf{T}}

\newcommand{\Db}{D^{\mathrm{b}}}
\newcommand{\GL}{\mathbf{GL}}
\newcommand{\IC}{\mathrm{IC}}



\newcounter{sarrow}
\newcommand\xrsquigarrow[1]{%
\stepcounter{sarrow}%
\begin{tikzpicture}[decoration=snake]
\node (\thesarrow) {\strut#1};
\draw[->,decorate] (\thesarrow.south west) -- (\thesarrow.south east);
\end{tikzpicture}%
}

\tikzset{
    ncbar angle/.initial=90,
    ncbar/.style={
        to path=(\tikztostart)
        -- ($(\tikztostart)!#1!\pgfkeysvalueof{/tikz/ncbar angle}:(\tikztotarget)$)
        -- ($(\tikztotarget)!($(\tikztostart)!#1!\pgfkeysvalueof{/tikz/ncbar angle}:(\tikztotarget)$)!\pgfkeysvalueof{/tikz/ncbar angle}:(\tikztostart)$)
        -- (\tikztotarget)
    },
    ncbar/.default=0.5cm,
}

\tikzset{square left brace/.style={ncbar=0.5cm}}
\tikzset{square right brace/.style={ncbar=-0.5cm}}

\tikzset{round left paren/.style={ncbar=0.5cm,out=120,in=-120}}
\tikzset{round right paren/.style={ncbar=0.5cm,out=60,in=-60}}

%----------------------------------------------------------------------------------------
%	TITLE SECTION 
%----------------------------------------------------------------------------------------

\title{Equivariant Chow Polynomials of Matroids} % Poster title

\author{Nutan Nepal} % Author(s)

\institute{Department of Mathematics, North Carolina State University} % Institution(s)

%----------------------------------------------------------------------------------------

\begin{document}
\tikzstyle{startstop} = [rectangle, rounded corners, minimum width=3cm, minimum height=1cm,text centered, draw=black, fill=dblue!10]
%\tikzstyle{startstop} = [rectangle, rounded corners, minimum width=3cm, minimum height=1cm,text centered, draw=black, fill=ngreen]
\tikzstyle{io} = [trapezium, trapezium left angle=70, trapezium right angle=110, minimum width=3cm, minimum height=1cm, text centered, draw=black, fill=dblue!40]
\tikzstyle{process} = [rectangle, minimum width=3cm, minimum height=1cm, text centered, draw=black, fill=orange!30]
\tikzstyle{decision} = [diamond, minimum width=3cm, minimum height=1cm, text centered, draw=black, fill=green!30]
\tikzstyle{arrow} = [thick,->,>=stealth]



\addtobeamertemplate{block end}{}{\vspace*{2ex}} % White space under blocks
\addtobeamertemplate{block alerted end}{}{\vspace*{2ex}} % White space under highlighted (alert) blocks

\setlength{\belowcaptionskip}{2ex} % White space under figures
\setlength\belowdisplayshortskip{2ex} % White space under equations

\begin{frame}[t] % The whole poster is enclosed in one beamer frame

\begin{columns}[t] % The whole poster consists of three major columns, the second of which is split into two columns twice - the [t] option aligns each column's content to the top

\begin{column}{\sepwid}\end{column} % Empty spacer column

\begin{column}{\onecolwid} % The first column

\begin{block}{Goal}
When the Chow ring $\uCH(\M)$ of a matroid $\M$ carries an action of a
group $G$, we study the equivariant Chow polynomial $\uH_{\M}^G(x) \in \VRep_G[x]$:

\[\uH_\M^G(x) = \uCH^0(\M)+\uCH^1(\M)x+\cdots +\uCH^{\rk(\M)-1}(\M)x^{\rk(\M)-1}\]

and describe some of its properties.
\end{block}

\vspace{-5mm}
\begin{block}{Introduction}

A matroid $\M$ on the ground set $E$ with $n$ elements
can be identified with a geometric lattice
$\cL(\M)\subseteq 2^{[n]}$.
The following are  lattices corresponding to the braid matroid $\M(K_4)$
(the graphic matroid associated to the
complete graph on $4$ vertices)
and the uniform matroid $U_{3,4}$.

\begin{center}
    \begin{minipage}[t]{0.5\textwidth}
    \kern0pt
    \raggedright
    
    \begin{figure}
        \scalebox{0.3}{
        \begin{tikzpicture}[scale=4]
    
            % Level 1 (topmost node)
            \node at (0, 4) (A) {1234};
            
            % Level 2 (second row of nodes)
            \node at (-6, 2) (B1) {$124$};
            \node at (-4, 2) (B2) {$135$};
            \node at (-2, 2) (B3) {$25$};
            \node at (0, 2) (B4) {$16$};
            \node at (2, 2) (B5) {$34$};
            \node at (4, 2) (B6) {$236$};
            \node at (6, 2) (B7) {$456$};
            
            % Level 3 (third row of nodes)
            \node at (-5, 0) (C1) {$1$};
            \node at (-3, 0) (C2) {$2$};
            \node at (-1, 0) (C3) {$3$};
            \node at (1, 0) (C4) {$4$};
            \node at (3, 0) (C5) {$5$};
            \node at (5, 0) (C6) {$6$};
            
            % Level 4 (bottommost node)
            \node at (0, -2) (D) {$\varnothing$};
            
            % Arrows from top to second row
            \draw[->] (A) -- (B1);
            \draw[->] (A) -- (B2);
            \draw[->] (A) -- (B3);
            \draw[->] (A) -- (B4);
            \draw[->] (A) -- (B5);
            \draw[->] (A) -- (B6);
            \draw[->] (A) -- (B7);
            
            % Arrows from second to third row
            \draw[->] (B1) -- (C1);
            \draw[->] (B1) -- (C2);
            \draw[->] (B1) -- (C4);
            
            \draw[->] (B2) -- (C1);
            \draw[->] (B2) -- (C3);
            \draw[->] (B2) -- (C5);
            
            \draw[->] (B3) -- (C2);
            \draw[->] (B3) -- (C5);
            
            \draw[->] (B4) -- (C1);
            \draw[->] (B4) -- (C6);
            
            \draw[->] (B5) -- (C3);
            \draw[->] (B5) -- (C4);
            
            \draw[->] (B6) -- (C2);
            \draw[->] (B6) -- (C3);
            \draw[->] (B6) -- (C6);
            
            \draw[->] (B7) -- (C4);
            \draw[->] (B7) -- (C5);
            \draw[->] (B7) -- (C6);
            
            % Arrows from third row to bottom
            \draw[->] (C1) -- (D);
            \draw[->] (C2) -- (D);
            \draw[->] (C3) -- (D);
            \draw[->] (C4) -- (D);
            \draw[->] (C5) -- (D);
            \draw[->] (C6) -- (D);
            
            \end{tikzpicture}
        }
            \caption{\ $\M(K_4)$.} \label{fig:M1}
    \end{figure}
    \end{minipage}%
    \begin{minipage}[t]{0.5\textwidth}
    \kern0pt
    \center
    %\raggedleft
    \vspace{.5cm}
    \begin{figure}
        \scalebox{0.3}{
        \begin{tikzpicture}[scale=4]
    
            % Level 1 (topmost node)
            \node at (0, 4) (A) {1234};
            
            % Level 2 (second row of nodes)
            \node at (-5, 2) (B1) {12};
            \node at (-3, 2) (B2) {13};
            \node at (-1, 2) (B3) {14};
            \node at (1, 2) (B4) {23};
            \node at (3, 2) (B5) {24};
            \node at (5, 2) (B6) {34};
            
            % Level 3 (third row of nodes)
            \node at (-3, 0) (C1) {1};
            \node at (-1, 0) (C2) {2};
            \node at (1, 0) (C3) {3};
            \node at (3, 0) (C4) {4};
            
            % Level 4 (bottommost node)
            \node at (0, -2) (D) {$\varnothing$};
            
            % Arrows from top to second row
            \draw[->] (A) -- (B1);
            \draw[->] (A) -- (B2);
            \draw[->] (A) -- (B3);
            \draw[->] (A) -- (B4);
            \draw[->] (A) -- (B5);
            \draw[->] (A) -- (B6);
            
            % Arrows from second to third row
            \draw[->] (B1) -- (C1);
            \draw[->] (B1) -- (C2);
            
            \draw[->] (B2) -- (C1);
            \draw[->] (B2) -- (C3);
            
            \draw[->] (B3) -- (C1);
            \draw[->] (B3) -- (C4);
            
            \draw[->] (B4) -- (C2);
            \draw[->] (B4) -- (C3);
            
            \draw[->] (B5) -- (C2);
            \draw[->] (B5) -- (C4);
            
            \draw[->] (B6) -- (C3);
            \draw[->] (B6) -- (C4);
            
            % Arrows from third row to bottom
            \draw[->] (C1) -- (D);
            \draw[->] (C2) -- (D);
            \draw[->] (C3) -- (D);
            \draw[->] (C4) -- (D);
            
            \end{tikzpicture}
        }
        \caption{\ $U_{3,4}$.} \label{fig:M2}
    \end{figure}
    \end{minipage}%
    \end{center}
    
\end{block}

\vspace{-5mm}
\begin{block}{Chow rings of matroids}
%\begin{center}
%\begin{minipage}[t]{0.5\textwidth}
\kern0pt
\raggedright
The Chow ring of a matroid was first introduced by Feichtner and Yuzvinsky in \cite{Feichtner_2004}.
Adiprasito, Huh and Katz \cite{adiprasito-2018} use this ring to prove
the Heron--Rota--Welsh conjecture: the sequence of absolute values of the
coefficients of the characteristic polynomial of a matroid is log-concave.

For a loopless matroid $\M$ on $[n]$,
\textbf{the Chow ring} $\uCH(\M)$ is defined as:
\[\uCH(\M):=\mathbb{Q}\left[\{x_F\}_{F\in \cL(\M)\setminus\{\varnothing\}}\right]/(I+J)\]
where $I$ is the ideal $\langle x_Fx_G:\, F,\, G \text{ are incomparable}\rangle$
and $J$ is the ideal $\langle \sum_i{x_F}:\, F\ni i\rangle$ for $1\leq i\leq n$. The ring is graded and has a basis given
by the following \textbf{$\FY$-monomials}:
\begin{align*}
    x_{F_1}^{m_1}x_{F_2}^{m_2}\cdots x_{F_k}^{m_k}\ :&\
    \varnothing = F_0\subset F_1\subset\cdots\subset F_k;\\
    &\ 0\leq m_i\leq \rk(F_i)-\rk(F_{i-1})-1.
\end{align*}

\vspace{2mm}
We denote by $\FY^i$ the set of degree $i$ $\FY$-monomials.
The restriction on the exponents $m_i$ of $x_{F_i}$ ensures that there are exactly $\rk(\M)$ graded pieces.
The \textbf{(non-equivariant) Chow polynomial} $\uH_\M(x)$ is defined as the Hilbert series of $\uCH(\M)$:
\[\uH_{\M}(x) = a_0 + a_1x+\cdots + a_{\rk(\M)-1}x^{\rk(\M)-1},\]
where $a_i = \dim \uCH^i(\M)$.

\end{block}

\end{column} % End of the first column

\begin{column}{\twocolwid} % The second col

\begin{block}{Examples of Chow rings of matroids}

    \begin{itemize}
        \item The Chow ring of $\M(K_4)$ has basis given by the $\FY$-monomials
        \(\FY^0 = \{1\}\), \(\FY^1 = \{x_{124},
        x_{135},x_{25},x_{16},x_{34},x_{236},x_{456},x_{1234}\}\), \(\FY^2 = \{x^2_{1234}\}.\)
        Thus, $\dim \uCH^0(\M) = 1$, $\dim \uCH^1(\M) = 8$ and $\dim \uCH^2(\M) = 1$.

        \item THe Chow ring of $U_{3,4}$ has basis given by
        \(\FY^0 = \{1\}\), \(\FY^1 = \{x_{12},
        x_{13},x_{14},x_{23},x_{24},x_{34},x_{1234}\}\), \(\FY^2 = \{x^2_{1234}\}.\)
        Thus, $\dim \uCH^0(\M) = 1$, $\dim \uCH^1(\M) = 7$ and $\dim \uCH^2(\M) = 1$.
    \end{itemize}

\end{block}

\vspace*{-5mm}
\begin{block}{Equivariant Chow Polynomial}
For a matroid $\M$ with an action of a group $G$, there is an
induced action on the Chow ring of $\M$. It can be shown that
$G$ acts on each graded piece of $\uCH(\M)$ separately by permuting
the $\FY$-monomials of that degree.

\begin{alertblock}{Theorem [Angarone--Nathanson--Reiner\cite{angarone2024chowringsmatroidspermutation}}]
    Let $\M$ be a simple matroid of rank $r+1$ with $G$ a group of automorphisms of $\M$. Then there exist
    \begin{itemize}
        \item $G$-equivariant bijections $\pi: \FY^k\to \FY^{r-k}$ for $k\leq r/2$, and
        \item $G$-equivariant injections $\lambda: \FY^k\to \FY^{k+1}$ for $k<r/2$.
    \end{itemize}
    \end{alertblock}

The
\textbf{equivariant Chow polynomial}
$\uH^G_\M(x)\in\VRep_G[x]$ is defined as:
\[\uH_\M^G(x) = P(\FY^0)+P(\FY^1)x+\cdots P(\FY^{\rk(\M)-1})x^{\rk(\M)-1}\]

where $P(\FY^i)$ denotes the permutation representation of $G$ on the
set $\FY^i$ of degree $i$ $\FY$-monomials.

\end{block}
\begin{block}{Examples of equivariant Chow polynomials}
    \begin{itemize}
        \item For $\mathfrak{S}_4 \curvearrowright \M(K_4)$, the equivariant Chow polynomial is:
        \[\uH_\M^G(x) = V_{(4)}+(V_{(4)}^{\oplus 3}\oplus V_{(3,1)}\oplus V_{(2,2)})x + V_{(4)}x^2\]
        where $V_\lambda$ denotes the irreducible representation of $\mathfrak{S}_4$ corresponding to the
        partition $\lambda$.

        \item For $\mathfrak{S}_4 \curvearrowright U_{3,4}$: the polynomial is
            \[V_{(4)}+(V_{(4)}^{\oplus 2}\oplus V_{(3,1)}\oplus V_{(2,2)})x + V_{(4)}x^2.\]

        \item For $\mathfrak{S}_4 \curvearrowright U_{4,4}$: the polynomial is
            \[V_{(4)}+(V_{(4)}^{\oplus 3}\oplus V_{(3,1)}^{\oplus 2}\oplus V_{(2,2)})x +
            (V_{(4)}^{\oplus 3}\oplus V_{(3,1)}^{\oplus 2}\oplus V_{(2,2)})x^2 + V_{(4)}x^3.\]
    \end{itemize}
\end{block}
\end{column} % End of the second column

\begin{column}{\threecolwid} % The third col

    In these examples, we can see the $G$-equivariant bijections and injections claimed in the
    previous theorem.

    \vspace*{5mm}
\begin{alertblock}{Theorem [Nepal]}
    There is a unique way to assign to each loopless matroid $\M$ a polynomial $\uH^G_\M(x)\in \VRep_G[x]$
    such that the following conditions hold:
    \begin{enumerate}
        \item If $\rk(M) = 0$, then $\uH_\M (x) = 1_G$.
        \item For every matroid M, the following recursion holds:
            \[\uH^G_\M(x)=\sum_{[F]\in\cL(M)/G}{\Ind_{G_F}^G\left(\overline{\chi}^{G_F}_{\M|_F}(x)\otimes \uH^{G_F}_{\M/F}(x)\right)}.\]
    \end{enumerate}
\end{alertblock}
\vspace*{-10mm}
This theorem relies on results of Liao \cite{liao2024chowringsaugmentedchow} and equivariant versions of results in \cite{Ferroni_2024}.
Future work on equivariant Chow polynomials includes:
\begin{itemize}
    \item Recover formulas for uniform matroids in \cite{liao2024chowringsaugmentedchow} using the recursion.
    \item Find formulas for braid matroids and thagomizer matroids.
    \item Find similar recursion for other building sets.
\end{itemize}

\begin{alertblock}{Acknowledgements}
I sincerely thank Dr. Jacob P. Matherne for his guidance, support and
encouragement on this project.
\end{alertblock}
\vspace{-7mm}
\begin{alertblock}{References}

\printbibliography
\end{alertblock}
\end{column} % End of the third column

\end{columns} % End of all the columns in the poster

\end{frame} % End of the enclosing frame

\end{document}
