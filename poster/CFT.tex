%%%%%%%%%%%%%%%%%%%%%%%%%%%%%%%%%%%%%%%%%
% Jacobs Landscape Poster
% LaTeX Template
% Version 1.1 (14/06/14)
%
% Created by:
% Computational Physics and Biophysics Group, Jacobs University
% https://teamwork.jacobs-university.de:8443/confluence/display/CoPandBiG/LaTeX+Poster
% 
% Further modified by:
% Nathaniel Johnston (nathaniel@njohnston.ca)
%
% This template has been downloaded from:
% http://www.LaTeXTemplates.com
%
% License:
% CC BY-NC-SA 3.0 (http://creativecommons.org/licenses/by-nc-sa/3.0/)
%
%%%%%%%%%%%%%%%%%%%%%%%%%%%%%%%%%%%%%%%%%

%----------------------------------------------------------------------------------------
%	PACKAGES AND OTHER DOCUMENT CONFIGURATIONS
%----------------------------------------------------------------------------------------

\documentclass[final]{beamer}

\usepackage[scale=1.24]{beamerposter} % Use the beamerposter package for laying out the poster


\usetheme{confposter} % Use the confposter theme supplied with this template

\setbeamercolor{block title}{fg=ngreen,bg=white} % Colors of the block titles
\setbeamercolor{block body}{fg=black,bg=white} % Colors of the body of blocks
\setbeamercolor{block alerted title}{fg=white,bg=dblue!70} % Colors of the highlighted block titles
\setbeamercolor{block alerted body}{fg=black,bg=dblue!10} % Colors of the body of highlighted blocks
% Many more colors are available for use in beamerthemeconfposter.sty

%-----------------------------------------------------------
% Define the column widths and overall poster size
% To set effective sepwid, onecolwid and twocolwid values, first choose how many columns you want and how much separation you want between columns
% In this template, the separation width chosen is 0.024 of the paper width and a 4-column layout
% onecolwid should therefore be (1-(# of columns+1)*sepwid)/# of columns e.g. (1-(4+1)*0.024)/4 = 0.22
% Set twocolwid to be (2*onecolwid)+sepwid = 0.464
% Set threecolwid to be (3*onecolwid)+2*sepwid = 0.708

\newlength{\sepwid}
\newlength{\onecolwid}
\newlength{\twocolwid}
\newlength{\threecolwid}
\setlength{\paperwidth}{48in} % A0 width: 46.8in
\setlength{\paperheight}{36in} % A0 height: 33.1in
\setlength{\sepwid}{0.003\paperwidth} % Separation width (white space) between columns
\setlength{\onecolwid}{0.31\paperwidth} % Width of one column
\setlength{\twocolwid}{0.28\paperwidth} % Width of two columns
\setlength{\threecolwid}{0.31\paperwidth} % Width of three columns
\setlength{\topmargin}{-0.6in} % Reduce the top margin size
%-----------------------------------------------------------

\usepackage{graphicx}  % Required for including images

\usepackage{booktabs} % Top and bottom rules for tables

\usepackage{tikz}
\usepackage{tikz-cd}

%\usepackage[utf8]{inputenc}

\usetikzlibrary{matrix,shapes,arrows,positioning,chains,decorations.pathmorphing,snakes,calc}

\DeclareMathOperator{\rk}{rk}
\DeclareMathOperator{\cusp}{cusp}
\DeclareMathOperator{\Rel}{Rel}
\DeclareMathOperator{\Poin}{Poin}
\DeclareMathOperator{\Tor}{Tor}
\DeclareMathOperator{\Ext}{Ext}

\newcommand{\M}{\mathsf{M}}
\newcommand{\grp}{\mathsf{G}}
\newcommand{\stab}{\mathsf{Stab}}
%\newcommand{\B}{\mathsf{B}} let us only write \U_{n,n} for boolean matroids
\newcommand{\N}{\mathsf{N}}
\newcommand{\U}{\mathsf{U}}
\newcommand{\cI}{\mathcal{I}}
\newcommand{\symm}{\mathfrak{S}}
\newcommand{\A}{\mathrm{A}}
\newcommand{\Q}{\mathbb{Q}}
\newcommand{\R}{\mathbb{R}}
\newcommand{\Z}{\mathbb{Z}}

\newcommand{\LL}{\mathsf{\Lambda}}
\newcommand{\Hilb}{\operatorname{Hilb}}
\newcommand{\rank}{\operatorname{rk}}
\newcommand{\cl}{\operatorname{cl}}
\renewcommand{\H}{\mathrm{H}}
\newcommand{\CH}{\mathrm{CH}}
\newcommand{\aug}{\operatorname{aug}}
\newcommand{\IH}{\mathrm{IH}}
\newcommand{\uH}{\underline{\mathrm{H}}}
\newcommand{\uCH}{\underline{\mathrm{CH}}}
\newcommand\size[1]{|#1|}
\DeclareMathOperator{\Pal}{\mathrm{Pal}}
\DeclareMathOperator{\Poly}{\mathrm{Poly}}

\newcommand{\cL}{\mathcal{L}}
\newcommand{\Rep}{\operatorname{Rep}}
\newcommand{\gr}{\operatorname{gr}}
\newcommand{\VRep}{\operatorname{VRep}}
\newcommand{\Ind}{\operatorname{Ind}}
\newcommand{\Res}{\operatorname{Res}}
\newcommand{\Aut}{\operatorname{Aut}}
\newcommand{\col}{\operatorname{col}}
\newcommand{\asc}{\operatorname{asc}}
\newcommand{\bad}{\operatorname{bad}}
\newcommand{\des}{\operatorname{des}}

\newcommand{\Ver}{7}
\newcommand{\Vera}{2}
\newcommand{\Hor}{15}
\newcommand{\Hora}{20}
\newcommand{\Horb}{3}
\newcommand{\Horc}{4.5}
\newcommand{\Hord}{6.3}
\newcommand{\Hore}{11}
\newcommand{\Horf}{7.7}
\newcommand{\Tone}{1.85}

\newcommand{\CC}{\mathbb{C}}
\newcommand{\TT}{\mathbb{T}}

\newcommand{\fg}{\mathfrak{g}}

\newcommand{\cF}{\mathcal{F}}
\newcommand{\cU}{\mathcal{U}}
\newcommand{\cO}{\mathcal{O}}

\newcommand{\bw}{\mathbf{w}}
\newcommand{\bP}{\mathbf{P}}
\newcommand{\be}{\mathbf{e}}

\newcommand{\sT}{\mathsf{T}}

\newcommand{\Db}{D^{\mathrm{b}}}
\newcommand{\GL}{\mathbf{GL}}
\newcommand{\IC}{\mathrm{IC}}



\newcounter{sarrow}
\newcommand\xrsquigarrow[1]{%
\stepcounter{sarrow}%
\begin{tikzpicture}[decoration=snake]
\node (\thesarrow) {\strut#1};
\draw[->,decorate] (\thesarrow.south west) -- (\thesarrow.south east);
\end{tikzpicture}%
}

\tikzset{
    ncbar angle/.initial=90,
    ncbar/.style={
        to path=(\tikztostart)
        -- ($(\tikztostart)!#1!\pgfkeysvalueof{/tikz/ncbar angle}:(\tikztotarget)$)
        -- ($(\tikztotarget)!($(\tikztostart)!#1!\pgfkeysvalueof{/tikz/ncbar angle}:(\tikztotarget)$)!\pgfkeysvalueof{/tikz/ncbar angle}:(\tikztostart)$)
        -- (\tikztotarget)
    },
    ncbar/.default=0.5cm,
}

\tikzset{square left brace/.style={ncbar=0.5cm}}
\tikzset{square right brace/.style={ncbar=-0.5cm}}

\tikzset{round left paren/.style={ncbar=0.5cm,out=120,in=-120}}
\tikzset{round right paren/.style={ncbar=0.5cm,out=60,in=-60}}

%----------------------------------------------------------------------------------------
%	TITLE SECTION 
%----------------------------------------------------------------------------------------

\title{Equivariant Chow Polynomials of Matroids} % Poster title

\author{Nutan Nepal} % Author(s)

\institute{North Carolina State University} % Institution(s)

%----------------------------------------------------------------------------------------

\begin{document}
\tikzstyle{startstop} = [rectangle, rounded corners, minimum width=3cm, minimum height=1cm,text centered, draw=black, fill=dblue!10]
%\tikzstyle{startstop} = [rectangle, rounded corners, minimum width=3cm, minimum height=1cm,text centered, draw=black, fill=ngreen]
\tikzstyle{io} = [trapezium, trapezium left angle=70, trapezium right angle=110, minimum width=3cm, minimum height=1cm, text centered, draw=black, fill=dblue!40]
\tikzstyle{process} = [rectangle, minimum width=3cm, minimum height=1cm, text centered, draw=black, fill=orange!30]
\tikzstyle{decision} = [diamond, minimum width=3cm, minimum height=1cm, text centered, draw=black, fill=green!30]
\tikzstyle{arrow} = [thick,->,>=stealth]



\addtobeamertemplate{block end}{}{\vspace*{2ex}} % White space under blocks
\addtobeamertemplate{block alerted end}{}{\vspace*{2ex}} % White space under highlighted (alert) blocks

\setlength{\belowcaptionskip}{2ex} % White space under figures
\setlength\belowdisplayshortskip{2ex} % White space under equations

\begin{frame}[t] % The whole poster is enclosed in one beamer frame

\begin{columns}[t] % The whole poster consists of three major columns, the second of which is split into two columns twice - the [t] option aligns each column's content to the top

\begin{column}{\sepwid}\end{column} % Empty spacer column

\begin{column}{\onecolwid} % The first column

\begin{block}{Goal}
Define the equivariant Chow polynomial $\uH_{\M}^G(x) \in \VRep_G[x]$ of a matroid $\M$ :

\end{block}

\vspace{-5mm}
\begin{block}{Overview}
\[
    \begin{tikzpicture}[scale=3]

        % Level 1 (topmost node)
        \node at (0, 4) (A) {$\bullet$};
        
        % Level 2 (second row of nodes)
        \node at (-6, 2) (B1) {$\bullet$};
        \node at (-4, 2) (B2) {$\bullet$};
        \node at (-2, 2) (B3) {$\bullet$};
        \node at (0, 2) (B4) {$\bullet$};
        \node at (2, 2) (B5) {$\bullet$};
        \node at (4, 2) (B6) {$\bullet$};
        \node at (6, 2) (B7) {$\bullet$};
        
        % Level 3 (third row of nodes)
        \node at (-5, 0) (C1) {$\bullet$};
        \node at (-3, 0) (C2) {$\bullet$};
        \node at (-1, 0) (C3) {$\bullet$};
        \node at (1, 0) (C4) {$\bullet$};
        \node at (3, 0) (C5) {$\bullet$};
        \node at (5, 0) (C6) {$\bullet$};
        
        % Level 4 (bottommost node)
        \node at (0, -2) (D) {$\bullet$};
        
        % Arrows from top to second row
        \draw[->] (A) -- (B1);
        \draw[->] (A) -- (B2);
        \draw[->] (A) -- (B3);
        \draw[->] (A) -- (B4);
        \draw[->] (A) -- (B5);
        \draw[->] (A) -- (B6);
        \draw[->] (A) -- (B7);
        
        % Arrows from second to third row
        \draw[->] (B1) -- (C1);
        \draw[->] (B1) -- (C2);
        \draw[->] (B1) -- (C4);
        
        \draw[->] (B2) -- (C1);
        \draw[->] (B2) -- (C3);
        \draw[->] (B2) -- (C5);
        
        \draw[->] (B3) -- (C2);
        \draw[->] (B3) -- (C5);
        
        \draw[->] (B4) -- (C1);
        \draw[->] (B4) -- (C6);
        
        \draw[->] (B5) -- (C3);
        \draw[->] (B5) -- (C4);
        
        \draw[->] (B6) -- (C2);
        \draw[->] (B6) -- (C3);
        \draw[->] (B6) -- (C6);
        
        \draw[->] (B7) -- (C4);
        \draw[->] (B7) -- (C5);
        \draw[->] (B7) -- (C6);
        
        % Arrows from third row to bottom
        \draw[->] (C1) -- (D);
        \draw[->] (C2) -- (D);
        \draw[->] (C3) -- (D);
        \draw[->] (C4) -- (D);
        \draw[->] (C5) -- (D);
        \draw[->] (C6) -- (D);
        
        \end{tikzpicture}
\]
\end{block}

\vspace{-5mm}
\begin{block}{The Chow Ring}
%\begin{center}
%\begin{minipage}[t]{0.5\textwidth}
\kern0pt
\raggedright
For a matroid $\M$ with flats $F_1,\ldots, F_m$, \textbf{the Chow ring} $\uCH_{\M}$ can be defined as a graded
$\mathbb{Z}$-module generated by the following \textbf{FY-monomials}:

\[x_{F_1}^{m_1}x_{F_2}^{m_2}\cdots x_{F_k}^{m_k}\ |\
\varnothing\subset F_1\subset\cdots\subset F_k,\ 0\leq m_i\leq \rk(F_i)-\rk(F_{i-1})-1.\]

\vspace{2cm}
The restriction on the exponents $m_i$ of $x_{F_i}$ ensures that there are exactly $\rk(M)$ graded pieces.
The \textbf{(non equivariant) Chow polynomial} $\uH_\M$ is defined as:

\[\uH_{\M}(x) = a_0 + a_1x+\cdots a_{\rk(M)-1}x^{\rk(M)-1}\]
where $a_i$ is the rank of degree $i$ piece in $\uCH_\M$.
\vspace{10mm}

For the braid matroid $K_4$ depicted above, the Chow polynomial is $1+8x+x^2$.

\end{block}
\vspace{-6mm}
\begin{alertblock}{Theorem [Adiprasito-Huh-Katz]}
    
The sequence \((a_0,a_1,\ldots, a_{\rk(M)-1})\) is log-concave.
\end{alertblock}

\vspace{-2mm}

\end{column} % End of the first column

\begin{column}{\twocolwid} % The second col

\begin{block}{Equivariant Chow Polynomial}

For a matroid $\M$ with an action of a group $G$, there is an
induced action on the Chow ring of $\M$. It can be shown that
$G$ acts on each graded piece of $\uCH_\M$ separately by permuting
the FY-monomials of that degree. The
\textbf{equivariant Chow polynomial}
$\H^G_\M(x)\in\VRep_G[x]$ is defined as:

\[\H_\M^G(x) = P(A_0)+P(A_1)x+\cdots P(A_{\rk(M)-1})x^{\rk(M)-1}\]

where $P(A_i)$ denotes the permutation representation of $G$ on the
set $A_i$ of degree $i$ FY-monomials.

\end{block}


\begin{block}{Example of bijection}
\end{block}

\vspace{-11mm}
\begin{alertblock}{Theorem [Angarone-Nathanson-Reiner]}

\end{alertblock}

\end{column} % End of the second column

\begin{column}{\threecolwid} % The third col

\begin{block}{Definition of $\tau_j$}
Define $\tau_j : \bP(\bw) \rightarrow \bP(\bw + \be_1 + \ldots + \be_j)$ by:
\begin{itemize}
\item Add $1$ as far down the $j^{\mathrm{th}}$ chute as possible, drawing an impassable vertical line there.
\item Repeat for chutes $j-1,\ldots, 1$ not crossing lines.
\end{itemize}
\end{block}

\vspace{-8mm}
\begin{block}{Example of CFT}
\begin{center}

\end{center}
\end{block}


\begin{block}{Complete CFT example}
\begin{center}
\end{center}
\end{block}

\vspace{-7mm}
\begin{alertblock}{Main conjecture (proof in progress)}
The bijection $\sT$ determines $\TT$ on simple perverse sheaves; that is, $\TT(\IC(\cO_{\lambda})) = \IC(\cO_{\sT(\lambda)})$.    

\end{alertblock}

\end{column} % End of the third column

\end{columns} % End of all the columns in the poster

\end{frame} % End of the enclosing frame

\end{document}
