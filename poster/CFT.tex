%%%%%%%%%%%%%%%%%%%%%%%%%%%%%%%%%%%%%%%%%
% Jacobs Landscape Poster
% LaTeX Template
% Version 1.1 (14/06/14)
%
% Created by:
% Computational Physics and Biophysics Group, Jacobs University
% https://teamwork.jacobs-university.de:8443/confluence/display/CoPandBiG/LaTeX+Poster
% 
% Further modified by:
% Nathaniel Johnston (nathaniel@njohnston.ca)
%
% This template has been downloaded from:
% http://www.LaTeXTemplates.com
%
% License:
% CC BY-NC-SA 3.0 (http://creativecommons.org/licenses/by-nc-sa/3.0/)
%
%%%%%%%%%%%%%%%%%%%%%%%%%%%%%%%%%%%%%%%%%

%----------------------------------------------------------------------------------------
%	PACKAGES AND OTHER DOCUMENT CONFIGURATIONS
%----------------------------------------------------------------------------------------

\documentclass[final]{beamer}

\usepackage[scale=1.24]{beamerposter} % Use the beamerposter package for laying out the poster


\usetheme{confposter} % Use the confposter theme supplied with this template

\setbeamercolor{block title}{fg=ngreen,bg=white} % Colors of the block titles
\setbeamercolor{block body}{fg=black,bg=white} % Colors of the body of blocks
\setbeamercolor{block alerted title}{fg=white,bg=dblue!70} % Colors of the highlighted block titles
\setbeamercolor{block alerted body}{fg=black,bg=dblue!10} % Colors of the body of highlighted blocks
% Many more colors are available for use in beamerthemeconfposter.sty

%-----------------------------------------------------------
% Define the column widths and overall poster size
% To set effective sepwid, onecolwid and twocolwid values, first choose how many columns you want and how much separation you want between columns
% In this template, the separation width chosen is 0.024 of the paper width and a 4-column layout
% onecolwid should therefore be (1-(# of columns+1)*sepwid)/# of columns e.g. (1-(4+1)*0.024)/4 = 0.22
% Set twocolwid to be (2*onecolwid)+sepwid = 0.464
% Set threecolwid to be (3*onecolwid)+2*sepwid = 0.708

\newlength{\sepwid}
\newlength{\onecolwid}
\newlength{\twocolwid}
\newlength{\threecolwid}
\setlength{\paperwidth}{48in} % A0 width: 46.8in
\setlength{\paperheight}{36in} % A0 height: 33.1in
\setlength{\sepwid}{0.003\paperwidth} % Separation width (white space) between columns
\setlength{\onecolwid}{0.31\paperwidth} % Width of one column
\setlength{\twocolwid}{0.28\paperwidth} % Width of two columns
\setlength{\threecolwid}{0.31\paperwidth} % Width of three columns
\setlength{\topmargin}{-0.6in} % Reduce the top margin size
%-----------------------------------------------------------

\usepackage{graphicx}  % Required for including images

\usepackage{booktabs} % Top and bottom rules for tables

\usepackage{tikz}
\usepackage{tikz-cd}

%\usepackage[utf8]{inputenc}

\usetikzlibrary{matrix,shapes,arrows,positioning,chains,decorations.pathmorphing,snakes,calc}

\DeclareMathOperator{\rk}{rk}
\DeclareMathOperator{\cusp}{cusp}
\DeclareMathOperator{\Rel}{Rel}
\DeclareMathOperator{\Poin}{Poin}
\DeclareMathOperator{\Tor}{Tor}
\DeclareMathOperator{\Ext}{Ext}

\newcommand{\M}{\mathsf{M}}
\newcommand{\grp}{\mathsf{G}}
\newcommand{\stab}{\mathsf{Stab}}
%\newcommand{\B}{\mathsf{B}} let us only write \U_{n,n} for boolean matroids
\newcommand{\N}{\mathsf{N}}
\newcommand{\U}{\mathsf{U}}
\newcommand{\cI}{\mathcal{I}}
\newcommand{\symm}{\mathfrak{S}}
\newcommand{\A}{\mathrm{A}}
\newcommand{\Q}{\mathbb{Q}}
\newcommand{\R}{\mathbb{R}}
\newcommand{\Z}{\mathbb{Z}}

\newcommand{\LL}{\mathsf{\Lambda}}
\newcommand{\Hilb}{\operatorname{Hilb}}
\newcommand{\rank}{\operatorname{rk}}
\newcommand{\cl}{\operatorname{cl}}
\renewcommand{\H}{\mathrm{H}}
\newcommand{\CH}{\mathrm{CH}}
\newcommand{\aug}{\operatorname{aug}}
\newcommand{\IH}{\mathrm{IH}}
\newcommand{\uH}{\underline{\mathrm{H}}}
\newcommand{\uCH}{\underline{\mathrm{CH}}}
\newcommand\size[1]{|#1|}
\DeclareMathOperator{\Pal}{\mathrm{Pal}}
\DeclareMathOperator{\Poly}{\mathrm{Poly}}

\newcommand{\cL}{\mathcal{L}}
\newcommand{\Rep}{\operatorname{Rep}}
\newcommand{\gr}{\operatorname{gr}}
\newcommand{\VRep}{\operatorname{VRep}}
\newcommand{\Ind}{\operatorname{Ind}}
\newcommand{\Res}{\operatorname{Res}}
\newcommand{\Aut}{\operatorname{Aut}}
\newcommand{\col}{\operatorname{col}}
\newcommand{\asc}{\operatorname{asc}}
\newcommand{\bad}{\operatorname{bad}}
\newcommand{\des}{\operatorname{des}}

\newcommand{\Ver}{7}
\newcommand{\Vera}{2}
\newcommand{\Hor}{15}
\newcommand{\Hora}{20}
\newcommand{\Horb}{3}
\newcommand{\Horc}{4.5}
\newcommand{\Hord}{6.3}
\newcommand{\Hore}{11}
\newcommand{\Horf}{7.7}
\newcommand{\Tone}{1.85}

\newcommand{\CC}{\mathbb{C}}
\newcommand{\TT}{\mathbb{T}}

\newcommand{\fg}{\mathfrak{g}}

\newcommand{\cF}{\mathcal{F}}
\newcommand{\cU}{\mathcal{U}}
\newcommand{\cO}{\mathcal{O}}

\newcommand{\bw}{\mathbf{w}}
\newcommand{\bP}{\mathbf{P}}
\newcommand{\be}{\mathbf{e}}

\newcommand{\sT}{\mathsf{T}}

\newcommand{\Db}{D^{\mathrm{b}}}
\newcommand{\GL}{\mathbf{GL}}
\newcommand{\IC}{\mathrm{IC}}



\newcounter{sarrow}
\newcommand\xrsquigarrow[1]{%
\stepcounter{sarrow}%
\begin{tikzpicture}[decoration=snake]
\node (\thesarrow) {\strut#1};
\draw[->,decorate] (\thesarrow.south west) -- (\thesarrow.south east);
\end{tikzpicture}%
}

\tikzset{
    ncbar angle/.initial=90,
    ncbar/.style={
        to path=(\tikztostart)
        -- ($(\tikztostart)!#1!\pgfkeysvalueof{/tikz/ncbar angle}:(\tikztotarget)$)
        -- ($(\tikztotarget)!($(\tikztostart)!#1!\pgfkeysvalueof{/tikz/ncbar angle}:(\tikztotarget)$)!\pgfkeysvalueof{/tikz/ncbar angle}:(\tikztostart)$)
        -- (\tikztotarget)
    },
    ncbar/.default=0.5cm,
}

\tikzset{square left brace/.style={ncbar=0.5cm}}
\tikzset{square right brace/.style={ncbar=-0.5cm}}

\tikzset{round left paren/.style={ncbar=0.5cm,out=120,in=-120}}
\tikzset{round right paren/.style={ncbar=0.5cm,out=60,in=-60}}

%----------------------------------------------------------------------------------------
%	TITLE SECTION 
%----------------------------------------------------------------------------------------

\title{Equivariant Chow Polynomials of Matroids} % Poster title

\author{Nutan Nepal} % Author(s)

\institute{North Carolina State University} % Institution(s)

%----------------------------------------------------------------------------------------

\begin{document}
\tikzstyle{startstop} = [rectangle, rounded corners, minimum width=3cm, minimum height=1cm,text centered, draw=black, fill=dblue!10]
%\tikzstyle{startstop} = [rectangle, rounded corners, minimum width=3cm, minimum height=1cm,text centered, draw=black, fill=ngreen]
\tikzstyle{io} = [trapezium, trapezium left angle=70, trapezium right angle=110, minimum width=3cm, minimum height=1cm, text centered, draw=black, fill=dblue!40]
\tikzstyle{process} = [rectangle, minimum width=3cm, minimum height=1cm, text centered, draw=black, fill=orange!30]
\tikzstyle{decision} = [diamond, minimum width=3cm, minimum height=1cm, text centered, draw=black, fill=green!30]
\tikzstyle{arrow} = [thick,->,>=stealth]



\addtobeamertemplate{block end}{}{\vspace*{2ex}} % White space under blocks
\addtobeamertemplate{block alerted end}{}{\vspace*{2ex}} % White space under highlighted (alert) blocks

\setlength{\belowcaptionskip}{2ex} % White space under figures
\setlength\belowdisplayshortskip{2ex} % White space under equations

\begin{frame}[t] % The whole poster is enclosed in one beamer frame

\begin{columns}[t] % The whole poster consists of three major columns, the second of which is split into two columns twice - the [t] option aligns each column's content to the top

\begin{column}{\sepwid}\end{column} % Empty spacer column

\begin{column}{\onecolwid} % The first column

\begin{block}{Goal}
Define the equivariant Chow polynomial $\uH_{\M}^G(x) \in \VRep_G[x]$ of a matroid $\M$ :

\end{block}

\vspace{-5mm}
\begin{block}{Overview}
\[
    \begin{tikzpicture}[scale=3]

        % Level 1 (topmost node)
        \node at (0, 4) (A) {$\bullet$};
        
        % Level 2 (second row of nodes)
        \node at (-6, 2) (B1) {$\bullet$};
        \node at (-4, 2) (B2) {$\bullet$};
        \node at (-2, 2) (B3) {$\bullet$};
        \node at (0, 2) (B4) {$\bullet$};
        \node at (2, 2) (B5) {$\bullet$};
        \node at (4, 2) (B6) {$\bullet$};
        \node at (6, 2) (B7) {$\bullet$};
        
        % Level 3 (third row of nodes)
        \node at (-5, 0) (C1) {$\bullet$};
        \node at (-3, 0) (C2) {$\bullet$};
        \node at (-1, 0) (C3) {$\bullet$};
        \node at (1, 0) (C4) {$\bullet$};
        \node at (3, 0) (C5) {$\bullet$};
        \node at (5, 0) (C6) {$\bullet$};
        
        % Level 4 (bottommost node)
        \node at (0, -2) (D) {$\bullet$};
        
        % Arrows from top to second row
        \draw[->] (A) -- (B1);
        \draw[->] (A) -- (B2);
        \draw[->] (A) -- (B3);
        \draw[->] (A) -- (B4);
        \draw[->] (A) -- (B5);
        \draw[->] (A) -- (B6);
        \draw[->] (A) -- (B7);
        
        % Arrows from second to third row
        \draw[->] (B1) -- (C1);
        \draw[->] (B1) -- (C2);
        \draw[->] (B1) -- (C4);
        
        \draw[->] (B2) -- (C1);
        \draw[->] (B2) -- (C3);
        \draw[->] (B2) -- (C5);
        
        \draw[->] (B3) -- (C2);
        \draw[->] (B3) -- (C5);
        
        \draw[->] (B4) -- (C1);
        \draw[->] (B4) -- (C6);
        
        \draw[->] (B5) -- (C3);
        \draw[->] (B5) -- (C4);
        
        \draw[->] (B6) -- (C2);
        \draw[->] (B6) -- (C3);
        \draw[->] (B6) -- (C6);
        
        \draw[->] (B7) -- (C4);
        \draw[->] (B7) -- (C5);
        \draw[->] (B7) -- (C6);
        
        % Arrows from third row to bottom
        \draw[->] (C1) -- (D);
        \draw[->] (C2) -- (D);
        \draw[->] (C3) -- (D);
        \draw[->] (C4) -- (D);
        \draw[->] (C5) -- (D);
        \draw[->] (C6) -- (D);
        
        \end{tikzpicture}
\]
\end{block}

\vspace{-5mm}
\begin{block}{The Chow Ring}
%\begin{center}
%\begin{minipage}[t]{0.5\textwidth}
\kern0pt
\raggedright
For a matroid $\M$ with flats $F_1,\ldots, F_m$, \textbf{the Chow ring} $\uCH_{\M}$ can be defined as a graded
$\mathbb{Z}$-module generated by the following monomials:

\[x_{F_1}^{m_1}x_{F_2}^{m_2}\cdots x_{F_k}^{m_k}\ |\
\varnothing\subset F_1\subset\cdots\subset F_k,\ 0\leq m_i\leq \rk(F_i)-\rk(F_{i-1})-1.\]
%\end{minipage}%
%\begin{minipage}[t]{0.5\textwidth}
%\kern0pt
%\center
%\raggedleft
%\vspace{.5cm}
%\end{minipage}%
%\end{center}

\vspace{2cm}
The restriction on the exponents $m_i$ of $x_{F_i}$ ensures that there are exactly $\rk(M)$ graded pieces.
The \textbf{(non equivariant) Chow polynomial} $\uH_\M$ is defined as:

\[\uH_{\M}(x) = a_0 + a_1x+\cdots a_{\rk(M)-1}x^{\rk(M)-1}\]
where $a_i$ is the rank of degree $i$ piece in $\uCH_\M$.
\vspace{10mm}

For the braid matroid $K_4$ depicted above, the Chow polynomial is $1+8x+x^2$.

\end{block}

\end{column} % End of the first column

\begin{column}{\twocolwid} % The second col

\begin{block}{The set of triangular arrays $\bP(\bw)$}


\kern0pt
%\center
\raggedright
\vspace{-.5cm}
Define the set $\bP(\bw)$ of triangular arrays of nonnegative integers such that:
\begin{itemize}
\item $\forall j$, the entries in the $j^{\mathrm{th}}$ chute sum to $w_j$.
\item Ladders are weakly decreasing.
\end{itemize}
\end{block}

\vspace{-6mm}
\begin{alertblock}{Theorem [Achar--Kulkarni--M.]}
%\vspace{-10mm}
There is a bijection
\[
\{G(\bw)\text{-orbits in } E(\bw)\} \stackrel{1-1}{\longleftrightarrow} \bP(\bw) = \{\text{certain tri. arrays}\}.
\]
\end{alertblock}

%\begin{block}{Example of bijection}
%\begin{tikzpicture}[scale=1.5]
%\draw(0,0) coordinate (a)--+(0,4) coordinate (b)--+(30:4) coordinate (c) --cycle;
%\draw (0,2.5) -- (30:2.5cm);
%\draw (0,3) -- (30:3cm);
%\draw (0,1.5) -- +(30:2.5cm);
%\draw (0,0.5) -- +(30:3.5cm);
%\end{tikzpicture}

\vspace{-2mm}
\begin{block}{Example of bijection}
\begin{center}
\begin{tikzpicture}
\node (v1) at (0,0) {$\CC$};
\node (v2) at (7,0) {$\CC$};
\node (v3) at (14,0) {$\CC^2$};

\draw[thick, ->] (v1) -- (v2);
\draw[thick, ->] (v2) -- (v3);

\draw [arrow] (v1) -- node[anchor=south] {$0$} (v2);
\draw [arrow] (v2) -- node[anchor=south] {$0$} (v3);
%
\draw (\Hora+0,3) -- (\Hora+5,0) -- (\Hora+0,-3) -- cycle;
\node (11) at (\Hora+.70,1.5) {1};
\node (21) at (\Hora+.70,0) {1};
\node (31) at (\Hora+.70,-1.5) {2};

\node (12) at (\Hora+2.20,.75) {0};
\node (22) at (\Hora+2.20,-.75) {0};

\node (13) at (\Hora+3.70,0) {0};
%
\node (v1) at (0,0-\Ver) {$\CC$};
\node (v2) at (7,0-\Ver) {$\CC$};
\node (v3) at (14,0-\Ver) {$\CC^2$};

\draw[thick, ->] (v1) -- (v2);
\draw[thick, ->] (v2) -- (v3);

\draw [arrow] (v1) -- node[anchor=south] {$\mathrm{rank}\ 1$} (v2);
\draw [arrow] (v2) -- node[anchor=south] {$\left(\begin{array}{ll}0\\0\end{array}\right)$} (v3);
%
\draw (\Hora+0,3-\Ver) -- (\Hora+5,0-\Ver) -- (\Hora+0,-3-\Ver) -- cycle;
\node (11) at (\Hora+.70,1.5-\Ver) {0};
\node (21) at (\Hora+.70,0-\Ver) {1};
\node (31) at (\Hora+.70,-1.5-\Ver) {2};

\node (12) at (\Hora+2.20,.75-\Ver) {1};
\node (22) at (\Hora+2.20,-.75-\Ver) {0};

\node (13) at (\Hora+3.70,0-\Ver) {0};
%
\node (v1) at (0,0-2*\Ver) {$\CC$};
\node (v2) at (7,0-2*\Ver) {$\CC$};
\node (v3) at (14,0-2*\Ver) {$\CC^2$};

\draw[thick, ->] (v1) -- (v2);
\draw[thick, ->] (v2) -- (v3);

\draw [arrow] (v1) -- node[anchor=south] {$0$} (v2);
\draw [arrow] (v2) -- node[anchor=south] {$\mathrm{rank}\ 1$} (v3);
%
\draw (\Hora+0,3-2*\Ver) -- (\Hora+5,0-2*\Ver) -- (\Hora+0,-3-2*\Ver) -- cycle;
\node (11) at (\Hora+.70,1.5-2*\Ver) {1};
\node (21) at (\Hora+.70,0-2*\Ver) {0};
\node (31) at (\Hora+.70,-1.5-2*\Ver) {2};

\node (12) at (\Hora+2.20,.75-2*\Ver) {0};
\node (22) at (\Hora+2.20,-.75-2*\Ver) {1};

\node (13) at (\Hora+3.70,0-2*\Ver) {0};
%
\node (v1) at (0,0-3*\Ver) {$\CC$};
\node (v2) at (7,0-3*\Ver) {$\CC$};
\node (v3) at (14,0-3*\Ver) {$\CC^2$};

\draw[thick, ->] (v1) -- (v2);
\draw[thick, ->] (v2) -- (v3);

\draw [arrow] (v1) -- node[anchor=south] {$\mathrm{rank}\ 1$} (v2);
\draw [arrow] (v2) -- node[anchor=south] {$\mathrm{rank}\ 1$} (v3);
%
\draw (\Hora+0,3-3*\Ver) -- (\Hora+5,0-3*\Ver) -- (\Hora+0,-3-3*\Ver) -- cycle;
\node (11) at (\Hora+.70,1.5-3*\Ver) {0};
\node (21) at (\Hora+.70,0-3*\Ver) {0};
\node (31) at (\Hora+.70,-1.5-3*\Ver) {2};

\node (12) at (\Hora+2.20,.75-3*\Ver) {0};
\node (22) at (\Hora+2.20,-.75-3*\Ver) {1};

\node (13) at (\Hora+3.70,0-3*\Ver) {1};
\end{tikzpicture}
\end{center}

%\begin{tikzpicture}
%\node (v1) at (0,7) {};
%\node (v2) at (14,0) {};
%\node (v3) at (0,-7) {};
%
%\draw[thick] (v1) -- (v2);
%\draw[thick] (v2) -- (v3);
%\draw[thick] (v1) -- (v3);
%
%%\draw [arrow] (v1) -- node[anchor=south] {$1$} (v2);
%%\draw [arrow] (v2) -- node[anchor=south] {$\left(\begin{array}{l}2\\5\end{array}\right)$} (v3);
%\end{tikzpicture}
%
%
%%\begin{tikzpicture}
%%\node at (0,0) (a) {A node};
%%\node at (120:1cm) (b) {B node};
%%\path (a) ++(-45:2cm) node (c) {C node}; 
%%\end{tikzpicture}
%
%\begin{tikzpicture}
%\draw(0,0) coordinate (a)--+(0,8) coordinate (b)--+(30:8) coordinate (c) --cycle;
%\node (d) at (0.75,7) {$a_{n,n}$};
%\path (d) ++(-30:3cm) node (e) {$a_{n-1,n}$};
%\path (e) ++(-30:3cm) node (f) {$a_{n-1,n}$};
%\path (f) ++(-30:3cm) node (g) {$a_{1,n}$};
%
%\node (d) at (0.75,1) {$a_{1,1}$};
%\path (d) ++(30:1.1cm) node (e) {$a_{1,2}$};
%\path (e) ++(30:1.25cm) node (f) {$a_{1,3}$};
%
%
%%\draw (0,2.5) -- (30:2.5cm);
%%\draw (0,3) -- (30:3cm);
%%\draw (0,1.5) -- +(30:2.5cm);
%%\draw (0,0.5) -- +(30:3.5cm);
%\end{tikzpicture}
%
%
%\begin{tikzpicture}
%\draw(0,0) coordinate (a)--+(0,8) coordinate (b)--+(30:8) coordinate (c) --cycle;
%\node (d) at (0.75,7) {$a_{n,n}$};
%\path (d) ++(-30:1.1cm) node (e) {$a_{n-1,n}$};
%\path (e) ++(-30:1.25cm) node (f) {$a_{n-1,n}$};
%\path (f) ++(-30:3.5cm) node (g) {$a_{1,n}$};
%
%\node (d) at (0.75,1) {$a_{1,1}$};
%\path (d) ++(30:1.1cm) node (e) {$a_{1,2}$};
%\path (e) ++(30:1.25cm) node (f) {$a_{1,3}$};
%
%
%%\draw (0,2.5) -- (30:2.5cm);
%%\draw (0,3) -- (30:3cm);
%%\draw (0,1.5) -- +(30:2.5cm);
%%\draw (0,0.5) -- +(30:3.5cm);
%\end{tikzpicture}

%change line thickness on simple hyperplanes
%\begin{center}
%\begin{tikzpicture}[scale=1.1]
%  \draw[fill=dblue!11] (\C*0,\C*0)--(\C*1,\C*0)--(\C*.5,\C*.865)--cycle;
%  \draw [line width=1mm] (\C*2,\C*0)--(-\C*2,\C*0);
%%  \draw (2,.865)--(-2,.865);
%\draw (\C*1.5,\C*.865)--(-\C*1.5,\C*.865);
%%  \draw (2, -.865)--(-2,-.865);
%\draw (\C*1.5, -\C*.865)--(-\C*1.5,-\C*.865);
%\draw (-\C*1,\C*1.73)--(\C*1,\C*1.73);
%\draw (-\C*1,-\C*1.73)--(\C*1,-\C*1.73);
%
%  \draw [line width=1mm] (\C*1,\C*1.73)--(-\C*1,-\C*1.73);
%  \draw (\C*0,\C*1.73)--(-\C*1.5,-\C*0.865);
%%  \draw (0,1.73)--(-2,-1.73);
%%  \draw (2,1.73)--(0,-1.73);
%\draw (\C*1.5,\C*.865)--(\C*0,-\C*1.73);
%
%  \draw [line width=1mm] (-\C*1,\C*1.73)--(\C*1,-\C*1.73);
%%  \draw (0,1.73)--(2,-1.73);
%\draw (\C*0,\C*1.73)--(\C*1.5,-\C*.865);
%%  \draw (-2,1.73)--(0,-1.73);
%  \draw (-\C*1.5,\C*.865)--(\C*0,-\C*1.73);
%
%  \draw (-\C*2,\C*0)--(-\C*1,-\C*1.73);
%  \draw (-\C*2,\C*0)--(-\C*1,\C*1.73);
%  \draw (\C*2,\C*0)--(\C*1,-\C*1.73);
%  \draw (\C*2,\C*0)--(\C*1,\C*1.73);
%
%
%  \draw (3*2+\A,0+\B)--(-3*2+\A,0+\B)--(0+\A,3*2.9+\B)--(3*2+\A,0+\B);
%\draw[fill=dblue!11] (3*2+\A,0+\B)--(-3*2+\A,0+\B)--(0+\A,3*2.9+\B)--cycle;
%\draw (0+\A,3*1.2+\B) [fill=black] circle (5pt);\draw (-3*2+\A,0+\B) [fill=black] circle (5pt);\draw (0+\A,0+\B) [fill=black] circle (5pt);
%\draw (0+\A, 3*1.2+\B) node [below] {barycenter};
%\draw (-3*2+\A, 0+\B) node [below] {origin};
%\draw (0+\A, -3*.1+\B) node [below] {$P$};
%\end{tikzpicture}
%\end{center}

%Each point in the fundamental alcove gives a class of periodic filtrations:
%
%\begin{align*}
%\text{origin} & \longleftrightarrow \text{{\bf split filtrations}} \\
%P & \longleftrightarrow \text{``regular, period 2'' filtrations} \\
%\text{barycenter} & \longleftrightarrow \text{{\bf Coxeter} (or {\bf standard Iwahori})} \\
%& \hspace{32mm} \text{{\bf filtrations}}
%\end{align*}
\end{block}

\vspace{-11mm}
\begin{alertblock}{Theorem (Combinatorial Fourier transform) [Achar--Kulkarni--M.]}
There is a bijection
\[
\bP(\bw) \stackrel{\sT}{\longrightarrow} \bP(\bw^*)
\]
defined inductively by
\[
\begin{tikzpicture}
\draw (0,3) -- (5,0) -- (0,-3) -- cycle;
%\node (l11) at (.70,1.5) {1};
\node (l21) at (.70,0) {};
\node (l31) at (.70+.1,-1.5-.27+.1) {\footnotesize$y_{n,1}$};

%\node (l12) at (2.20,.75) {};
\node (l22) at (2.20+.1,-.75-.27+.1) {\rotatebox{30}{$\cdots$}};

\node (l13a) at (3.70+.1,-.27+.1) {\footnotesize$y_{1,n}$};

\node (try) at (-2,0) {$\sT$};

    \draw [thick] (-.2,-3.3) to [round left paren] (-.2,3.3);
        \draw [thick] (4.5,-3.3) to [round right paren] (4.5,3.3);
\draw (0,-1) -- (10/3,1);
\node (yprime) at (10/6-.3,1) {$Y^{\prime}$};

%\draw [thick] (-.5,3.5) to [round left paren] (-.5,-3.5);
\end{tikzpicture}\raisebox{3cm}{$=\tau_{n}^{y_{1,n}} \tau_{n-1}^{y_{2,n-1}-y_{1,n}}\cdots \tau_1^{y_{n,1}-y_{n-1,2}}$}\begin{tikzpicture}
\draw (0,3) -- (5,0) -- (0,-3) -- cycle;
\node (l11) at (.70,1.5) {0};
%\node (l21) at (.70,0) {};
%\node (l31) at (.70,-1.5) {2};

\node (l12) at (2.20,.75) {\rotatebox{-30}{$\cdots$}};
%\node (l22) at (2.20,-.75) {0};

\node (l13a) at (3.70,0) {0};

%\node (try) [left=of l21] {$\sT$};

    \draw [thick] (-.2,-3.3) to [round left paren] (-.2,3.3);
        \draw [thick] (4.5,-3.3) to [round right paren] (4.5,3.3);
\draw (0,1) -- (10/3,-1);
\node (typrime) at (10/6-.3,-1-.05) {\footnotesize$\sT(Y^\prime)$};
%\draw [thick] (-.5,3.5) to [round left paren] (-.5,-3.5);
\end{tikzpicture}\]
where $\sT(a) = a$.

%\begin{tikzpicture}
%    \draw [red, thick] (0,0) to [square left brace] (0,4);
%    \draw [red, thick] (1,0) to [square right brace] (1,4);
%
%
%    \draw [blue, thick] (3,0) to [round left paren] (3,4);
%    \draw [blue, thick] (4,0) to [round right paren] (4,4);
%\end{tikzpicture}
\end{alertblock}

\end{column} % End of the second column

\begin{column}{\threecolwid} % The third col

\begin{block}{Definition of $\tau_j$}
\begin{center}
\begin{minipage}[t]{0.3\textwidth}
\kern0pt
\center
\vspace{-10mm}
\begin{tikzpicture}
\draw (0,3) -- (5,0) -- (0,-3) -- cycle;

\draw (0,0.5) -- (35/12, -5/4);
\draw (0,-0.5) -- (25/12,-7/4);

\draw (2.5,3) edge[out=225,in=95,dblue!70,very thick,->] (1.5, -.25);
\node (jchu) at (3.5,4) {\textcolor{dblue!70}{$j^{\mathrm{th}}$ chute}};
%\node (l11) at (.70,1.5) {1};
%\node (l21) at (.70,0) {1};
%\node (l31) at (.70,-1.5) {2};
%
%\node (l12) at (2.20,.75) {0};
%\node (l22) at (2.20,-.75) {0};
%
%\node (l13a) at (3.70,0) {0};
\end{tikzpicture}
\end{minipage}%
\begin{minipage}[t]{0.7\textwidth}
\kern0pt
%\center
\raggedright
\vspace{-.5cm}
Define $\tau_j : \bP(\bw) \rightarrow \bP(\bw + \be_1 + \ldots + \be_j)$ by:
\begin{itemize}
\item Add $1$ as far down the $j^{\mathrm{th}}$ chute as possible, drawing an impassable vertical line there.
\item Repeat for chutes $j-1,\ldots, 1$ not crossing lines.
\end{itemize}
\end{minipage}%
\end{center}
\end{block}

\vspace{-8mm}
\begin{block}{Example of CFT}
\begin{center}
\begin{tikzpicture}
\draw (0,3-\Vera) -- (5,0-\Vera) -- (0,-3-\Vera) -- cycle;
\node (l11) at (.70,1.5-\Vera) {1};
\node (l21) at (.70,0-\Vera) {0};
\node (l31) at (.70,-1.5-\Vera) {2};

\node (l12) at (2.20,.75-\Vera) {0};
\node (l22) at (2.20,-.75-\Vera) {1};

\node (l13a) at (3.70,0-\Vera) {0};

%\draw (0,0.5) -- (35/12, -5/4);
%\draw (0,-0.5) -- (25/12,-7/4);

%\draw (2.5,3) edge[out=225,in=95,dblue!70,very thick,->] (1.5, -.25);
%\node (jchu) at (3.5,4) {\textcolor{dblue!70}{$j^{\mathrm{th}}$ chute}};

\draw (6.5+\Horb,5) -- (49/6+\Horb,4) -- (6.5+\Horb,3) -- cycle;
\node (111a) at (6.5+.6+\Horb,5-1) {1};
\node (t1) at (6.5+.6+\Horb-\Tone,5-1) {$\sT$};
\draw [thick] (6.5+\Horb-.2,3-.3) to [round left paren] (6.5+\Horb-.2,5+.3);
\draw [thick] (49/6+\Horb, 3-.3) to [round right paren] (49/6+\Horb,5+.3);
\draw (6.5+\Horb+\Horc,5) -- (49/6+\Horb+\Horc,4) -- (6.5+\Horb+\Horc,3) -- cycle;
\node (111b) at (6.5+.6+\Horb+\Horc,5-1) {\textcolor{dblue!70}{1}};
\node (=1) at (6.5+.6+\Horb+\Horc-\Tone,5-1) {$=$};
%\node (=1) [left=of 111b] {$=$};

%\node (try1) [left=of l21] {$\sT$};
%
%    \draw [thick] (-.2,-3.3) to [round left paren] (-.2,3.3);
%        \draw [thick] (4.5,-3.3) to [round right paren] (4.5,3.3);

\draw (6.5+\Horb,2) -- (59/6+\Horb,0) -- (6.5+\Horb,-2) -- cycle;
\node (211a) at (6.5+.6+\Horb,2-1.25) {1};
\node (221a) at (6.5+.6+\Horb,2-1.25-1.5) {0};
\node (212a) at (6.5+.6+\Horb+1.5,2-1.25-.75) {0};
\node(t2) at (6.5+.6+\Horb-\Tone-.2,2-1.25-.75) {$\sT$};
\draw [thick] (6.5+\Horb-.2,-2-.3) to [round left paren] (6.5+\Horb-.2,2+.3);
\draw [thick] (59/6+\Horb-.2, -2-.3) to [round right paren] (59/6+\Horb-.2,2+.3);
\draw (6.5+\Horb+\Hord,2) -- (59/6+\Horb+\Hord,0) -- (6.5+\Horb+\Hord,-2) -- cycle;
\node (211a) at (6.5+.6+\Horb+\Hord,2-1.25) {\textcolor{ngreen}{0}};
\node (221a) at (6.5+.6+\Horb+\Hord,2-1.25-1.5) {\textcolor{dblue!70}{1}};
\node (212a) at (6.5+.6+\Horb+\Hord+1.5,2-1.25-.75) {\textcolor{ngreen}{0}};
\node (=2) at (6.5+.6+\Horb+\Hord-\Tone,2-1.25-.75) {$=$};

\draw (6.5+\Horb,-3) -- (11.5+\Horb,-6) -- (6.5+\Horb,-9) -- cycle;
\node (311a) at (6.5+\Horb+.7,-3-1.5) {1};
\node (321a) at (6.5+\Horb+.7,-3-3) {0};
\node (331a) at (6.5+\Horb+.7,-3-4.5) {2};

\node (312a) at (6.5+\Horb+.7+1.5,-3-1.5-.75) {0};
\node (322a) at (6.5+\Horb+.7+1.5,-3-1.5-.75-1.5) {1};

\node (313a) at (6.5+\Horb+.7+1.5+1.5,-3-1.5-1.5) {0};
\node(t3) at (6.5+\Horb+.7-\Tone-.7,-3-3) {$\sT$};
\draw [thick] (6.5+\Horb-.2,-9-.3) to [round left paren] (6.5+\Horb-.2,-3+.3);
\draw [thick] (11.5+\Horb-.5, -9-.3) to [round right paren] (11.5+\Horb-.5,-3+.3);

\draw (6.5+\Horb+\Hore,-3) -- (11.5+\Horb+\Hore,-6) -- (6.5+\Horb+\Hore,-9) -- cycle;
\node (311a) at (6.5+\Horb+\Hore+.7,-3-1.5) {0};
\node (321a) at (6.5+\Horb+\Hore+.7,-3-3) {\textcolor{ngreen}{0}};
\node (331a) at (6.5+\Horb+\Hore+.7,-3-4.5) {\textcolor{dblue!70}{1}};

\node (312a) at (6.5+\Horb+\Hore+.7+1.5,-3-1.5-.75) {0};
\node (322a) at (6.5+\Horb+\Hore+.7+1.5,-3-1.5-.75-1.5) {\textcolor{ngreen}{0}};

\node (313a) at (6.5+\Horb+\Hore+.7+1.5+1.5,-3-1.5-1.5) {0};
\draw [thick] (6.5+\Horb+\Hore-.2,-9-.3) to [round left paren] (6.5+\Horb+\Hore-.2,-3+.3);
\draw [thick] (11.5+\Horb+\Hore-.5, -9-.3) to [round right paren] (11.5+\Horb+\Hore-.5,-3+.3);
\node (=3a) at (6.5+\Horb+\Hore-\Tone-1.3, -3-3) {$=\tau_2 \tau_1$};

\draw (6.5+\Horb+\Hore+\Horf,-3) -- (11.5+\Horb+\Hore+\Horf,-6) -- (6.5+\Horb+\Hore+\Horf,-9) -- cycle;
\node (311a) at (6.5+\Horb+\Hore+\Horf+.7,-3-1.5) {\textcolor{red}{2}};
\node (321a) at (6.5+\Horb+\Hore+\Horf+.7,-3-3) {\textcolor{ngreen}{0}};
\node (331a) at (6.5+\Horb+\Hore+\Horf+.7,-3-4.5) {\textcolor{dblue!70}{1}};

\node (312a) at (6.5+\Horb+\Hore+\Horf+.7+1.5,-3-1.5-.75) {0};
\node (322a) at (6.5+\Horb+\Hore+\Horf+.7+1.5,-3-1.5-.75-1.5) {\textcolor{red}{1}};

\node (313a) at (6.5+\Horb+\Hore+\Horf+.7+1.5+1.5,-3-1.5-1.5) {0};
\node (=3b) at (6.5+\Horb+\Hore+\Horf+.7-\Tone, -3-3) {$=$};

%\node (l11) at (.70,1.5) {1};
%\node (l21) at (.70,0) {1};
%\node (l31) at (.70,-1.5) {2};
%
%\node (l12) at (2.20,.75) {0};
%\node (l22) at (2.20,-.75) {0};
%
%\node (l13a) at (3.70,0) {0};
%
%\node (try) [left=of l21] {$\sT$};
%
%    \draw [thick] (-.2,-3.3) to [round left paren] (-.2,3.3);
%        \draw [thick] (4.5,-3.3) to [round right paren] (4.5,3.3);
\end{tikzpicture}
\end{center}
\end{block}


\begin{block}{Complete CFT example}
\begin{center}
$\bP(\bw)\ \ \ \ \ \ \ \ \ \ \ \ \ \ \ \ \ \ \ \ \ \ \ \ \ \ \ \ \ \ \ \ \ \bP(\bw^*)$

\begin{tikzpicture}
%%%%%
\draw (0,3) -- (5,0) -- (0,-3) -- cycle;
\node (l11) at (.70,1.5) {1};
\node (l21) at (.70,0) {1};
\node (l31) at (.70,-1.5) {2};

\node (l12) at (2.20,.75) {0};
\node (l22) at (2.20,-.75) {0};

\node (l13a) at (3.70,0) {0};

%%%%%
\draw (\Hor+0,3) -- (\Hor+5,0) -- (\Hor+0,-3) -- cycle;
\node (11) at (\Hor+.70,1.5) {2};
\node (r21a) at (\Hor+.70,0) {1};
\node (31) at (\Hor+.70,-1.5) {1};

\node (12) at (\Hor+2.20,.75) {0};
\node (22) at (\Hor+2.20,-.75) {0};

\node (13) at (\Hor+3.70,0) {0};

%%%%%
\draw (0,3-\Ver) -- (5,0-\Ver) -- (0,-3-\Ver) -- cycle;
\node (11) at (.70,1.5-\Ver) {0};
\node (21) at (.70,0-\Ver) {1};
\node (31) at (.70,-1.5-\Ver) {2};

\node (12) at (2.20,.75-\Ver) {1};
\node (22) at (2.20,-.75-\Ver) {0};

\node (l13b) at (3.70,0-\Ver) {0};

%%%%%
\draw (\Hor+0,3-\Ver) -- (\Hor+5,0-\Ver) -- (\Hor+0,-3-\Ver) -- cycle;
\node (11) at (\Hor+.70,1.5-\Ver) {1};
\node (r21b) at (\Hor+.70,0-\Ver) {1};
\node (31) at (\Hor+.70,-1.5-\Ver) {1};

\node (12) at (\Hor+2.20,.75-\Ver) {1};
\node (22) at (\Hor+2.20,-.75-\Ver) {0};

\node (13) at (\Hor+3.70,0-\Ver) {0};

%%%%%
\draw (0,3-2*\Ver) -- (5,0-2*\Ver) -- (0,-3-2*\Ver) -- cycle;
\node (11) at (.70,1.5-2*\Ver) {1};
\node (21) at (.70,0-2*\Ver) {0};
\node (31) at (.70,-1.5-2*\Ver) {2};

\node (12) at (2.20,.75-2*\Ver) {0};
\node (22) at (2.20,-.75-2*\Ver) {1};

\node (l13c) at (3.70,0-2*\Ver) {0};

%%%%%
\draw (\Hor+0,3-2*\Ver) -- (\Hor+5,0-2*\Ver) -- (\Hor+0,-3-2*\Ver) -- cycle;
\node (11) at (\Hor+.70,1.5-2*\Ver) {2};
\node (r21c) at (\Hor+.70,0-2*\Ver) {0};
\node (31) at (\Hor+.70,-1.5-2*\Ver) {1};

\node (12) at (\Hor+2.20,.75-2*\Ver) {0};
\node (22) at (\Hor+2.20,-.75-2*\Ver) {1};

\node (13) at (\Hor+3.70,0-2*\Ver) {0};

%%%%%
\draw (0,3-3*\Ver) -- (5,0-3*\Ver) -- (0,-3-3*\Ver) -- cycle;
\node (11) at (.70,1.5-3*\Ver) {0};
\node (21) at (.70,0-3*\Ver) {0};
\node (31) at (.70,-1.5-3*\Ver) {2};

\node (12) at (2.20,.75-3*\Ver) {0};
\node (22) at (2.20,-.75-3*\Ver) {1};

\node (l13d) at (3.70,0-3*\Ver) {1};

%%%%%
\draw (\Hor+0,3-3*\Ver) -- (\Hor+5,0-3*\Ver) -- (\Hor+0,-3-3*\Ver) -- cycle;
\node (11) at (\Hor+.70,1.5-3*\Ver) {1};
\node (r21d) at (\Hor+.70,0-3*\Ver) {0};
\node (31) at (\Hor+.70,-1.5-3*\Ver) {1};

\node (12) at (\Hor+2.20,.75-3*\Ver) {0};
\node (22) at (\Hor+2.20,-.75-3*\Ver) {1};

\node (13) at (\Hor+3.70,0-3*\Ver) {1};

%%%%%arrows
\draw[thick, ->] (l13a) -- (r21d);
\draw[thick, ->] (l13b) -- (r21b);
\draw[thick, ->] (l13c) -- (r21c);
\draw[thick, ->] (l13d) -- (r21a);

\end{tikzpicture}
\end{center}

%\begin{align*}
% & \! \! \! \! \! \! \underline{\text{split lattice chain}} & & \! \! \! \! \! \! \underline{\text{split parahoric}}\\
%%& \underline{\text{filtration}} & & \underline{\text{filtration}}\\
%%\rotatebox{90}{$\ldots$} \hspace{2mm} & & \rotatebox{90}{$\ldots$} \hspace{3mm} & & \\
%\rotatebox{90}{$\subsetneq$} & & \rotatebox{90}{$\subsetneq$} \hspace{2mm} & &\\
%L^0 & = \sigma \textrm{-span}\left\{ e_1, e_2, e_3\right\} & \mathfrak{P}^0 & = \left(\begin{array}{ccc} \sigma & \sigma & \sigma \\ \sigma & \sigma & \sigma \\ \sigma & \sigma & \sigma \end{array}\right)\\
%\rotatebox{90}{$\subsetneq$} \hspace{1mm} & & \rotatebox{90}{$\subsetneq$} \hspace{3mm} & &\\
%L^1 & =  \sigma \textrm{-span}\left\{{\color{red} {\bf z}}e_1, {\color{red} {\bf z}}e_2, {\color{red} {\bf z}}e_3\right\} & \mathfrak{P}^1 & =  \left(\begin{array}{ccc} {\color{red} {\bf z}} & {\color{red} {\bf z}} & {\color{red} {\bf z}} \\ {\color{red} {\bf z}} & {\color{red} {\bf z}} & {\color{red} {\bf z}} \\ {\color{red} {\bf z}} & {\color{red} {\bf z}} & {\color{red} {\bf z}} \end{array}\right)\\
%\rotatebox{90}{$\subsetneq$} & & \rotatebox{90}{$\subsetneq$} \hspace{2mm} & &\\
%%\rotatebox{90}{$\ldots$} \hspace{2mm} & &\rotatebox{90}{$\ldots$} \hspace{3mm} & &
%\end{align*}
\end{block}

\vspace{-7mm}
\begin{alertblock}{Main conjecture (proof in progress)}
The bijection $\sT$ determines $\TT$ on simple perverse sheaves; that is, $\TT(\IC(\cO_{\lambda})) = \IC(\cO_{\sT(\lambda)})$.    
% Convolution commutes with $\TT$, so can prove inductively.

%There are two steps for writing a formal type for a connection $\nabla$:
%
%%is ``degree'' okay here?
%\begin{enumerate}
% \item Find a lattice chain and an integer $m$ for which $\nabla \left(L^i\right) \subseteq L^{i-m}$.
% \item Write $\nabla$ with respect to quotients of the corresponding parahorics $\mathfrak{P}^i/\mathfrak{P}^{i+1}$.  The ``degree'' of the leading term is $-m$.
%\end{enumerate}
\end{alertblock}

%\begin{block}{Future work}
%\begin{enumerate}
%\item Develop combinatorial techniques for more quivers.
%\item Applications to cluster algebras and representations of $\cU_q(L\fg)$.
%\end{enumerate}
%
%%{\bf \large A connection with split formal type}
%%
%%$$\nabla = d + \left(\begin{array}{ccc} a_{-1}z^{-1} + a_0 & 0 & 0 \\ 0 & b_{-1}z^{-1}+b_0 & 0 \\ 0 & 0 & c_{-1}z^{-1}+c_0 \end{array}\right)\frac{dz}{z}$$
%%\newline
%%The split lattice chain satisfies $\nabla \left(L^i\right) \subseteq L^{i-1}$.
%%
%%Formal type: $\left(\begin{array}{ccc} a_{-1} & 0 & 0 \\ 0 & b_{-1} & 0 \\ 0 & 0 & c_{-1} \end{array}\right)z^{-1}+\left(\begin{array}{ccc} a_0 & 0 & 0 \\ 0 & b_0 & 0 \\ 0 & 0 & c_0 \end{array}\right)$
%%\vspace{20mm}
%%
%%{\bf \large A connection with Coxeter formal type}
%%
%%$$\nabla = d + \left(\begin{array}{ccc} a_0 & a_{-2}z^{-1} & a_{-1}z^{-1} \\ a_{-1} & a_0 & a_{-2}z^{-1} \\ a_{-2} & a_{-1} & a_0 \end{array}\right)\frac{dz}{z}$$
%%\newline
%%The Coxeter lattice chain satisfies $\nabla \left(L^i\right) \subseteq L^{i-2}$.
%%
%%Formal type: $a_{-2}\left(\begin{array}{ccc} 0 & z^{-1} & 0 \\ 0 & 0 & z^{-1} \\ 1 & 0 & 0 \end{array}\right) + a_{-1} \left(\begin{array}{ccc} 0 & 0 & z^{-1} \\ 1 & 0 & 0 \\ 0 & 1 & 0 \end{array}\right) + a_0 \left(\begin{array}{ccc} 1 & 0 & 0 \\ 0 & 1 & 0 \\ 0 & 0 & 1 \end{array}\right)$
%\end{block}

\end{column} % End of the third column

\end{columns} % End of all the columns in the poster

\end{frame} % End of the enclosing frame

\end{document}
