\documentclass{article}
\usepackage{amsmath}
\usepackage{amssymb}
\usepackage{amsfonts} % Added to support \mathbb
\usepackage[margin=1in]{geometry} % Added to decrease margins

\title{On a Recurrence Relation for Matroid Polynomials}
\author{Your Name}
\date{\today}

\newcommand{\rk}{\text{rk}}
\newcommand{\Qhat}{\hat{Q}}
\newcommand{\bz}{\mathbb{Z}} % Define shorthand for integers (requires amsfonts)
\newcommand{\tauinv}{\tau} % Define shorthand for tau invariant

\begin{document}
\maketitle
\section{Introduction}

Let $M$ be a matroid of rank $r$ on a finite ground set $E$.
Let $P_M(t)$ and $Q(M)$ denote the Kazhdan-Lusztig polynomial and the
inverse Kazhdan-Lusztig polynomial, respectively.

We also define
\[\Qhat_M(t) = (-1)^{\rk(M)} Q_M(t),\]
\[ A(M)(t) = t^{\rk(M)} \Qhat_M(t^{-2}),\]
and
\[B(M)(t) = t^{\rk(M)} P_M(t^{-2}).\]
Let $[F]$ and $\zeta^F$, indexed by
$\mathcal{L}(M)$, be the standard basis and the Kazhdan-Lusztig basis of
the module $\mathcal{H}(M)$, respectively.
We have the following relation between the two bases:
\[
\zeta^F = \sum_{G \leq F} B(M^F_G) \cdot [G],\]
and
\[[F] = \sum_{G \leq F} A(M^F_G) \cdot \zeta^G.\]
where $M^F_G$ is the matroid obtained by restricting $M$ to the flat $F$ and
then contracting the flat $G$.

\section{The Braden-Vysogorets Approach}

The proof technique in the Braden-Vysogorets paper [BV20] centers around an algebraic framework using Hecke algebra modules and a specific homomorphism relating the modules for $M$ and its deletion $M \setminus e$.

Let $\mathcal{H}(M)$ be the Hecke algebra module associated with the lattice of flats $L(M)$, equipped with the standard basis $\{[F] \mid F \in L(M)\}$ and the Kazhdan-Lusztig (KL) basis $\{\zeta^F \mid F \in L(M)\}$.

BV introduce a $\mathbb{Z}[t, t^{-1}]$-linear homomorphism $\Delta: \mathcal{H}(M) \rightarrow \mathcal{H}(M \setminus e)$. This map is defined by its action on the standard basis elements:
$$ \Delta([F]) =
\begin{cases}
    t^{-1}\cdot [F \setminus e] & \text{if } e \in \text{coloops}(M|_F) \\
    [F\setminus e] & \text{otherwise}.
\end{cases}$$
where $F \setminus e$ is interpreted as the corresponding flat in $L(M \setminus e)$.

We then have:
\begin{align*}
    \Delta(\zeta^F) &= \sum_{G \leq F} B(M^F_G) \cdot \Delta[G]\\
    &= \sum_{H \leq F;\ e\notin coloops(H)} B(M^F_G)\cdot [H \setminus e] +
        \sum_{H \leq F;\ e\in coloops(H)} B(M^F_G)\cdot t^{-1}[H\setminus e]\\
        &= \zeta^{F \setminus e} + \sum_{H \leq F;\ e\in coloops(H)} B(M^F_G)\cdot t^{-1}[H\setminus e]\\
\end{align*}

This leads to the expansion (Equation (12) in [BV20]):
$$ \Delta(\zeta^E) = \zeta^{E \setminus e} + \sum_{F' \in S} \tauinv(M_{F' \cup e}) \zeta^{F'} $$
Further steps in the proof involve applying $\Delta$ to another specific element, $Z_M(t)$, related to the identity element of the algebra, and manipulating the resulting expressions using properties of the KL basis and the $P_M(t)$ polynomial to arrive at their main recurrence relation for $P_M(t)$ (Theorem 2.8).

(Note: The recurrence stated in Theorem 2.8 of the BV paper relates $P_M$, $P_{M \setminus e}$, and terms involving $P_{M_{F'\cup e}}$, which differs from the formula involving contraction $P_{M/e}$ that you included in your LaTeX outline. The formula you stated might be from a different source or context).

\end{document}