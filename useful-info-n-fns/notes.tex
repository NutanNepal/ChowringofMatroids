\documentclass[11pt]{article}
\usepackage[margin = 1in]{geometry}
\usepackage{parskip}
\usepackage{amssymb,amsmath,amsfonts,verbatim, amsthm}
\usepackage[breakable, skins]{tcolorbox}
\usepackage{graphicx, tikz-cd, adjustbox}
\usepackage{float}
\usepackage{scalerel}
\usepackage{stackengine,wasysym}
\usepackage{biblatex}
\AtBeginBibliography{\small}
\bibliography{biblio.bib}

\newtcolorbox{mybox}[2][]{
    arc=0mm, enhanced, frame hidden, breakable
}
\newcommand\noverline[1]{%
  \kern\nindent\overline
  {\kern-\nindent#1\kern-\nindent}\kern\nindent}

%\newcommand{\bqed}{$\hfill\blacksquare$}
\newcommand{\bz}{\mathbb{Z}}
\newcommand{\calA}{\mathcal{A}}
\newcommand{\ind}{\text{\normalfont Ind}}
\newcommand{\stab}{\text{\normalfont Stab}}
\newcommand{\hilbM}{\normalfont\text{ \underline{H}}_\text{M}}
\newcommand{\rank}{\normalfont\text{rk}}
\newcommand{\flats}{\normalfont\text{Flats}}
\newcommand{\flags}{\normalfont\text{Flags}}
\newcommand{\matM}{\normalfont\text{M}}

\newcommand{\hilbMmodF}{\normalfont\text{ \underline{H}}_\text{M/F}}

\newtheorem{theorem}{Theorem}
\newtheorem{corollary}[theorem]{Corollary}
\newtheorem{lemma}[theorem]{Lemma}
\newtheorem{proposition}[theorem]{Proposition}

\theoremstyle{remark}
\newtheorem*{remark}{Remark}

%\setlength{\topmargin}{0mm}
%\setlength{\textwidth}{150mm} 
%\setlength{\textheight}{240mm}
%\setlength{\oddsidemargin}{0mm} 
%\setlength{\evensidemargin}{0mm}

\title{Notes (Equivariant Chow Polynomial of a Matroid)}
%\author{Nutan Nepal}
%\date{\today}


\begin{document}
{\textbf{Notes: Equivariant Chow Polynomial of a Matroid}}\hfill {\small{\today}}

\hrulefill
\begin{lemma}
    If $1_H$ denotes the trivial module for $KH$, then $\ind_H^G(1_H)$ affords the permutation representation
    of $G$ on the right (or left) cosets of $H$.
\end{lemma}
\begin{proof}
    \cite{collins-1990}
\end{proof}

\begin{theorem}
    Let $X = \{x_1,\ldots, x_n\}$ be a transitive $G$-set and $G_{x_1} = \stab_G(x_1)$ denote the
    stabilizer of $x_1\in E$ under the action of $G$. Let $Y$ denote the set of left cosets of $G_{x_1}$.
    Then $X\cong Y$ as $G$-sets.
\end{theorem}
\begin{proof}
    By Orbit-Stabilizer theorem, we have $[G:H] = n$.
    Since the action is transitive, there exist $g_1,\ldots,g_n\in G$ with $g_i\cdot x_1 = x_i$ such that
    $g_1H,\ldots,g_nH$ are the distinct left cosets of $H$ in $G$.
    (Rotman's group theory book, section on G-sets.)

    $\mathfrak{to. be. continued.(?)}$
\end{proof}

\begin{corollary}
    Let $X$ be a $G$-set and $V$ be the permutation representation of $G$ on $X$. Let $X/G$ be the set of orbits
    under the action of $G$ and $G_x$ be the stabilizer subgroup of $x\in X$. Then
    $$V = \bigoplus_{\overline{x}\in X/G}{\ind_{G_x}^G(1_{G_x})}.$$
\end{corollary}

\hrulefill

\begin{corollary}
    [\cite{fy-2004}\cite{angarone2024chowringsmatroidspermutation}]
    The Chow ring $A(\matM)$ of a matroid $\matM$
    is free as a $\bz$-module, with $\bz$-basis given by the $FY$-monomials
            $$\text{FY}=\{x_{F_1}^{m_1}x_{F_2}^{m_2}\cdots x_{F_k}^{m_k}:\ (\emptyset = F_0)\subset
            F_1\subset F_2\subset\cdots\subset
            F_k,\ \text{and}\ m_i\leq\rank(F_i)-\rank(F_{i-1})-1 \}.$$
\end{corollary}


\begin{proposition}
    \label{prop:fy-formula}
    Let M be a loopless matroid. The Hilbert-Poincare series of the Chow ring $A(\matM)$ is given by
    $$\hilbM = \sum_{\emptyset = F_0\subset F_1\subset \cdots\subset F_m}{\prod_{i=1}^m{
        \frac{x(1-x^{\rank(F_i)-\rank(F_{i-1})-1})}{1-x}}}.$$
\end{proposition}

\hspace*{5mm}
Given a flag $F = F_0\subset F_1\subset\ldots\subset F_m$ of strictly increasing flats, the formula in
proposition~\ref{prop:fy-formula} essentially computes the number of degree $i$ monomials divisible
precisely by $x_{F_1}x_{F_2}\cdots x_{F_m}$ as coefficients of $x^i$ and sums over all flags to obtain the
Hilbert-Poincare series.

\hrulefill

Consider the polynomial ring $2^{FY}[x]$ with addition and multiplication defined as follows:
\begin{enumerate}
    \item $Sx^i + Tx^i = (S\cup T)x^i$,
    \item $Sx^i\cdot Tx^j = (ST)x^{i+j}$,
\end{enumerate}
where $S$ and $T$ are subsets of $FY$ and $ST=\{s\cdot t\ |\ s\in S,\ t\in T\}$.
A slight modification of the formula in proposition~\ref{prop:fy-formula} is as follows:

\begin{equation}
    \label{eqn:modH}
    A = \sum_{\emptyset = F_0\subset F_1\subset \cdots\subset F_m}{\left(\prod_{i=1}^m{
        \{F_i\}x + \{F_i^2\}x^2+ \cdots \{F_i^{\rank(F_i)-\rank(F_{i-1})-1}\}x^{\rank(F_i)-\rank(F_{i-1})-1}
    }\right)}
\end{equation}

assuming that the factor is 0 when $\rank(F_i)-\rank(F_{i-1}) = 1$. The new formula explicitly computes
the FY-monomials corresponding to the flag $F_0\subset F_1\subset \cdots\subset F_m$ and summing over all flags gives us
$$A = FY^0 + FY^1x + \cdots + FY^{r-1}x^{r-1}$$
for a matroid of rank $r$.

\hrulefill

If a group $G$ acts on the set of atoms, then we have an induced action on the sets $\flats$ and $FY$.
For a fixed flag $P = F_0\subset F_1\subset \cdots\subset F_m$, consider the summand of $P$ in $A$:
    $$A_P = 
    \prod_{i=1}^m{
        \{F_i\}x + \{F_i^2\}x^2+ \cdots \{F_i^{\rank(F_i)-\rank(F_{i-1})-1}\}x^{\rank(F_i)-\rank(F_{i-1})-1}
    } = P^0 + P^1x+\cdots P^{r-1}x^{r-1}.$$

We have the following facts about the action of $G$ on $\flats$ and $FY$:
\begin{enumerate}
    \item For every monomial $m$ in $P^i\subseteq FY^i$, the stabilizer group is $G_P=
        G_{F_0}\cap G_{F_1}\cap \cdots\cap G_{F_m}$.
        Equivalently, the orbits of any two elements in $P^i$ under the action of $G$ are disjoint.
    \item If $Q$ is in the orbit of the flag $P$, then $Q^i = P^i$.
    \item The permutation representation of $G$ on the orbit $\overline{m}\in FY^i/G$ is
        given by $\ind_{G_P}^G(1_{G_P})$ (theorem above).
    \item From above, the permutation representation on orbits of $P^i$ equals $|P^i|\cdot\ind_{G_P}^G(1_{G_P})$.
    \item Hence, the permutation representation on the orbits of $P$ can be decomposed as
        \begin{align*}
            V(A_{\overline{P}}) =& (|P^0|+|P^1|x+\cdots+|P^{r-1}|x^{r-1})\cdot\ind^G(1_{G_P})\\
            =&\left(\prod_{i=1}^m{
                \frac{x(1-x^{\rank(F_i)-\rank(F_{i-1})-1})}{1-x}}\right)\cdot\ind^G(1_{G_P}).
        \end{align*}
\end{enumerate}

From (5), we have the decomposition of the permutation representation on $FY$ as follows:

$$\hilbM^G = \sum_{\overline{P}\in\flags_0/G}\left(\prod_{i=1}^m{
    \frac{x(1-x^{\rank(F_i)-\rank(F_{i-1})-1})}{1-x}}\right)\cdot\ind^G(1_{G_P})$$
    
where $\flags_0$ denotes the set of flags starting at $F_0=\emptyset$.

\hrulefill

Some computation then provides us with the following form:

$$\hilbM = 1_G + \sum_{F\neq \emptyset}{\frac{x(1-x^{\rank(F)-1})}{1-x}\cdot
\frac{|G_F|}{|G|}\cdot \ind^G(\hilbMmodF^{G_F})}$$

\hrulefill
\newpage
\printbibliography

\end{document}