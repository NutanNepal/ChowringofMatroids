\documentclass[12pt]{article}
\usepackage[margin=1in]{geometry}% Change the margins here if you wish.\
\usepackage{parskip}

%\setlength{\parindent}{5pt} % This is the set the indent length for new paragraphs, change if you want.
%\setlength{\parskip}{5pt} % This sets the distance between paragraphs, which will be used anytime you have a blank line in your LaTeX code.

\usepackage{amssymb,amsmath,amsfonts,verbatim}
\usepackage[breakable, skins]{tcolorbox}

\newtcolorbox{mybox}[2][]{
    arc=0mm, enhanced, frame hidden, breakable
}

\newcommand{\qed}{$\hfill\blacksquare$}
\newcommand{\calA}{\mathcal{A}}
\newcommand{\ind}{\text{Ind}}

%\setlength{\topmargin}{0mm}
%\setlength{\textwidth}{190mm} 
%\setlength{\textheight}{240mm}
%\setlength{\oddsidemargin}{0mm} 
%\setlength{\evensidemargin}{0mm}

\title{Notes (Equivariant Chow Polynomial of a Matroid)}
%\author{Nutan Nepal}
%\date{\today}


\begin{document}
{\textbf{Notes: Equivariant Chow Polynomial of a Matroid}}\hfill \small{\today}

\hrulefill

\vspace*{5mm}
\hspace*{5mm} Let $G$ be a group and $E$ be a $G$-set. Let $\calA$ be a set of multisets over $E$.
The action of $G$ on $E$ extends to action on each multiset and
we say that $G$ acts on $\calA$ if $\calA$ is $G$-invariant.

\hspace*{5mm} Let $V$ denote the permutation representation of $G$ on the set $\calA$. For a multiset
$A\in\calA$, let $G_A$ denote the stabilizer subgroup of $A$ in $G$ and $T_{G_A}$ denote the trivial
representation of $G_A$. Then,
$$V = \bigoplus_{\underline{A}\in \calA/G}{\ind_{G_A}^G(T_{G_A})} $$

\hrulefill

\vspace*{5mm}
\hspace*{5mm} Let $G$ be a group and $E$ be a $G$-set.
Let $V$ denote the permutation representation of $G$ on the set $E$. For $x\in E$,
Let $G_x$ denote the stabilizer subgroup of $x$ in $G$ and $T_{G_x}$ denote the trivial representation
of $G_x$. Then,
$$V = \bigoplus_{\underline{x}\in E/G}{\ind_{G_x}^G(T_{G_x})} $$
\end{document}