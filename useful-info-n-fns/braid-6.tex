\documentclass[11pt]{article}

    \usepackage[breakable]{tcolorbox}
    \usepackage{parskip} % Stop auto-indenting (to mimic markdown behaviour)
    

    % Basic figure setup, for now with no caption control since it's done
    % automatically by Pandoc (which extracts ![](path) syntax from Markdown).
    \usepackage{graphicx}
    % Keep aspect ratio if custom image width or height is specified
    \setkeys{Gin}{keepaspectratio}
    % Maintain compatibility with old templates. Remove in nbconvert 6.0
    \let\Oldincludegraphics\includegraphics
    % Ensure that by default, figures have no caption (until we provide a
    % proper Figure object with a Caption API and a way to capture that
    % in the conversion process - todo).
    \usepackage{caption}
    \DeclareCaptionFormat{nocaption}{}
    \captionsetup{format=nocaption,aboveskip=0pt,belowskip=0pt}

    \usepackage{float}
    \floatplacement{figure}{H} % forces figures to be placed at the correct location
    \usepackage{xcolor} % Allow colors to be defined
    \usepackage{enumerate} % Needed for markdown enumerations to work
    \usepackage{geometry} % Used to adjust the document margins
    \usepackage{amsmath} % Equations
    \usepackage{amssymb} % Equations
    \usepackage{textcomp} % defines textquotesingle
    % Hack from http://tex.stackexchange.com/a/47451/13684:
    \AtBeginDocument{%
        \def\PYZsq{\textquotesingle}% Upright quotes in Pygmentized code
    }
    \usepackage{upquote} % Upright quotes for verbatim code
    \usepackage{eurosym} % defines \euro

    \usepackage{iftex}
    \ifPDFTeX
        \usepackage[T1]{fontenc}
        \IfFileExists{alphabeta.sty}{
              \usepackage{alphabeta}
          }{
              \usepackage[mathletters]{ucs}
              \usepackage[utf8x]{inputenc}
          }
    \else
        \usepackage{fontspec}
        \usepackage{unicode-math}
    \fi

    \usepackage{fancyvrb} % verbatim replacement that allows latex
    \usepackage{grffile} % extends the file name processing of package graphics
                         % to support a larger range
    \makeatletter % fix for old versions of grffile with XeLaTeX
    \@ifpackagelater{grffile}{2019/11/01}
    {
      % Do nothing on new versions
    }
    {
      \def\Gread@@xetex#1{%
        \IfFileExists{"\Gin@base".bb}%
        {\Gread@eps{\Gin@base.bb}}%
        {\Gread@@xetex@aux#1}%
      }
    }
    \makeatother
    \usepackage[Export]{adjustbox} % Used to constrain images to a maximum size
    \adjustboxset{max size={0.9\linewidth}{0.9\paperheight}}

    % The hyperref package gives us a pdf with properly built
    % internal navigation ('pdf bookmarks' for the table of contents,
    % internal cross-reference links, web links for URLs, etc.)
    \usepackage{hyperref}
    % The default LaTeX title has an obnoxious amount of whitespace. By default,
    % titling removes some of it. It also provides customization options.
    \usepackage{titling}
    \usepackage{longtable} % longtable support required by pandoc >1.10
    \usepackage{booktabs}  % table support for pandoc > 1.12.2
    \usepackage{array}     % table support for pandoc >= 2.11.3
    \usepackage{calc}      % table minipage width calculation for pandoc >= 2.11.1
    \usepackage[inline]{enumitem} % IRkernel/repr support (it uses the enumerate* environment)
    \usepackage[normalem]{ulem} % ulem is needed to support strikethroughs (\sout)
                                % normalem makes italics be italics, not underlines
    \usepackage{soul}      % strikethrough (\st) support for pandoc >= 3.0.0
    \usepackage{mathrsfs}
    

    
    % Colors for the hyperref package
    \definecolor{urlcolor}{rgb}{0,.145,.698}
    \definecolor{linkcolor}{rgb}{.71,0.21,0.01}
    \definecolor{citecolor}{rgb}{.12,.54,.11}

    % ANSI colors
    \definecolor{ansi-black}{HTML}{3E424D}
    \definecolor{ansi-black-intense}{HTML}{282C36}
    \definecolor{ansi-red}{HTML}{E75C58}
    \definecolor{ansi-red-intense}{HTML}{B22B31}
    \definecolor{ansi-green}{HTML}{00A250}
    \definecolor{ansi-green-intense}{HTML}{007427}
    \definecolor{ansi-yellow}{HTML}{DDB62B}
    \definecolor{ansi-yellow-intense}{HTML}{B27D12}
    \definecolor{ansi-blue}{HTML}{208FFB}
    \definecolor{ansi-blue-intense}{HTML}{0065CA}
    \definecolor{ansi-magenta}{HTML}{D160C4}
    \definecolor{ansi-magenta-intense}{HTML}{A03196}
    \definecolor{ansi-cyan}{HTML}{60C6C8}
    \definecolor{ansi-cyan-intense}{HTML}{258F8F}
    \definecolor{ansi-white}{HTML}{C5C1B4}
    \definecolor{ansi-white-intense}{HTML}{A1A6B2}
    \definecolor{ansi-default-inverse-fg}{HTML}{FFFFFF}
    \definecolor{ansi-default-inverse-bg}{HTML}{000000}

    % common color for the border for error outputs.
    \definecolor{outerrorbackground}{HTML}{FFDFDF}

    % commands and environments needed by pandoc snippets
    % extracted from the output of `pandoc -s`
    \providecommand{\tightlist}{%
      \setlength{\itemsep}{0pt}\setlength{\parskip}{0pt}}
    \DefineVerbatimEnvironment{Highlighting}{Verbatim}{commandchars=\\\{\}}
    % Add ',fontsize=\small' for more characters per line
    \newenvironment{Shaded}{}{}
    \newcommand{\KeywordTok}[1]{\textcolor[rgb]{0.00,0.44,0.13}{\textbf{{#1}}}}
    \newcommand{\DataTypeTok}[1]{\textcolor[rgb]{0.56,0.13,0.00}{{#1}}}
    \newcommand{\DecValTok}[1]{\textcolor[rgb]{0.25,0.63,0.44}{{#1}}}
    \newcommand{\BaseNTok}[1]{\textcolor[rgb]{0.25,0.63,0.44}{{#1}}}
    \newcommand{\FloatTok}[1]{\textcolor[rgb]{0.25,0.63,0.44}{{#1}}}
    \newcommand{\CharTok}[1]{\textcolor[rgb]{0.25,0.44,0.63}{{#1}}}
    \newcommand{\StringTok}[1]{\textcolor[rgb]{0.25,0.44,0.63}{{#1}}}
    \newcommand{\CommentTok}[1]{\textcolor[rgb]{0.38,0.63,0.69}{\textit{{#1}}}}
    \newcommand{\OtherTok}[1]{\textcolor[rgb]{0.00,0.44,0.13}{{#1}}}
    \newcommand{\AlertTok}[1]{\textcolor[rgb]{1.00,0.00,0.00}{\textbf{{#1}}}}
    \newcommand{\FunctionTok}[1]{\textcolor[rgb]{0.02,0.16,0.49}{{#1}}}
    \newcommand{\RegionMarkerTok}[1]{{#1}}
    \newcommand{\ErrorTok}[1]{\textcolor[rgb]{1.00,0.00,0.00}{\textbf{{#1}}}}
    \newcommand{\NormalTok}[1]{{#1}}

    % Additional commands for more recent versions of Pandoc
    \newcommand{\ConstantTok}[1]{\textcolor[rgb]{0.53,0.00,0.00}{{#1}}}
    \newcommand{\SpecialCharTok}[1]{\textcolor[rgb]{0.25,0.44,0.63}{{#1}}}
    \newcommand{\VerbatimStringTok}[1]{\textcolor[rgb]{0.25,0.44,0.63}{{#1}}}
    \newcommand{\SpecialStringTok}[1]{\textcolor[rgb]{0.73,0.40,0.53}{{#1}}}
    \newcommand{\ImportTok}[1]{{#1}}
    \newcommand{\DocumentationTok}[1]{\textcolor[rgb]{0.73,0.13,0.13}{\textit{{#1}}}}
    \newcommand{\AnnotationTok}[1]{\textcolor[rgb]{0.38,0.63,0.69}{\textbf{\textit{{#1}}}}}
    \newcommand{\CommentVarTok}[1]{\textcolor[rgb]{0.38,0.63,0.69}{\textbf{\textit{{#1}}}}}
    \newcommand{\VariableTok}[1]{\textcolor[rgb]{0.10,0.09,0.49}{{#1}}}
    \newcommand{\ControlFlowTok}[1]{\textcolor[rgb]{0.00,0.44,0.13}{\textbf{{#1}}}}
    \newcommand{\OperatorTok}[1]{\textcolor[rgb]{0.40,0.40,0.40}{{#1}}}
    \newcommand{\BuiltInTok}[1]{{#1}}
    \newcommand{\ExtensionTok}[1]{{#1}}
    \newcommand{\PreprocessorTok}[1]{\textcolor[rgb]{0.74,0.48,0.00}{{#1}}}
    \newcommand{\AttributeTok}[1]{\textcolor[rgb]{0.49,0.56,0.16}{{#1}}}
    \newcommand{\InformationTok}[1]{\textcolor[rgb]{0.38,0.63,0.69}{\textbf{\textit{{#1}}}}}
    \newcommand{\WarningTok}[1]{\textcolor[rgb]{0.38,0.63,0.69}{\textbf{\textit{{#1}}}}}


    % Define a nice break command that doesn't care if a line doesn't already
    % exist.
    \def\br{\hspace*{\fill} \\* }
    % Math Jax compatibility definitions
    \def\gt{>}
    \def\lt{<}
    \let\Oldtex\TeX
    \let\Oldlatex\LaTeX
    \renewcommand{\TeX}{\textrm{\Oldtex}}
    \renewcommand{\LaTeX}{\textrm{\Oldlatex}}
    % Document parameters
    % Document title
    \title{braid-6}
    
    
    
    
    
    
    
% Pygments definitions
\makeatletter
\def\PY@reset{\let\PY@it=\relax \let\PY@bf=\relax%
    \let\PY@ul=\relax \let\PY@tc=\relax%
    \let\PY@bc=\relax \let\PY@ff=\relax}
\def\PY@tok#1{\csname PY@tok@#1\endcsname}
\def\PY@toks#1+{\ifx\relax#1\empty\else%
    \PY@tok{#1}\expandafter\PY@toks\fi}
\def\PY@do#1{\PY@bc{\PY@tc{\PY@ul{%
    \PY@it{\PY@bf{\PY@ff{#1}}}}}}}
\def\PY#1#2{\PY@reset\PY@toks#1+\relax+\PY@do{#2}}

\@namedef{PY@tok@w}{\def\PY@tc##1{\textcolor[rgb]{0.73,0.73,0.73}{##1}}}
\@namedef{PY@tok@c}{\let\PY@it=\textit\def\PY@tc##1{\textcolor[rgb]{0.24,0.48,0.48}{##1}}}
\@namedef{PY@tok@cp}{\def\PY@tc##1{\textcolor[rgb]{0.61,0.40,0.00}{##1}}}
\@namedef{PY@tok@k}{\let\PY@bf=\textbf\def\PY@tc##1{\textcolor[rgb]{0.00,0.50,0.00}{##1}}}
\@namedef{PY@tok@kp}{\def\PY@tc##1{\textcolor[rgb]{0.00,0.50,0.00}{##1}}}
\@namedef{PY@tok@kt}{\def\PY@tc##1{\textcolor[rgb]{0.69,0.00,0.25}{##1}}}
\@namedef{PY@tok@o}{\def\PY@tc##1{\textcolor[rgb]{0.40,0.40,0.40}{##1}}}
\@namedef{PY@tok@ow}{\let\PY@bf=\textbf\def\PY@tc##1{\textcolor[rgb]{0.67,0.13,1.00}{##1}}}
\@namedef{PY@tok@nb}{\def\PY@tc##1{\textcolor[rgb]{0.00,0.50,0.00}{##1}}}
\@namedef{PY@tok@nf}{\def\PY@tc##1{\textcolor[rgb]{0.00,0.00,1.00}{##1}}}
\@namedef{PY@tok@nc}{\let\PY@bf=\textbf\def\PY@tc##1{\textcolor[rgb]{0.00,0.00,1.00}{##1}}}
\@namedef{PY@tok@nn}{\let\PY@bf=\textbf\def\PY@tc##1{\textcolor[rgb]{0.00,0.00,1.00}{##1}}}
\@namedef{PY@tok@ne}{\let\PY@bf=\textbf\def\PY@tc##1{\textcolor[rgb]{0.80,0.25,0.22}{##1}}}
\@namedef{PY@tok@nv}{\def\PY@tc##1{\textcolor[rgb]{0.10,0.09,0.49}{##1}}}
\@namedef{PY@tok@no}{\def\PY@tc##1{\textcolor[rgb]{0.53,0.00,0.00}{##1}}}
\@namedef{PY@tok@nl}{\def\PY@tc##1{\textcolor[rgb]{0.46,0.46,0.00}{##1}}}
\@namedef{PY@tok@ni}{\let\PY@bf=\textbf\def\PY@tc##1{\textcolor[rgb]{0.44,0.44,0.44}{##1}}}
\@namedef{PY@tok@na}{\def\PY@tc##1{\textcolor[rgb]{0.41,0.47,0.13}{##1}}}
\@namedef{PY@tok@nt}{\let\PY@bf=\textbf\def\PY@tc##1{\textcolor[rgb]{0.00,0.50,0.00}{##1}}}
\@namedef{PY@tok@nd}{\def\PY@tc##1{\textcolor[rgb]{0.67,0.13,1.00}{##1}}}
\@namedef{PY@tok@s}{\def\PY@tc##1{\textcolor[rgb]{0.73,0.13,0.13}{##1}}}
\@namedef{PY@tok@sd}{\let\PY@it=\textit\def\PY@tc##1{\textcolor[rgb]{0.73,0.13,0.13}{##1}}}
\@namedef{PY@tok@si}{\let\PY@bf=\textbf\def\PY@tc##1{\textcolor[rgb]{0.64,0.35,0.47}{##1}}}
\@namedef{PY@tok@se}{\let\PY@bf=\textbf\def\PY@tc##1{\textcolor[rgb]{0.67,0.36,0.12}{##1}}}
\@namedef{PY@tok@sr}{\def\PY@tc##1{\textcolor[rgb]{0.64,0.35,0.47}{##1}}}
\@namedef{PY@tok@ss}{\def\PY@tc##1{\textcolor[rgb]{0.10,0.09,0.49}{##1}}}
\@namedef{PY@tok@sx}{\def\PY@tc##1{\textcolor[rgb]{0.00,0.50,0.00}{##1}}}
\@namedef{PY@tok@m}{\def\PY@tc##1{\textcolor[rgb]{0.40,0.40,0.40}{##1}}}
\@namedef{PY@tok@gh}{\let\PY@bf=\textbf\def\PY@tc##1{\textcolor[rgb]{0.00,0.00,0.50}{##1}}}
\@namedef{PY@tok@gu}{\let\PY@bf=\textbf\def\PY@tc##1{\textcolor[rgb]{0.50,0.00,0.50}{##1}}}
\@namedef{PY@tok@gd}{\def\PY@tc##1{\textcolor[rgb]{0.63,0.00,0.00}{##1}}}
\@namedef{PY@tok@gi}{\def\PY@tc##1{\textcolor[rgb]{0.00,0.52,0.00}{##1}}}
\@namedef{PY@tok@gr}{\def\PY@tc##1{\textcolor[rgb]{0.89,0.00,0.00}{##1}}}
\@namedef{PY@tok@ge}{\let\PY@it=\textit}
\@namedef{PY@tok@gs}{\let\PY@bf=\textbf}
\@namedef{PY@tok@ges}{\let\PY@bf=\textbf\let\PY@it=\textit}
\@namedef{PY@tok@gp}{\let\PY@bf=\textbf\def\PY@tc##1{\textcolor[rgb]{0.00,0.00,0.50}{##1}}}
\@namedef{PY@tok@go}{\def\PY@tc##1{\textcolor[rgb]{0.44,0.44,0.44}{##1}}}
\@namedef{PY@tok@gt}{\def\PY@tc##1{\textcolor[rgb]{0.00,0.27,0.87}{##1}}}
\@namedef{PY@tok@err}{\def\PY@bc##1{{\setlength{\fboxsep}{\string -\fboxrule}\fcolorbox[rgb]{1.00,0.00,0.00}{1,1,1}{\strut ##1}}}}
\@namedef{PY@tok@kc}{\let\PY@bf=\textbf\def\PY@tc##1{\textcolor[rgb]{0.00,0.50,0.00}{##1}}}
\@namedef{PY@tok@kd}{\let\PY@bf=\textbf\def\PY@tc##1{\textcolor[rgb]{0.00,0.50,0.00}{##1}}}
\@namedef{PY@tok@kn}{\let\PY@bf=\textbf\def\PY@tc##1{\textcolor[rgb]{0.00,0.50,0.00}{##1}}}
\@namedef{PY@tok@kr}{\let\PY@bf=\textbf\def\PY@tc##1{\textcolor[rgb]{0.00,0.50,0.00}{##1}}}
\@namedef{PY@tok@bp}{\def\PY@tc##1{\textcolor[rgb]{0.00,0.50,0.00}{##1}}}
\@namedef{PY@tok@fm}{\def\PY@tc##1{\textcolor[rgb]{0.00,0.00,1.00}{##1}}}
\@namedef{PY@tok@vc}{\def\PY@tc##1{\textcolor[rgb]{0.10,0.09,0.49}{##1}}}
\@namedef{PY@tok@vg}{\def\PY@tc##1{\textcolor[rgb]{0.10,0.09,0.49}{##1}}}
\@namedef{PY@tok@vi}{\def\PY@tc##1{\textcolor[rgb]{0.10,0.09,0.49}{##1}}}
\@namedef{PY@tok@vm}{\def\PY@tc##1{\textcolor[rgb]{0.10,0.09,0.49}{##1}}}
\@namedef{PY@tok@sa}{\def\PY@tc##1{\textcolor[rgb]{0.73,0.13,0.13}{##1}}}
\@namedef{PY@tok@sb}{\def\PY@tc##1{\textcolor[rgb]{0.73,0.13,0.13}{##1}}}
\@namedef{PY@tok@sc}{\def\PY@tc##1{\textcolor[rgb]{0.73,0.13,0.13}{##1}}}
\@namedef{PY@tok@dl}{\def\PY@tc##1{\textcolor[rgb]{0.73,0.13,0.13}{##1}}}
\@namedef{PY@tok@s2}{\def\PY@tc##1{\textcolor[rgb]{0.73,0.13,0.13}{##1}}}
\@namedef{PY@tok@sh}{\def\PY@tc##1{\textcolor[rgb]{0.73,0.13,0.13}{##1}}}
\@namedef{PY@tok@s1}{\def\PY@tc##1{\textcolor[rgb]{0.73,0.13,0.13}{##1}}}
\@namedef{PY@tok@mb}{\def\PY@tc##1{\textcolor[rgb]{0.40,0.40,0.40}{##1}}}
\@namedef{PY@tok@mf}{\def\PY@tc##1{\textcolor[rgb]{0.40,0.40,0.40}{##1}}}
\@namedef{PY@tok@mh}{\def\PY@tc##1{\textcolor[rgb]{0.40,0.40,0.40}{##1}}}
\@namedef{PY@tok@mi}{\def\PY@tc##1{\textcolor[rgb]{0.40,0.40,0.40}{##1}}}
\@namedef{PY@tok@il}{\def\PY@tc##1{\textcolor[rgb]{0.40,0.40,0.40}{##1}}}
\@namedef{PY@tok@mo}{\def\PY@tc##1{\textcolor[rgb]{0.40,0.40,0.40}{##1}}}
\@namedef{PY@tok@ch}{\let\PY@it=\textit\def\PY@tc##1{\textcolor[rgb]{0.24,0.48,0.48}{##1}}}
\@namedef{PY@tok@cm}{\let\PY@it=\textit\def\PY@tc##1{\textcolor[rgb]{0.24,0.48,0.48}{##1}}}
\@namedef{PY@tok@cpf}{\let\PY@it=\textit\def\PY@tc##1{\textcolor[rgb]{0.24,0.48,0.48}{##1}}}
\@namedef{PY@tok@c1}{\let\PY@it=\textit\def\PY@tc##1{\textcolor[rgb]{0.24,0.48,0.48}{##1}}}
\@namedef{PY@tok@cs}{\let\PY@it=\textit\def\PY@tc##1{\textcolor[rgb]{0.24,0.48,0.48}{##1}}}

\def\PYZbs{\char`\\}
\def\PYZus{\char`\_}
\def\PYZob{\char`\{}
\def\PYZcb{\char`\}}
\def\PYZca{\char`\^}
\def\PYZam{\char`\&}
\def\PYZlt{\char`\<}
\def\PYZgt{\char`\>}
\def\PYZsh{\char`\#}
\def\PYZpc{\char`\%}
\def\PYZdl{\char`\$}
\def\PYZhy{\char`\-}
\def\PYZsq{\char`\'}
\def\PYZdq{\char`\"}
\def\PYZti{\char`\~}
% for compatibility with earlier versions
\def\PYZat{@}
\def\PYZlb{[}
\def\PYZrb{]}
\makeatother


    % For linebreaks inside Verbatim environment from package fancyvrb.
    \makeatletter
        \newbox\Wrappedcontinuationbox
        \newbox\Wrappedvisiblespacebox
        \newcommand*\Wrappedvisiblespace {\textcolor{red}{\textvisiblespace}}
        \newcommand*\Wrappedcontinuationsymbol {\textcolor{red}{\llap{\tiny$\m@th\hookrightarrow$}}}
        \newcommand*\Wrappedcontinuationindent {3ex }
        \newcommand*\Wrappedafterbreak {\kern\Wrappedcontinuationindent\copy\Wrappedcontinuationbox}
        % Take advantage of the already applied Pygments mark-up to insert
        % potential linebreaks for TeX processing.
        %        {, <, #, %, $, ' and ": go to next line.
        %        _, }, ^, &, >, - and ~: stay at end of broken line.
        % Use of \textquotesingle for straight quote.
        \newcommand*\Wrappedbreaksatspecials {%
            \def\PYGZus{\discretionary{\char`\_}{\Wrappedafterbreak}{\char`\_}}%
            \def\PYGZob{\discretionary{}{\Wrappedafterbreak\char`\{}{\char`\{}}%
            \def\PYGZcb{\discretionary{\char`\}}{\Wrappedafterbreak}{\char`\}}}%
            \def\PYGZca{\discretionary{\char`\^}{\Wrappedafterbreak}{\char`\^}}%
            \def\PYGZam{\discretionary{\char`\&}{\Wrappedafterbreak}{\char`\&}}%
            \def\PYGZlt{\discretionary{}{\Wrappedafterbreak\char`\<}{\char`\<}}%
            \def\PYGZgt{\discretionary{\char`\>}{\Wrappedafterbreak}{\char`\>}}%
            \def\PYGZsh{\discretionary{}{\Wrappedafterbreak\char`\#}{\char`\#}}%
            \def\PYGZpc{\discretionary{}{\Wrappedafterbreak\char`\%}{\char`\%}}%
            \def\PYGZdl{\discretionary{}{\Wrappedafterbreak\char`\$}{\char`\$}}%
            \def\PYGZhy{\discretionary{\char`\-}{\Wrappedafterbreak}{\char`\-}}%
            \def\PYGZsq{\discretionary{}{\Wrappedafterbreak\textquotesingle}{\textquotesingle}}%
            \def\PYGZdq{\discretionary{}{\Wrappedafterbreak\char`\"}{\char`\"}}%
            \def\PYGZti{\discretionary{\char`\~}{\Wrappedafterbreak}{\char`\~}}%
        }
        % Some characters . , ; ? ! / are not pygmentized.
        % This macro makes them "active" and they will insert potential linebreaks
        \newcommand*\Wrappedbreaksatpunct {%
            \lccode`\~`\.\lowercase{\def~}{\discretionary{\hbox{\char`\.}}{\Wrappedafterbreak}{\hbox{\char`\.}}}%
            \lccode`\~`\,\lowercase{\def~}{\discretionary{\hbox{\char`\,}}{\Wrappedafterbreak}{\hbox{\char`\,}}}%
            \lccode`\~`\;\lowercase{\def~}{\discretionary{\hbox{\char`\;}}{\Wrappedafterbreak}{\hbox{\char`\;}}}%
            \lccode`\~`\:\lowercase{\def~}{\discretionary{\hbox{\char`\:}}{\Wrappedafterbreak}{\hbox{\char`\:}}}%
            \lccode`\~`\?\lowercase{\def~}{\discretionary{\hbox{\char`\?}}{\Wrappedafterbreak}{\hbox{\char`\?}}}%
            \lccode`\~`\!\lowercase{\def~}{\discretionary{\hbox{\char`\!}}{\Wrappedafterbreak}{\hbox{\char`\!}}}%
            \lccode`\~`\/\lowercase{\def~}{\discretionary{\hbox{\char`\/}}{\Wrappedafterbreak}{\hbox{\char`\/}}}%
            \catcode`\.\active
            \catcode`\,\active
            \catcode`\;\active
            \catcode`\:\active
            \catcode`\?\active
            \catcode`\!\active
            \catcode`\/\active
            \lccode`\~`\~
        }
    \makeatother

    \let\OriginalVerbatim=\Verbatim
    \makeatletter
    \renewcommand{\Verbatim}[1][1]{%
        %\parskip\z@skip
        \sbox\Wrappedcontinuationbox {\Wrappedcontinuationsymbol}%
        \sbox\Wrappedvisiblespacebox {\FV@SetupFont\Wrappedvisiblespace}%
        \def\FancyVerbFormatLine ##1{\hsize\linewidth
            \vtop{\raggedright\hyphenpenalty\z@\exhyphenpenalty\z@
                \doublehyphendemerits\z@\finalhyphendemerits\z@
                \strut ##1\strut}%
        }%
        % If the linebreak is at a space, the latter will be displayed as visible
        % space at end of first line, and a continuation symbol starts next line.
        % Stretch/shrink are however usually zero for typewriter font.
        \def\FV@Space {%
            \nobreak\hskip\z@ plus\fontdimen3\font minus\fontdimen4\font
            \discretionary{\copy\Wrappedvisiblespacebox}{\Wrappedafterbreak}
            {\kern\fontdimen2\font}%
        }%

        % Allow breaks at special characters using \PYG... macros.
        \Wrappedbreaksatspecials
        % Breaks at punctuation characters . , ; ? ! and / need catcode=\active
        \OriginalVerbatim[#1,codes*=\Wrappedbreaksatpunct]%
    }
    \makeatother

    % Exact colors from NB
    \definecolor{incolor}{HTML}{303F9F}
    \definecolor{outcolor}{HTML}{D84315}
    \definecolor{cellborder}{HTML}{CFCFCF}
    \definecolor{cellbackground}{HTML}{F7F7F7}

    % prompt
    \makeatletter
    \newcommand{\boxspacing}{\kern\kvtcb@left@rule\kern\kvtcb@boxsep}
    \makeatother
    \newcommand{\prompt}[4]{
        {\ttfamily\llap{{\color{#2}[#3]:\hspace{3pt}#4}}\vspace{-\baselineskip}}
    }
    

    
    % Prevent overflowing lines due to hard-to-break entities
    \sloppy
    % Setup hyperref package
    \hypersetup{
      breaklinks=true,  % so long urls are correctly broken across lines
      colorlinks=true,
      urlcolor=urlcolor,
      linkcolor=linkcolor,
      citecolor=citecolor,
      }
    % Slightly bigger margins than the latex defaults
    
    \geometry{verbose,tmargin=1in,bmargin=1in,lmargin=1in,rmargin=1in}
    
    

\begin{document}
    
    \section*{Chow Polynomial of Braid-6}\label{chow-polynomial-of-braid-6}

    First, we initialize the matroid Braid-6 and then obtain the lattice of
flats. We relabel the flats for simple presentation.

    \begin{tcolorbox}[breakable, size=fbox, boxrule=1pt, pad at break*=1mm,colback=cellbackground, colframe=cellborder]
\prompt{In}{incolor}{15}{\boxspacing}
\begin{Verbatim}[commandchars=\\\{\}]
\PY{n}{n} \PY{o}{=} \PY{l+m+mi}{6}
\PY{n}{edgelist} \PY{o}{=} \PY{n+nb}{sorted}\PY{p}{(}\PY{n}{graphs}\PY{o}{.}\PY{n}{CompleteGraph}\PY{p}{(}\PY{n}{n}\PY{p}{)}\PY{o}{.}\PY{n}{edges}\PY{p}{(}\PY{n}{labels}\PY{o}{=}\PY{k+kc}{False}\PY{p}{)}\PY{p}{)}
\PY{n}{matroid} \PY{o}{=} \PY{n}{Matroid}\PY{p}{(}\PY{n}{graph}\PY{o}{=}\PY{n}{edgelist}\PY{p}{,} \PY{n}{groundset}\PY{o}{=}\PY{n}{edgelist}\PY{p}{)}
\PY{n}{flats} \PY{o}{=} \PY{p}{[}\PY{n+nb}{list}\PY{p}{(}\PY{n}{matroid}\PY{o}{.}\PY{n}{flats}\PY{p}{(}\PY{n}{i}\PY{p}{)}\PY{p}{)} \PY{k}{for} \PY{n}{i} \PY{o+ow}{in} \PY{n+nb}{range}\PY{p}{(}\PY{n}{n}\PY{p}{)}\PY{p}{]}
\PY{n}{all\PYZus{}elements} \PY{o}{=} \PY{n+nb}{sum}\PY{p}{(}\PY{n}{flats}\PY{p}{,} \PY{p}{[}\PY{p}{]}\PY{p}{)}  \PY{c+c1}{\PYZsh{} Flatten the list of flats}
\PY{n}{labels} \PY{o}{=} \PY{p}{\PYZob{}}\PY{n}{element}\PY{p}{:} \PY{n}{idx} \PY{k}{for} \PY{n}{idx}\PY{p}{,} \PY{n}{element} \PY{o+ow}{in} \PY{n+nb}{enumerate}\PY{p}{(}\PY{n}{all\PYZus{}elements}\PY{p}{)}\PY{p}{\PYZcb{}}
\PY{n}{matroid}\PY{o}{.}\PY{n}{lattice\PYZus{}of\PYZus{}flats}\PY{p}{(}\PY{p}{)}\PY{o}{.}\PY{n}{plot}\PY{p}{(}
    \PY{n}{element\PYZus{}labels} \PY{o}{=} \PY{n}{labels}\PY{p}{,} \PY{n}{element\PYZus{}color} \PY{o}{=} \PY{l+s+s2}{\PYZdq{}}\PY{l+s+s2}{white}\PY{l+s+s2}{\PYZdq{}}\PY{p}{,}
    \PY{n}{figsize}\PY{o}{=} \PY{p}{(}\PY{l+m+mi}{20}\PY{p}{,}\PY{l+m+mi}{20}\PY{p}{)} \PY{p}{,}\PY{n}{cover\PYZus{}color} \PY{o}{=} \PY{l+s+s2}{\PYZdq{}}\PY{l+s+s2}{grey}\PY{l+s+s2}{\PYZdq{}}\PY{p}{)}
\end{Verbatim}
\end{tcolorbox}
 
            
\prompt{Out}{outcolor}{15}{}
    
    \begin{center}
    \adjustimage{max size={0.9\linewidth}{0.9\paperheight}}{braid-6_files/braid-6_2_0.png}
    \end{center}
    { \hspace*{\fill} \\}
    

    We now generate the possible degrees that a flat can have in a monomial.
For example, if \([x_1, x_2, x_3,x_4]\) is a chain and \([0,1,0,2]\) is
a weight, then we say that \(x_2x_4^2\) is an fy-monomial.

    \begin{tcolorbox}[breakable, size=fbox, boxrule=1pt, pad at break*=1mm,colback=cellbackground, colframe=cellborder]
\prompt{In}{incolor}{16}{\boxspacing}
\begin{Verbatim}[commandchars=\\\{\}]
\PY{k}{def} \PY{n+nf}{generate\PYZus{}weights}\PY{p}{(}\PY{n}{rank}\PY{p}{)}\PY{p}{:}
    \PY{n}{weights} \PY{o}{=} \PY{n+nb}{set}\PY{p}{(}\PY{p}{)} 
    \PY{k}{for} \PY{n}{i} \PY{o+ow}{in} \PY{n+nb}{range}\PY{p}{(}\PY{l+m+mi}{1}\PY{p}{,} \PY{n}{rank}\PY{p}{)}\PY{p}{:}
        \PY{k}{for} \PY{n}{j} \PY{o+ow}{in} \PY{n+nb}{range}\PY{p}{(}\PY{n}{rank}\PY{p}{)}\PY{p}{:}
            \PY{n}{weight} \PY{o}{=} \PY{p}{[}\PY{l+m+mi}{0}\PY{p}{]} \PY{o}{*} \PY{n}{rank}
            
            \PY{k}{if} \PY{n}{i} \PY{o}{\PYZgt{}}\PY{o}{=} \PY{n}{j}\PY{p}{:}
                \PY{n}{weight}\PY{p}{[}\PY{n}{i}\PY{p}{]} \PY{o}{=} \PY{n}{j}
                \PY{n}{weights}\PY{o}{.}\PY{n}{add}\PY{p}{(}\PY{n+nb}{tuple}\PY{p}{(}\PY{n}{weight}\PY{p}{)}\PY{p}{)}

            \PY{k}{if} \PY{n}{rank} \PY{o}{\PYZhy{}} \PY{p}{(}\PY{n}{i}\PY{o}{+}\PY{l+m+mi}{1}\PY{p}{)} \PY{o}{\PYZgt{}} \PY{l+m+mi}{1}\PY{p}{:}
                \PY{c+c1}{\PYZsh{} Recursion to get the complete list of weights...}
                \PY{n}{y} \PY{o}{=} \PY{n}{generate\PYZus{}weights}\PY{p}{(}\PY{n}{rank} \PY{o}{\PYZhy{}} \PY{p}{(}\PY{n}{i} \PY{o}{+} \PY{l+m+mi}{1}\PY{p}{)}\PY{p}{)} 
                \PY{k}{for} \PY{n}{x} \PY{o+ow}{in} \PY{n}{y}\PY{p}{:}
                    \PY{n}{temp\PYZus{}weight} \PY{o}{=} \PY{n}{weight}\PY{o}{.}\PY{n}{copy}\PY{p}{(}\PY{p}{)}
                    \PY{n}{weights}\PY{o}{.}\PY{n}{add}\PY{p}{(}\PY{n+nb}{tuple}\PY{p}{(}\PY{n}{temp\PYZus{}weight}\PY{p}{[}\PY{p}{:}\PY{n}{i}\PY{o}{+}\PY{l+m+mi}{1}\PY{p}{]} \PY{o}{+} \PY{n}{x}\PY{p}{)}\PY{p}{)} 

    \PY{k}{return} \PY{p}{[}\PY{n+nb}{list}\PY{p}{(}\PY{n}{w}\PY{p}{)} \PY{k}{for} \PY{n}{w} \PY{o+ow}{in} \PY{n}{weights}\PY{p}{]}

\PY{n}{rank} \PY{o}{=} \PY{n}{matroid}\PY{o}{.}\PY{n}{rank}\PY{p}{(}\PY{p}{)}
\PY{n}{weights} \PY{o}{=} \PY{n}{generate\PYZus{}weights}\PY{p}{(}\PY{n}{rank}\PY{p}{)}
\PY{n}{weights}
\end{Verbatim}
\end{tcolorbox}

            \begin{tcolorbox}[breakable, size=fbox, boxrule=.5pt, pad at break*=1mm, opacityfill=0]
\prompt{Out}{outcolor}{16}{\boxspacing}
\begin{Verbatim}[commandchars=\\\{\}]
[[0, 0, 1, 0, 1],
 [0, 0, 0, 0, 2],
 [0, 1, 0, 0, 1],
 [0, 0, 1, 0, 0],
 [0, 0, 2, 0, 1],
 [0, 0, 0, 0, 1],
 [0, 0, 0, 3, 0],
 [0, 1, 0, 0, 0],
 [0, 0, 2, 0, 0],
 [0, 0, 0, 0, 4],
 [0, 1, 0, 1, 0],
 [0, 0, 0, 0, 0],
 [0, 0, 0, 1, 0],
 [0, 0, 0, 0, 3],
 [0, 0, 0, 2, 0],
 [0, 1, 0, 0, 2]]
\end{Verbatim}
\end{tcolorbox}
        
    Next, we compute all the possible fy-monomials in the example matroid.

    \begin{tcolorbox}[breakable, size=fbox, boxrule=1pt, pad at break*=1mm,colback=cellbackground, colframe=cellborder]
\prompt{In}{incolor}{17}{\boxspacing}
\begin{Verbatim}[commandchars=\\\{\}]
\PY{n}{rflats} \PY{o}{=} \PY{n}{flats}\PY{p}{[}\PY{l+m+mi}{1}\PY{p}{:}\PY{p}{]} \PY{c+c1}{\PYZsh{}empty flat is not required.}
\PY{n}{fy\PYZus{}monomials\PYZus{}list} \PY{o}{=} \PY{p}{[}\PY{p}{[}\PY{p}{]} \PY{k}{for} \PY{n}{\PYZus{}} \PY{o+ow}{in} \PY{n+nb}{range}\PY{p}{(}\PY{n}{rank}\PY{p}{)}\PY{p}{]}

\PY{k}{def} \PY{n+nf}{generate\PYZus{}monomials}\PY{p}{(}\PY{n}{weight}\PY{p}{,} \PY{n}{flats}\PY{p}{)}\PY{p}{:}
    \PY{k}{if} \PY{n+nb}{sum}\PY{p}{(}\PY{n}{weight}\PY{p}{)} \PY{o}{==} \PY{l+m+mi}{0}\PY{p}{:} \PY{k}{return} \PY{p}{[}\PY{p}{[}\PY{p}{]}\PY{p}{]}
    \PY{c+c1}{\PYZsh{} Find the first non\PYZhy{}zero weight}
    \PY{n}{start\PYZus{}index} \PY{o}{=} \PY{n+nb}{next}\PY{p}{(}\PY{n}{i} \PY{k}{for} \PY{n}{i} \PY{o+ow}{in} \PY{n+nb}{range}\PY{p}{(}\PY{n+nb}{len}\PY{p}{(}\PY{n}{weight}\PY{p}{)}\PY{p}{)} \PY{k}{if} \PY{n}{weight}\PY{p}{[}\PY{n}{i}\PY{p}{]} \PY{o}{!=} \PY{l+m+mi}{0}\PY{p}{)}
    \PY{n}{result} \PY{o}{=} \PY{p}{[}\PY{p}{]}

    \PY{k}{for} \PY{n}{initial\PYZus{}flat} \PY{o+ow}{in} \PY{n}{flats}\PY{p}{[}\PY{n}{start\PYZus{}index}\PY{p}{]}\PY{p}{:}
        \PY{n}{initial\PYZus{}monomial} \PY{o}{=} \PY{p}{[}\PY{n}{initial\PYZus{}flat}\PY{p}{]} \PY{o}{*} \PY{n}{weight}\PY{p}{[}\PY{n}{start\PYZus{}index}\PY{p}{]}
        \PY{n}{potential\PYZus{}combinations} \PY{o}{=} \PY{p}{[}\PY{n}{initial\PYZus{}monomial}\PY{p}{]}

        \PY{k}{for} \PY{n}{i} \PY{o+ow}{in} \PY{n+nb}{range}\PY{p}{(}\PY{n}{start\PYZus{}index} \PY{o}{+} \PY{l+m+mi}{1}\PY{p}{,} \PY{n+nb}{len}\PY{p}{(}\PY{n}{weight}\PY{p}{)}\PY{p}{)}\PY{p}{:}
            \PY{k}{if} \PY{n}{weight}\PY{p}{[}\PY{n}{i}\PY{p}{]} \PY{o}{!=} \PY{l+m+mi}{0}\PY{p}{:}
                \PY{n}{new\PYZus{}combinations} \PY{o}{=} \PY{p}{[}\PY{p}{]}
                \PY{k}{for} \PY{n}{flat} \PY{o+ow}{in} \PY{n}{flats}\PY{p}{[}\PY{n}{i}\PY{p}{]}\PY{p}{:}
                    \PY{k}{for} \PY{n}{combo} \PY{o+ow}{in} \PY{n}{potential\PYZus{}combinations}\PY{p}{:}
                        \PY{k}{if} \PY{n}{combo}\PY{p}{[}\PY{o}{\PYZhy{}}\PY{l+m+mi}{1}\PY{p}{]}\PY{o}{.}\PY{n}{issubset}\PY{p}{(}\PY{n}{flat}\PY{p}{)}\PY{p}{:}
                            \PY{n}{new\PYZus{}combo} \PY{o}{=} \PY{n}{combo} \PY{o}{+} \PY{p}{[}\PY{n}{flat}\PY{p}{]} \PY{o}{*} \PY{n}{weight}\PY{p}{[}\PY{n}{i}\PY{p}{]}
                            \PY{n}{new\PYZus{}combinations}\PY{o}{.}\PY{n}{append}\PY{p}{(}\PY{n}{new\PYZus{}combo}\PY{p}{)}
                \PY{n}{potential\PYZus{}combinations} \PY{o}{=} \PY{n}{new\PYZus{}combinations}

        \PY{n}{result}\PY{o}{.}\PY{n}{extend}\PY{p}{(}\PY{n}{potential\PYZus{}combinations}\PY{p}{)}

    \PY{k}{return} \PY{n}{result}

\PY{k}{for} \PY{n}{weight} \PY{o+ow}{in} \PY{n}{weights}\PY{p}{:}
    \PY{n}{fy\PYZus{}monomials\PYZus{}list}\PY{p}{[}\PY{n+nb}{sum}\PY{p}{(}\PY{n}{weight}\PY{p}{)}\PY{p}{]}\PY{o}{.}\PY{n}{extend}\PY{p}{(}\PY{n}{generate\PYZus{}monomials}\PY{p}{(}\PY{n}{weight}\PY{p}{,} \PY{n}{rflats}\PY{p}{)}\PY{p}{)}

\PY{c+c1}{\PYZsh{}example fy\PYZhy{}monomial}
\PY{n+nb}{print}\PY{p}{(}\PY{n}{fy\PYZus{}monomials\PYZus{}list}\PY{p}{[}\PY{l+m+mi}{2}\PY{p}{]}\PY{p}{[}\PY{l+m+mi}{0}\PY{p}{]}\PY{p}{)}
\end{Verbatim}
\end{tcolorbox}

    \begin{Verbatim}[commandchars=\\\{\}]
[frozenset(\{(0, 1), (2, 3), (0, 2), (1, 2), (0, 3), (1, 3)\}), frozenset(\{(0, 1),
(2, 4), (1, 2), (0, 4), (3, 4), (1, 5), (0, 3), (1, 4), (2, 3), (0, 2), (4, 5),
(0, 5), (2, 5), (1, 3), (3, 5)\})]
    \end{Verbatim}

    We write some simplification functions to make fy-monomials look
simpler. Then we write out all fy-monomials of braid-5 in this simple
form.

    \begin{tcolorbox}[breakable, size=fbox, boxrule=1pt, pad at break*=1mm,colback=cellbackground, colframe=cellborder]
\prompt{In}{incolor}{18}{\boxspacing}
\begin{Verbatim}[commandchars=\\\{\}]
\PY{k}{def} \PY{n+nf}{simplify}\PY{p}{(}\PY{n}{monomial}\PY{p}{)}\PY{p}{:}
    \PY{k}{return} \PY{n+nb}{tuple}\PY{p}{(}\PY{n+nb}{sorted}\PY{p}{(}\PY{p}{[}\PY{n}{labels}\PY{p}{[}\PY{n}{x}\PY{p}{]} \PY{k}{for} \PY{n}{x} \PY{o+ow}{in} \PY{n}{monomial}\PY{p}{]}\PY{p}{)}\PY{p}{)}

\PY{k}{def} \PY{n+nf}{set\PYZus{}simplify}\PY{p}{(}\PY{n}{monomial\PYZus{}set}\PY{p}{)}\PY{p}{:}
    \PY{k}{return} \PY{n+nb}{set}\PY{p}{(}\PY{n+nb}{sorted}\PY{p}{(}\PY{p}{[}\PY{n}{simplify}\PY{p}{(}\PY{n}{x}\PY{p}{)} \PY{k}{for} \PY{n}{x} \PY{o+ow}{in} \PY{n}{monomial\PYZus{}set}\PY{p}{]}\PY{p}{)}\PY{p}{)}
\end{Verbatim}
\end{tcolorbox}

    The symmetric group \(G=S_6\) acts on the example matroid. We now set up
the appropriate functions required to compute the actions on the vertex
set and obtain the set of orbits and their stabilizer groups.

    \begin{tcolorbox}[breakable, size=fbox, boxrule=1pt, pad at break*=1mm,colback=cellbackground, colframe=cellborder]
\prompt{In}{incolor}{19}{\boxspacing}
\begin{Verbatim}[commandchars=\\\{\}]
\PY{n}{G} \PY{o}{=} \PY{n}{SymmetricGroup}\PY{p}{(}\PY{n+nb}{range}\PY{p}{(}\PY{n}{n}\PY{p}{)}\PY{p}{)}

\PY{k}{def} \PY{n+nf}{action\PYZus{}on\PYZus{}flats}\PY{p}{(}\PY{n}{g}\PY{p}{,} \PY{n}{m}\PY{p}{)}\PY{p}{:}
    \PY{k}{def} \PY{n+nf}{action\PYZus{}on\PYZus{}groundset}\PY{p}{(}\PY{n}{g}\PY{p}{,} \PY{n}{x}\PY{p}{)}\PY{p}{:}
        \PY{k}{return} \PY{n+nb}{tuple}\PY{p}{(}\PY{n+nb}{sorted}\PY{p}{(}\PY{n}{g}\PY{p}{(}\PY{n}{y}\PY{p}{)} \PY{k}{for} \PY{n}{y} \PY{o+ow}{in} \PY{n}{x}\PY{p}{)}\PY{p}{)}
    \PY{k}{return} \PY{n+nb}{frozenset}\PY{p}{(}\PY{n+nb}{sorted}\PY{p}{(}\PY{p}{[}\PY{n}{action\PYZus{}on\PYZus{}groundset}\PY{p}{(}\PY{n}{g}\PY{p}{,}\PY{n}{x}\PY{p}{)} \PY{k}{for} \PY{n}{x} \PY{o+ow}{in} \PY{n}{m}\PY{p}{]}\PY{p}{)}\PY{p}{)}

\PY{k}{def} \PY{n+nf}{action\PYZus{}on\PYZus{}braidfymonomials}\PY{p}{(}\PY{n}{g}\PY{p}{,} \PY{n}{monomial}\PY{p}{)}\PY{p}{:}
    \PY{k}{return} \PY{n+nb}{tuple}\PY{p}{(}\PY{n+nb}{sorted}\PY{p}{(}\PY{p}{[}\PY{n}{action\PYZus{}on\PYZus{}flats}\PY{p}{(}\PY{n}{g}\PY{p}{,}\PY{n}{m}\PY{p}{)} \PY{k}{for} \PY{n}{m} \PY{o+ow}{in} \PY{n}{monomial}\PY{p}{]}\PY{p}{)}\PY{p}{)}

\PY{k}{def} \PY{n+nf}{stab}\PY{p}{(}\PY{n}{G}\PY{p}{,} \PY{n}{m}\PY{p}{,} \PY{n}{action}\PY{p}{)}\PY{p}{:}
    \PY{k}{return} \PY{n}{G}\PY{o}{.}\PY{n}{subgroup}\PY{p}{(}\PY{n+nb}{set}\PY{p}{(}\PY{n}{g} \PY{k}{for} \PY{n}{g} \PY{o+ow}{in} \PY{n}{G} \PY{k}{if} \PY{n}{action}\PY{p}{(}\PY{n}{g}\PY{p}{,} \PY{n}{m}\PY{p}{)} \PY{o}{==} \PY{n+nb}{tuple}\PY{p}{(}\PY{n}{m}\PY{p}{)}\PY{p}{)}\PY{p}{)}

\PY{k}{def} \PY{n+nf}{orbit}\PY{p}{(}\PY{n}{G}\PY{p}{,} \PY{n}{m}\PY{p}{,} \PY{n}{action}\PY{p}{)}\PY{p}{:}
    \PY{k}{return} \PY{n+nb}{frozenset}\PY{p}{(}\PY{n+nb}{sorted}\PY{p}{(}\PY{n}{action}\PY{p}{(}\PY{n}{g}\PY{p}{,} \PY{n}{m}\PY{p}{)} \PY{k}{for} \PY{n}{g} \PY{o+ow}{in} \PY{n}{G}\PY{p}{)}\PY{p}{)}
    
\PY{k}{def} \PY{n+nf}{orbits}\PY{p}{(}\PY{n}{G}\PY{p}{,} \PY{n}{X}\PY{p}{,} \PY{n}{action}\PY{p}{)}\PY{p}{:}
    \PY{k}{return} \PY{n+nb}{set}\PY{p}{(}\PY{n}{orbit}\PY{p}{(}\PY{n}{G}\PY{p}{,} \PY{n}{x}\PY{p}{,} \PY{n}{action}\PY{p}{)} \PY{k}{for} \PY{n}{x} \PY{o+ow}{in} \PY{n}{X}\PY{p}{)}
\end{Verbatim}
\end{tcolorbox}

    Finally, we compute the orbits of the fy-monomials under the action of
\(G\).

    \begin{tcolorbox}[breakable, size=fbox, boxrule=1pt, pad at break*=1mm,colback=cellbackground, colframe=cellborder]
\prompt{In}{incolor}{20}{\boxspacing}
\begin{Verbatim}[commandchars=\\\{\}]
\PY{n}{fy\PYZus{}monomials\PYZus{}orbits} \PY{o}{=} \PY{p}{[}\PY{n}{orbits}\PY{p}{(}\PY{n}{G}\PY{p}{,} \PY{n}{fy\PYZus{}monomials\PYZus{}list}\PY{p}{[}\PY{n}{i}\PY{p}{]}\PY{p}{,} \PY{n}{action\PYZus{}on\PYZus{}braidfymonomials}\PY{p}{)}
    \PY{k}{for} \PY{n}{i} \PY{o+ow}{in} \PY{n+nb}{range}\PY{p}{(}\PY{n}{rank}\PY{p}{)}
    \PY{p}{]}
    
\PY{n}{orbits\PYZus{}dict} \PY{o}{=} \PY{p}{\PYZob{}}\PY{p}{\PYZcb{}}
\PY{k}{for} \PY{n}{idx}\PY{p}{,} \PY{n}{orbit\PYZus{}set} \PY{o+ow}{in} \PY{n+nb}{enumerate}\PY{p}{(}\PY{n}{fy\PYZus{}monomials\PYZus{}orbits}\PY{p}{)}\PY{p}{:}
    \PY{n+nb}{print}\PY{p}{(}\PY{l+s+sa}{f}\PY{l+s+s2}{\PYZdq{}}\PY{l+s+se}{\PYZbs{}n}\PY{l+s+s2}{rank: }\PY{l+s+si}{\PYZob{}}\PY{n}{idx}\PY{l+s+si}{\PYZcb{}}\PY{l+s+s2}{\PYZdq{}}\PY{p}{)}
    \PY{k}{for} \PY{n}{x} \PY{o+ow}{in} \PY{n}{orbit\PYZus{}set}\PY{p}{:}
        \PY{n}{orbits\PYZus{}dict}\PY{p}{[}\PY{n}{x}\PY{p}{]} \PY{o}{=} \PY{n+nb}{len}\PY{p}{(}\PY{n}{orbits\PYZus{}dict}\PY{p}{)}
        \PY{n+nb}{print}\PY{p}{(}\PY{l+s+sa}{f}\PY{l+s+s2}{\PYZdq{}}\PY{l+s+si}{\PYZob{}}\PY{n}{orbits\PYZus{}dict}\PY{p}{[}\PY{n}{x}\PY{p}{]}\PY{l+s+si}{\PYZcb{}}\PY{l+s+s2}{: }\PY{l+s+si}{\PYZob{}}\PY{n}{set\PYZus{}simplify}\PY{p}{(}\PY{n}{x}\PY{p}{)}\PY{l+s+si}{\PYZcb{}}\PY{l+s+se}{\PYZbs{}n}\PY{l+s+s2}{\PYZdq{}}\PY{p}{)}
\end{Verbatim}
\end{tcolorbox}

    \begin{Verbatim}[commandchars=\\\{\}]

rank: 0
0: \{()\}


rank: 1
1: \{(176,), (182,), (185,), (175,), (188,), (194,), (197,), (187,), (200,),
(190,), (180,), (193,), (199,), (173,), (179,)\}

2: \{(202,)\}

3: \{(118,), (121,), (127,), (124,), (130,), (133,), (139,), (142,), (145,),
(84,), (90,), (148,), (93,), (99,), (160,), (96,), (102,), (108,), (166,),
(105,), (169,), (114,), (117,), (126,), (132,), (129,), (135,), (138,), (89,),
(141,), (86,), (150,), (95,), (98,), (165,), (162,), (168,), (113,), (110,),
(116,), (119,), (128,), (131,), (140,), (85,), (143,), (149,), (94,), (152,),
(91,), (97,), (158,), (155,), (161,), (100,), (164,), (109,), (115,), (167,),
(112,)\}

4: \{(186,), (172,), (201,), (178,), (171,), (174,)\}

5: \{(57,), (63,), (75,), (17,), (26,), (35,), (56,), (62,), (16,), (74,), (80,),
(19,), (77,), (28,), (43,), (58,), (67,), (18,), (27,), (36,)\}

6: \{(192,), (198,), (195,), (191,), (184,), (181,), (177,), (196,), (183,),
(189,)\}

7: \{(51,), (60,), (66,), (69,), (72,), (78,), (23,), (20,), (29,), (32,), (38,),
(44,), (41,), (47,), (50,), (53,), (59,), (68,), (65,), (71,), (25,), (22,),
(31,), (37,), (34,), (40,), (49,), (46,), (52,), (55,), (61,), (64,), (70,),
(73,), (76,), (21,), (79,), (24,), (30,), (33,), (42,), (39,), (45,), (48,),
(54,)\}

8: \{(147,), (153,), (137,), (101,), (146,), (120,), (104,), (154,), (123,),
(136,), (151,), (103,), (144,), (122,), (134,)\}

9: \{(163,), (92,), (82,), (111,), (156,), (170,), (159,), (88,), (107,), (81,),
(87,), (106,), (125,), (83,), (157,)\}


rank: 2
10: \{(31, 191), (70, 183), (78, 196), (50, 198), (61, 198), (30, 192), (33,
191), (76, 189), (71, 184), (76, 198), (20, 184), (22, 181), (40, 196), (47,
183), (66, 181), (51, 192), (49, 183), (73, 195), (78, 191), (33, 177), (70,
196), (42, 189), (68, 189), (29, 191), (23, 184), (61, 177), (37, 195), (30,
189), (41, 195), (32, 192), (46, 191), (23, 177), (38, 196), (72, 191), (55,
184), (21, 181), (47, 198), (25, 177), (51, 198), (38, 189), (72, 184), (22,
183), (53, 198), (65, 181), (41, 181), (24, 183), (66, 198), (44, 195), (49,
191), (79, 192), (21, 184), (48, 195), (59, 177), (64, 191), (60, 196), (52,
192), (53, 184), (34, 189), (45, 198), (69, 183), (69, 192), (20, 183), (45,
191), (37, 196), (52, 196), (48, 183), (46, 195), (39, 189), (54, 196), (32,
177), (59, 195), (25, 181), (65, 192), (73, 184), (64, 181), (40, 181), (34,
177), (60, 177), (42, 181), (71, 189), (29, 192), (31, 189), (55, 192), (24,
177), (50, 196), (44, 198), (68, 183), (79, 195), (54, 184), (39, 195)\}

11: \{(187, 187), (200, 200), (180, 180), (194, 194), (197, 197), (190, 190),
(179, 179), (199, 199), (176, 176), (173, 173), (193, 193), (182, 182), (175,
175), (188, 188), (185, 185)\}

12: \{(21, 202), (54, 202), (23, 202), (34, 202), (45, 202), (78, 202), (60,
202), (47, 202), (69, 202), (51, 202), (25, 202), (38, 202), (49, 202), (71,
202), (29, 202), (40, 202), (73, 202), (55, 202), (20, 202), (31, 202), (53,
202), (64, 202), (22, 202), (42, 202), (33, 202), (44, 202), (66, 202), (24,
202), (46, 202), (68, 202), (79, 202), (72, 202), (48, 202), (39, 202), (37,
202), (59, 202), (70, 202), (50, 202), (61, 202), (30, 202), (41, 202), (52,
202), (65, 202), (76, 202), (32, 202)\}

13: \{(36, 196), (28, 192), (43, 198), (77, 181), (80, 177), (67, 189), (56,
198), (58, 195), (16, 177), (57, 196), (18, 183), (63, 191), (62, 192), (17,
181), (19, 184), (27, 191), (74, 184), (75, 183), (26, 189), (35, 195)\}

14: \{(87, 87), (107, 107), (125, 125), (106, 106), (156, 156), (81, 81), (163,
163), (83, 83), (92, 92), (170, 170), (157, 157), (82, 82), (159, 159), (111,
111), (88, 88)\}

15: \{(164, 202), (142, 202), (100, 202), (133, 202), (109, 202), (155, 202),
(166, 202), (91, 202), (89, 202), (102, 202), (124, 202), (135, 202), (93, 202),
(126, 202), (148, 202), (161, 202), (168, 202), (113, 202), (160, 202), (95,
202), (117, 202), (115, 202), (128, 202), (139, 202), (86, 202), (97, 202), (84,
202), (119, 202), (141, 202), (150, 202), (110, 202), (143, 202), (152, 202),
(121, 202), (108, 202), (130, 202), (132, 202), (90, 202), (112, 202), (99,
202), (145, 202), (165, 202), (167, 202), (114, 202), (169, 202), (158, 202),
(116, 202), (127, 202), (138, 202), (149, 202), (85, 202), (96, 202), (94, 202),
(105, 202), (118, 202), (129, 202), (98, 202), (131, 202), (140, 202), (162,
202)\}

16: \{(146, 202), (134, 202), (137, 202), (153, 202), (154, 202), (144, 202),
(120, 202), (104, 202), (101, 202), (123, 202), (151, 202), (103, 202), (136,
202), (147, 202), (122, 202)\}

17: \{(22, 176), (54, 190), (64, 200), (34, 190), (47, 187), (55, 194), (42,
194), (46, 182), (60, 199), (33, 188), (79, 185), (68, 200), (40, 187), (73,
193), (37, 173), (71, 199), (72, 197), (49, 194), (50, 176), (44, 175), (52,
182), (51, 180), (59, 200), (41, 188), (61, 197), (66, 193), (76, 185), (23,
179), (45, 179), (48, 190), (78, 185), (30, 175), (65, 199), (70, 193), (69,
197), (25, 182), (24, 180), (32, 187), (29, 173), (38, 179), (21, 175), (20,
173), (39, 180), (31, 176), (53, 188)\}

18: \{(202, 202)\}

19: \{(123, 123), (136, 136), (134, 134), (154, 154), (103, 103), (151, 151),
(144, 144), (153, 153), (137, 137), (146, 146), (147, 147), (122, 122), (120,
120), (104, 104), (101, 101)\}

20: \{(163, 202), (159, 202), (170, 202), (92, 202), (87, 202), (82, 202), (106,
202), (88, 202), (83, 202), (107, 202), (81, 202), (111, 202), (125, 202), (156,
202), (157, 202)\}

21: \{(178, 178), (171, 171), (174, 174), (201, 201), (186, 186), (172, 172)\}

22: \{(198, 198), (196, 196), (177, 177), (181, 181), (183, 183), (192, 192),
(189, 189), (191, 191), (195, 195), (184, 184)\}

23: \{(98, 98), (162, 162), (139, 139), (84, 84), (116, 116), (126, 126), (112,
112), (167, 167), (89, 89), (135, 135), (121, 121), (130, 130), (131, 131),
(140, 140), (117, 117), (85, 85), (94, 94), (149, 149), (158, 158), (145, 145),
(168, 168), (90, 90), (99, 99), (108, 108), (164, 164), (86, 86), (141, 141),
(150, 150), (118, 118), (127, 127), (95, 95), (113, 113), (114, 114), (169,
169), (91, 91), (155, 155), (132, 132), (100, 100), (109, 109), (110, 110),
(119, 119), (160, 160), (96, 96), (105, 105), (128, 128), (124, 124), (165,
165), (133, 133), (142, 142), (143, 143), (152, 152), (129, 129), (138, 138),
(97, 97), (115, 115), (161, 161), (93, 93), (148, 148), (166, 166), (102, 102)\}

24: \{(34, 193), (47, 199), (79, 179), (49, 199), (31, 193), (78, 180), (59,
176), (61, 173), (64, 190), (41, 179), (33, 175), (59, 194), (76, 173), (76,
182), (66, 190), (46, 187), (40, 180), (72, 187), (21, 185), (69, 179), (69,
188), (54, 200), (25, 173), (72, 180), (78, 175), (42, 173), (45, 190), (55,
200), (45, 199), (68, 182), (48, 179), (21, 180), (39, 197), (37, 197), (65,
179), (32, 176), (52, 188), (50, 200), (41, 197), (33, 193), (64, 180), (70,
175), (47, 182), (30, 193), (60, 194), (51, 200), (73, 176), (53, 182), (22,
185), (53, 200), (66, 173), (49, 175), (79, 176), (68, 188), (24, 185), (73,
187), (29, 193), (37, 194), (44, 199), (34, 173), (39, 187), (48, 199), (32,
193), (65, 190), (55, 176), (20, 185), (40, 197), (60, 175), (51, 190), (42,
197), (44, 194), (71, 187), (54, 180), (29, 190), (31, 187), (70, 188), (24,
175), (50, 194), (61, 194), (30, 188), (52, 200), (46, 199), (23, 176), (23,
185), (25, 185), (71, 182), (38, 188), (20, 182), (22, 179), (38, 197)\}

25: \{(63, 202), (80, 202), (56, 202), (67, 202), (36, 202), (58, 202), (27,
202), (16, 202), (18, 202), (62, 202), (77, 202), (75, 202), (35, 202), (57,
202), (26, 202), (17, 202), (28, 202), (19, 202), (74, 202), (43, 202)\}

26: \{(60, 172), (72, 201), (30, 174), (41, 186), (59, 201), (70, 201), (25,
178), (22, 172), (21, 174), (23, 171), (55, 178), (34, 186), (42, 178), (24,
172), (46, 178), (39, 172), (29, 171), (40, 171), (22, 174), (70, 178), (41,
172), (64, 201), (68, 171), (48, 186), (66, 201), (59, 171), (52, 174), (30,
171), (51, 178), (71, 172), (76, 186), (38, 178), (49, 178), (31, 172), (20,
172), (21, 171), (69, 174), (34, 174), (47, 171), (78, 186), (54, 186), (33,
172), (66, 178), (38, 171), (60, 201), (31, 174), (40, 186), (64, 171), (50,
172), (37, 172), (55, 186), (73, 201), (65, 172), (48, 174), (33, 186), (79,
201), (45, 178), (61, 174), (32, 171), (29, 172), (73, 178), (54, 174), (45,
171), (47, 186), (69, 201), (24, 178), (39, 178), (53, 186), (71, 201), (52,
178), (46, 174), (44, 174), (42, 186), (72, 174), (37, 171), (51, 172), (50,
174), (32, 186), (61, 201), (53, 172), (65, 201), (76, 201), (78, 201), (25,
174), (20, 171), (49, 186), (44, 171), (79, 186), (68, 201), (23, 178)\}

27: \{(62, 200), (63, 193), (63, 199), (27, 175), (67, 193), (77, 185), (17,
180), (77, 197), (80, 193), (19, 180), (27, 187), (56, 199), (27, 190), (35,
179), (58, 199), (75, 185), (16, 175), (26, 173), (75, 188), (18, 175), (35,
194), (63, 180), (57, 200), (26, 188), (57, 197), (62, 193), (17, 173), (36,
180), (77, 190), (28, 176), (17, 179), (19, 176), (19, 182), (28, 188), (43,
182), (57, 175), (74, 185), (74, 200), (35, 187), (43, 194), (75, 199), (26,
187), (56, 173), (67, 182), (58, 176), (56, 200), (80, 185), (36, 188), (36,
194), (67, 197), (80, 194), (28, 190), (58, 197), (74, 187), (16, 173), (16,
176), (18, 179), (43, 190), (62, 179), (18, 182)\}

28: \{(67, 178), (17, 171), (36, 178), (80, 178), (28, 174), (74, 171), (43,
174), (19, 174), (28, 186), (74, 186), (43, 186), (16, 172), (74, 201), (18,
178), (62, 178), (67, 174), (56, 171), (58, 174), (27, 174), (63, 201), (27,
171), (36, 186), (75, 172), (80, 186), (27, 186), (56, 201), (67, 201), (26,
172), (58, 201), (16, 171), (35, 178), (80, 201), (16, 174), (18, 171), (62,
171), (18, 174), (77, 174), (17, 172), (28, 172), (77, 186), (19, 172), (62,
201), (17, 178), (19, 178), (75, 186), (57, 174), (77, 201), (35, 171), (43,
178), (57, 171), (26, 171), (35, 186), (63, 172), (75, 201), (26, 186), (63,
178), (56, 172), (57, 201), (36, 172), (58, 172)\}


rank: 3
29: \{(195, 195, 195), (198, 198, 198), (196, 196, 196), (189, 189, 189), (184,
184, 184), (177, 177, 177), (191, 191, 191), (183, 183, 183), (192, 192, 192),
(181, 181, 181)\}

30: \{(82, 82, 202), (163, 163, 202), (81, 81, 202), (107, 107, 202), (125, 125,
202), (88, 88, 202), (92, 92, 202), (159, 159, 202), (111, 111, 202), (157, 157,
202), (156, 156, 202), (83, 83, 202), (87, 87, 202), (106, 106, 202), (170, 170,
202)\}

31: \{(98, 98, 202), (124, 124, 202), (143, 143, 202), (105, 105, 202), (85, 85,
202), (155, 155, 202), (133, 133, 202), (89, 89, 202), (115, 115, 202), (160,
160, 202), (84, 84, 202), (129, 129, 202), (148, 148, 202), (126, 126, 202),
(152, 152, 202), (90, 90, 202), (135, 135, 202), (116, 116, 202), (94, 94, 202),
(142, 142, 202), (161, 161, 202), (139, 139, 202), (165, 165, 202), (86, 86,
202), (112, 112, 202), (131, 131, 202), (95, 95, 202), (138, 138, 202), (118,
118, 202), (121, 121, 202), (99, 99, 202), (110, 110, 202), (91, 91, 202), (117,
117, 202), (162, 162, 202), (114, 114, 202), (140, 140, 202), (166, 166, 202),
(130, 130, 202), (108, 108, 202), (168, 168, 202), (149, 149, 202), (127, 127,
202), (100, 100, 202), (141, 141, 202), (97, 97, 202), (164, 164, 202), (167,
167, 202), (145, 145, 202), (96, 96, 202), (93, 93, 202), (119, 119, 202), (158,
158, 202), (102, 102, 202), (169, 169, 202), (109, 109, 202), (150, 150, 202),
(128, 128, 202), (132, 132, 202), (113, 113, 202)\}

32: \{(69, 202, 202), (47, 202, 202), (29, 202, 202), (41, 202, 202), (31, 202,
202), (53, 202, 202), (70, 202, 202), (42, 202, 202), (54, 202, 202), (66, 202,
202), (78, 202, 202), (44, 202, 202), (38, 202, 202), (50, 202, 202), (65, 202,
202), (55, 202, 202), (39, 202, 202), (45, 202, 202), (68, 202, 202), (52, 202,
202), (30, 202, 202), (34, 202, 202), (40, 202, 202), (24, 202, 202), (46, 202,
202), (51, 202, 202), (25, 202, 202), (64, 202, 202), (48, 202, 202), (20, 202,
202), (59, 202, 202), (71, 202, 202), (37, 202, 202), (49, 202, 202), (61, 202,
202), (76, 202, 202), (21, 202, 202), (32, 202, 202), (22, 202, 202), (72, 202,
202), (73, 202, 202), (79, 202, 202), (33, 202, 202), (60, 202, 202), (23, 202,
202)\}

33: \{(194, 194, 194), (179, 179, 179), (187, 187, 187), (175, 175, 175), (200,
200, 200), (193, 193, 193), (180, 180, 180), (182, 182, 182), (176, 176, 176),
(185, 185, 185), (188, 188, 188), (173, 173, 173), (190, 190, 190), (197, 197,
197), (199, 199, 199)\}

34: \{(174, 174, 174), (201, 201, 201), (172, 172, 172), (178, 178, 178), (171,
171, 171), (186, 186, 186)\}

35: \{(202, 202, 202)\}

36: \{(19, 202, 202), (43, 202, 202), (56, 202, 202), (77, 202, 202), (16, 202,
202), (28, 202, 202), (27, 202, 202), (63, 202, 202), (74, 202, 202), (62, 202,
202), (18, 202, 202), (75, 202, 202), (80, 202, 202), (58, 202, 202), (36, 202,
202), (26, 202, 202), (57, 202, 202), (35, 202, 202), (67, 202, 202), (17, 202,
202)\}

37: \{(104, 104, 202), (123, 123, 202), (120, 120, 202), (101, 101, 202), (136,
136, 202), (147, 147, 202), (154, 154, 202), (146, 146, 202), (153, 153, 202),
(137, 137, 202), (151, 151, 202), (122, 122, 202), (103, 103, 202), (134, 134,
202), (144, 144, 202)\}


rank: 4
38: \{(202, 202, 202, 202)\}

    \end{Verbatim}

    Now, we compute the stabilizer of each orbit fy\_monomials\_orbits(rank)
for all ranks.

    \begin{tcolorbox}[breakable, size=fbox, boxrule=1pt, pad at break*=1mm,colback=cellbackground, colframe=cellborder]
\prompt{In}{incolor}{21}{\boxspacing}
\begin{Verbatim}[commandchars=\\\{\}]
\PY{k}{for} \PY{n}{orbits} \PY{o+ow}{in} \PY{n}{fy\PYZus{}monomials\PYZus{}orbits}\PY{p}{:}
    \PY{n}{fn} \PY{o}{=} \PY{n}{ClassFunction}\PY{p}{(}\PY{n}{G}\PY{p}{,} \PY{p}{[}\PY{l+m+mi}{0}\PY{p}{]} \PY{o}{*} \PY{n+nb}{len}\PY{p}{(}\PY{n}{G}\PY{o}{.}\PY{n}{conjugacy\PYZus{}classes}\PY{p}{(}\PY{p}{)}\PY{p}{)}\PY{p}{)}
    \PY{n}{orbits\PYZus{}simplify} \PY{o}{=} \PY{p}{\PYZob{}}\PY{n}{orbits\PYZus{}dict}\PY{p}{[}\PY{n}{orbit}\PY{p}{]} \PY{k}{for} \PY{n}{orbit} \PY{o+ow}{in} \PY{n}{orbits}\PY{p}{\PYZcb{}}

    \PY{k}{for} \PY{n}{orbit} \PY{o+ow}{in} \PY{n}{orbits}\PY{p}{:}
        \PY{n}{orbit\PYZus{}num} \PY{o}{=} \PY{n}{orbits\PYZus{}dict}\PY{p}{[}\PY{n}{orbit}\PY{p}{]}
        \PY{n}{orbit\PYZus{}stab} \PY{o}{=} \PY{n}{stab}\PY{p}{(}\PY{n}{G}\PY{p}{,} \PY{n+nb}{tuple}\PY{p}{(}\PY{n}{orbit}\PY{p}{)}\PY{p}{[}\PY{l+m+mi}{0}\PY{p}{]}\PY{p}{,} \PY{n}{action\PYZus{}on\PYZus{}braidfymonomials}\PY{p}{)}
        \PY{n+nb}{print}\PY{p}{(}\PY{l+s+sa}{f}\PY{l+s+s2}{\PYZdq{}}\PY{l+s+s2}{Orbit }\PY{l+s+si}{\PYZob{}}\PY{n}{orbit\PYZus{}num}\PY{l+s+si}{\PYZcb{}}\PY{l+s+s2}{ is stabilized by a subgroup of order }\PY{l+s+si}{\PYZob{}}\PY{n}{orbit\PYZus{}stab}\PY{o}{.}\PY{n}{order}\PY{p}{(}\PY{p}{)}\PY{l+s+si}{\PYZcb{}}\PY{l+s+s2}{\PYZdq{}}\PY{p}{)}
        \PY{n}{fn} \PY{o}{+}\PY{o}{=} \PY{n}{orbit\PYZus{}stab}\PY{o}{.}\PY{n}{trivial\PYZus{}character}\PY{p}{(}\PY{p}{)}\PY{o}{.}\PY{n}{induct}\PY{p}{(}\PY{n}{G}\PY{p}{)}

    \PY{n+nb}{print}\PY{p}{(}\PY{l+s+sa}{f}\PY{l+s+s2}{\PYZdq{}}\PY{l+s+se}{\PYZbs{}n}\PY{l+s+s2}{The permutation representation for }\PY{l+s+si}{\PYZob{}}\PY{n}{orbits\PYZus{}simplify}\PY{l+s+si}{\PYZcb{}}\PY{l+s+s2}{ is}\PY{l+s+s2}{\PYZdq{}}\PY{p}{)}
    \PY{k}{for} \PY{n}{x}\PY{p}{,} \PY{n}{y} \PY{o+ow}{in} \PY{n}{fn}\PY{o}{.}\PY{n}{decompose}\PY{p}{(}\PY{p}{)}\PY{p}{:}
        \PY{n+nb}{print}\PY{p}{(}\PY{l+s+sa}{f}\PY{l+s+s2}{\PYZdq{}}\PY{l+s+si}{\PYZob{}}\PY{n}{x}\PY{l+s+si}{\PYZcb{}}\PY{l+s+s2}{ copies of }\PY{l+s+si}{\PYZob{}}\PY{n+nb}{list}\PY{p}{(}\PY{n}{y}\PY{o}{.}\PY{n}{values}\PY{p}{(}\PY{p}{)}\PY{p}{)}\PY{l+s+si}{\PYZcb{}}\PY{l+s+s2}{\PYZdq{}}\PY{p}{)}
    \PY{n+nb}{print}\PY{p}{(}\PY{l+s+s2}{\PYZdq{}}\PY{l+s+se}{\PYZbs{}n}\PY{l+s+s2}{\PYZdq{}}\PY{p}{)}
\end{Verbatim}
\end{tcolorbox}

    \begin{Verbatim}[commandchars=\\\{\}]
Orbit 0 is stabilized by a subgroup of order 720

The permutation representation for \{0\} is
1 copies of [1, 1, 1, 1, 1, 1, 1, 1, 1, 1, 1]


Orbit 1 is stabilized by a subgroup of order 48
Orbit 2 is stabilized by a subgroup of order 720
Orbit 3 is stabilized by a subgroup of order 12
Orbit 4 is stabilized by a subgroup of order 120
Orbit 5 is stabilized by a subgroup of order 36
Orbit 6 is stabilized by a subgroup of order 72
Orbit 7 is stabilized by a subgroup of order 16
Orbit 8 is stabilized by a subgroup of order 48
Orbit 9 is stabilized by a subgroup of order 48

The permutation representation for \{1, 2, 3, 4, 5, 6, 7, 8, 9\} is
9 copies of [1, 1, 1, 1, 1, 1, 1, 1, 1, 1, 1]
2 copies of [5, -1, 1, 3, -1, -1, 2, 1, -1, 0, 0]
2 copies of [5, 1, 1, -3, -1, 1, 2, -1, -1, 0, 0]
7 copies of [5, 3, 1, -1, 2, 0, -1, 1, -1, 0, -1]
9 copies of [9, 3, 1, 3, 0, 0, 0, -1, 1, -1, 0]
1 copies of [10, 2, -2, -2, 1, -1, 1, 0, 0, 0, 1]
2 copies of [16, 0, 0, 0, -2, 0, -2, 0, 0, 1, 0]


Orbit 10 is stabilized by a subgroup of order 8
Orbit 11 is stabilized by a subgroup of order 48
Orbit 12 is stabilized by a subgroup of order 16
Orbit 13 is stabilized by a subgroup of order 36
Orbit 14 is stabilized by a subgroup of order 48
Orbit 15 is stabilized by a subgroup of order 12
Orbit 16 is stabilized by a subgroup of order 48
Orbit 17 is stabilized by a subgroup of order 16
Orbit 18 is stabilized by a subgroup of order 720
Orbit 19 is stabilized by a subgroup of order 48
Orbit 20 is stabilized by a subgroup of order 48
Orbit 21 is stabilized by a subgroup of order 120
Orbit 22 is stabilized by a subgroup of order 72
Orbit 23 is stabilized by a subgroup of order 12
Orbit 24 is stabilized by a subgroup of order 8
Orbit 25 is stabilized by a subgroup of order 36
Orbit 26 is stabilized by a subgroup of order 8
Orbit 27 is stabilized by a subgroup of order 12
Orbit 28 is stabilized by a subgroup of order 12

The permutation representation for \{10, 11, 12, 13, 14, 15, 16, 17, 18, 19, 20,
21, 22, 23, 24, 25, 26, 27, 28\} is
19 copies of [1, 1, 1, 1, 1, 1, 1, 1, 1, 1, 1]
7 copies of [5, -1, 1, 3, -1, -1, 2, 1, -1, 0, 0]
8 copies of [5, 1, 1, -3, -1, 1, 2, -1, -1, 0, 0]
21 copies of [5, 3, 1, -1, 2, 0, -1, 1, -1, 0, -1]
1 copies of [9, -3, 1, -3, 0, 0, 0, 1, 1, -1, 0]
28 copies of [9, 3, 1, 3, 0, 0, 0, -1, 1, -1, 0]
1 copies of [10, -2, -2, 2, 1, 1, 1, 0, 0, 0, -1]
7 copies of [10, 2, -2, -2, 1, -1, 1, 0, 0, 0, 1]
12 copies of [16, 0, 0, 0, -2, 0, -2, 0, 0, 1, 0]


Orbit 29 is stabilized by a subgroup of order 72
Orbit 30 is stabilized by a subgroup of order 48
Orbit 31 is stabilized by a subgroup of order 12
Orbit 32 is stabilized by a subgroup of order 16
Orbit 33 is stabilized by a subgroup of order 48
Orbit 34 is stabilized by a subgroup of order 120
Orbit 35 is stabilized by a subgroup of order 720
Orbit 36 is stabilized by a subgroup of order 36
Orbit 37 is stabilized by a subgroup of order 48

The permutation representation for \{32, 33, 34, 35, 36, 37, 29, 30, 31\} is
9 copies of [1, 1, 1, 1, 1, 1, 1, 1, 1, 1, 1]
2 copies of [5, -1, 1, 3, -1, -1, 2, 1, -1, 0, 0]
2 copies of [5, 1, 1, -3, -1, 1, 2, -1, -1, 0, 0]
7 copies of [5, 3, 1, -1, 2, 0, -1, 1, -1, 0, -1]
9 copies of [9, 3, 1, 3, 0, 0, 0, -1, 1, -1, 0]
1 copies of [10, 2, -2, -2, 1, -1, 1, 0, 0, 0, 1]
2 copies of [16, 0, 0, 0, -2, 0, -2, 0, 0, 1, 0]


Orbit 38 is stabilized by a subgroup of order 720

The permutation representation for \{38\} is
1 copies of [1, 1, 1, 1, 1, 1, 1, 1, 1, 1, 1]


    \end{Verbatim}


    % Add a bibliography block to the postdoc
    
    
    
\end{document}
