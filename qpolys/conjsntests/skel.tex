\documentclass[10pt]{article}
\usepackage[margin=1in]{geometry}
\usepackage{parskip}
\usepackage{amssymb,amsmath,amsfonts,verbatim,amsthm}
\usepackage[breakable,skins]{tcolorbox}
\usepackage{graphicx,tikz-cd,adjustbox}
\usepackage{float,breqn} % breqn can sometimes conflict with amsmath, be mindful
\usepackage{scalerel}
\usepackage{stackengine,wasysym}
\usepackage[backend=biber]{biblatex} % Using biber backend is recommended
\addbibresource{references.bib} % Use \addbibresource with biblatex

\newtcolorbox{mybox}[2][]{
    arc=0mm, enhanced, frame hidden, breakable
}

% Define math commands
\newcommand{\calL}{\mathcal{L}}
\newcommand{\calH}{\mathcal{H}}
\newcommand{\bz}{\mathbb{Z}}
\newcommand{\lt}{\normalfont\text{lt}}
\newcommand{\calA}{\mathcal{A}}
\newcommand{\ind}{\text{\normalfont Ind}}
\newcommand{\stab}{\text{\normalfont Stab}}
\newcommand{\hilbM}{\normalfont\text{\underline{H}}_\text{M}}
\newcommand{\aughilbM}{\normalfont\text{H}_\text{M}}
\newcommand{\rank}{\normalfont\text{rk}}
\newcommand{\flats}{\normalfont\text{Flats}} % Corrected typo \normalfSont -> \normalfont
\newcommand{\flags}{\normalfont\text{Flags}}
\newcommand{\matM}{\normalfont\text{M}}
\newcommand{\hilbMmodF}{\normalfont\text{\underline{H}}_\text{M/F}}

% Theorem environments
\newtheorem{theorem}{Theorem}
\newtheorem{corollary}[theorem]{Corollary}
\newtheorem{lemma}[theorem]{Lemma}
\newtheorem{proposition}[theorem]{Proposition}
\newtheorem{conjecture}[theorem]{Conjecture}

\theoremstyle{remark}
\newtheorem*{remark}{Remark} % Unnumbered remark

\title{Notes (Inverse Kazhdan-Lusztig Polynomials under Deletion)}
%\author{Nutan Nepal}
%\date{\today}

\begin{document}
{\textbf{Notes: Inverse Kazhdan-Lusztig Polynomials under Deletion}}\hfill {\small{\today}}

\hrulefill % Horizontal line after title block

\section{Introduction}

The Kazhdan-Lusztig polynomial $P_M(t)$ is a fundamental invariant associated with any matroid $M$, as defined by Elias, Proudfoot, and Wakefield in \cite{EPW16}. This polynomial, denoted $P_M(t)$, exhibits formal similarities to the Kazhdan–Lusztig polynomials defined for Coxeter groups. The coefficients of $P_M(t)$ depend only on the lattice of flats $L(M)$ of the matroid, and in fact, they are integral linear combinations of the flag Whitney numbers counting chains of flats with specified ranks.

In \cite{Bra19}, Braden and Vysogorets presented a formula that relates the Kazhdan-Lusztig polynomial of a matroid $M$ to that of the matroid obtained by deleting an element $e$, denoted $M\setminus e$, as well as various contractions and localizations of $M$. Specifically, for a simple matroid $M$ where $e$ is not a coloop, their main result, Theorem 2.8, states:
\[
P_M(t) = P_{M\setminus e}(t) - tP_{M/e}(t) + \sum_{F\in S} \tau(M_{F\cup e})\cdot
t^{\text{crk}(F)/2}\cdot P_{M^F}(t)
\]
where the sum is taken over the set $S$ of all subsets $F$ of $E \setminus e$ such that both $F$ and $F \cup e$ are flats of $M$, and $\tau(M)$ is the coefficient of $t^{(\rank(M)-1)/2}$ in $P_M(t)$ if $\rank(M)$ is odd, and zero otherwise.

The inverse Kazhdan-Lusztig polynomial $Q_M(x)$ is another important invariant.
There is a related polynomial $\hat{Q}_M(x) = (-1)^{\rank(M)} Q_M(x)$ which acts as the inverse of the Kazhdan-Lusztig polynomial $P_M(t)$ within the incidence algebra of the lattice of flats under appropriate variable transformation. % Added "(x)" to Q_M and hat{Q}_M for consistency

In this paper, we aim to prove the following deletion formula for $\hat{Q}_M(x)$:
\[
    \hat{Q}_M(x) = \hat{Q}_{M\setminus e}(x) -(1 + x)\cdot \hat{Q}_{M/e}(x) -
    \sum_{G \in S'}{\tau(M^G_e)\cdot x^{\rank(G)/2}\cdot
    \hat{Q}_{M/G}(x)}
\]
where $S' = \{F\in \calL(M)\mid e\in F \text{ and } F\setminus e \notin \calL(M)\}$.

\section{Perverse elements and the KL basis}
\label{sec:algebraic_framework}

Let $M$ be a matroid and $\calL(M)$ be its lattice of flats.

\begin{itemize}
    \item \textbf{The Module $\calH(M)$}: Let $\calH = \calH(M)$ be the
    \textbf{free $\mathbb{Z}[t, t^{-1}]$-module with basis indexed by $\calL(M)$}.
    Elements of $\calH$ are formal sums of the form
    \[
    \alpha = \sum_{F \in \calL(M)} \alpha_F \cdot F, \quad \alpha_F \in \mathbb{Z}
    [t, t^{-1}].
    \]

    \item \textbf{The Abelian Subgroup $\calH_p$}: $\calH_p$ is an
    \textbf{abelian subgroup of $\calH$} consisting of all $\alpha \in \calH$
    such that for every flat $F \in \calL(M)$, the following two conditions hold:
    \begin{itemize}
        \item[i.] $\alpha_F \in \mathbb{Z}[t]$.
        \item[ii.] $\sum_{G \geq F} t^{\rank(F) - \rank(G)} \alpha_G \in Pal(0)$, where $Pal(0)$ is the set of Laurent polynomials $f(t)$ such that $f(t) = f(t^{-1})$.
    \end{itemize}

    \item \textbf{The Elements $\zeta^F$}: For any flat $F \in \calL(M)$, an element
    $\zeta^F \in \calH$ is defined as
    \[
    \zeta^F = \sum_{G \leq F} \zeta_G^F \cdot G = \sum_{G \leq F} t^{\rank(F) - \rank(G)} P_{M_G^F}(t^{-2}) \cdot G.
    \]
    \item \textbf{Basis of $\calH_p$}: Proposition 2.13 states that the set of elements
    $\{\zeta^F\}_{F \in \calL(M)}$ forms a \textbf{$\mathbb{Z}$-basis for $\calH_p$}.
    Any element $\beta \in \calH_p$ can be uniquely expressed as a linear combination of
    the $\zeta^F$ with integer coefficients:
    \[
    \beta = \sum_{F \in \calL(M)} \beta_F(0) \zeta^F.
    \]
\end{itemize}

This algebraic framework, involving the module $\calH(M)$ and its subgroup $\calH_p$ with the basis $\{\zeta^F\}$, provides a foundation for studying the Kazhdan–Lusztig polynomials of matroids, as demonstrated by its role in the derivation of deletion formulas. % Removed dangling [4]

\printbibliography % Ensure you have a 'references.bib' file and run biber/bibtex

\end{document}