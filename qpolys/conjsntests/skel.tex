\documentclass[10pt]{article}
\usepackage[margin=1in]{geometry}
\usepackage{parskip}
\usepackage{amssymb,amsmath,amsfonts,verbatim,amsthm}
\usepackage[dvipsnames]{xcolor}
\usepackage[breakable,skins]{tcolorbox}
\usepackage{graphicx,tikz-cd,adjustbox}
\usepackage{float,breqn} % breqn can sometimes conflict with amsmath, be mindful
\usepackage{scalerel}
\usepackage{stackengine,wasysym}
\usepackage{lipsum}
\usepackage[backend=biber]{biblatex} % Using biber backend is recommended
\addbibresource{references.bib} % Use \addbibresource with biblatex

\newtcolorbox{mybox}[2][]{
    arc=0mm, enhanced, frame hidden, breakable
}

% Define math commands
\newcommand{\calL}{\mathcal{L}}
\newcommand{\calH}{\mathcal{H}}
\newcommand{\bz}{\mathbb{Z}}
\newcommand{\lt}{\normalfont\text{lt}}
\newcommand{\calA}{\mathcal{A}}
\newcommand{\ind}{\text{\normalfont Ind}}
\newcommand{\stab}{\text{\normalfont Stab}}
\newcommand{\hilbM}{\normalfont\text{\underline{H}}_\text{M}}
\newcommand{\aughilbM}{\normalfont\text{H}_\text{M}}
\newcommand{\rank}{\normalfont\text{rk}}
\newcommand{\flats}{\normalfont\text{Flats}}
\newcommand{\flags}{\normalfont\text{Flags}}
\newcommand{\matM}{\normalfont\text{M}}
\newcommand{\hilbMmodF}{\normalfont\text{\underline{H}}_\text{M/F}}

% Theorem environments
\newtheorem{theorem}{Theorem}
\newtheorem{corollary}[theorem]{Corollary}
\newtheorem{lemma}[theorem]{Lemma}
\newtheorem{proposition}[theorem]{Proposition}
\newtheorem{conjecture}[theorem]{Conjecture}

\theoremstyle{remark}
\newtheorem*{remark}{Remark} % Unnumbered remark
% \theoremstyle{definition} % For proof environment styling - amsthm handles this with \begin{proof}
% \newtheorem*{proof_env}{Proof}

\newcommand{\luis}[1]{{\color{blue} \sf $\clubsuit$ Luis: [#1]}}
\newcommand{\nutan}[1]{{\color{violet} \sf $\clubsuit$ Nutan: [#1]}}
\newcommand{\jacob}[1]{{\color{Bittersweet} \sf $\clubsuit$ Jacob: [#1]}}

\title{Notes (Inverse Kazhdan-Lusztig Polynomials under Deletion)}
%\author{Nutan Nepal}
%\date{\today}

\begin{document}
{\textbf{Notes: Inverse Kazhdan-Lusztig Polynomials under Deletion}}\hfill {\small{\today}}

\hrulefill % Horizontal line after title block

\section{Introduction}
The Kazhdan-Lusztig polynomial $P_M(t)$ is a fundamental invariant associated with any matroid $M$,
as defined by Elias, Proudfoot, and Wakefield in \cite{EPW16}. This polynomial, denoted $P_M(t)$,
exhibits formal similarities to the Kazhdan-Lusztig polynomials defined for Coxeter groups.
The coefficients of $P_M(t)$ depend only on the lattice of flats $\calL(M)$ of the matroid,
and in fact, they are integral linear combinations of the flag Whitney numbers counting chains of flats
with specified ranks.

In \cite{Bra19}, Braden and Vysogorets presented a formula that relates the Kazhdan-Lusztig polynomial
of a matroid $M$ to that of the matroid obtained by deleting an element $e$, denoted $M\setminus e$, as well as various contractions and localizations of $M$. Specifically, for a simple matroid $M$ where $e$ is not a coloop, their main result, Theorem 2.8, states:
\[
P_M(t) = P_{M\setminus e}(t) - tP_{M_e}(t) + \sum_{F\in S} \tau(M_{F\cup e})\cdot
t^{\text{crk}(F)/2}\cdot P_{M^F}(t)
\]
where the sum is taken over the set $S$ of all subsets $F$ of $E \setminus e$ such that both $F$ and $F \cup e$ are flats of $M$, and $\tau(M)$ is the coefficient of $t^{(\rank(M)-1)/2}$ in $P_M(t)$ if $\rank(M)$ is odd, and zero otherwise.

The inverse Kazhdan-Lusztig polynomial $Q_M(x)$ is another important invariant.
There is a related polynomial $\hat{Q}_M(x) = (-1)^{\rank(M)} Q_M(x)$ which acts as the inverse of
the Kazhdan-Lusztig polynomial $P_M(t)$ within the incidence algebra of the lattice of flats $\calL(M)$.

In this paper, we aim to prove the following deletion formula for $\hat{Q}_M(x)$:
\begin{theorem}[Deletion Formula for $\hat{Q}_M(x)$]
\label{thm:deletion_formula_Q_hat}
Let $M$ be a simple matroid and $e$ an element that is not a coloop. Then
\[
    \hat{Q}_M(x) = \hat{Q}_{M\setminus e}(x) -(1 + x)\cdot \hat{Q}_{M_e}(x) -
    \sum_{G \in S'}{\tau(M^G_e)\cdot x^{\rank(G)/2}\cdot
    \hat{Q}_{M_G}(x)}
\]
where $S' = \{F\in \calL(M)\mid e\in F \text{ and } F\setminus e \notin \calL(M)\}$.
\end{theorem}
The proof of this theorem is the main goal of these notes.

\section{Perverse elements and the KL basis}
\label{sec:algebraic_framework}

Let $M$ be a matroid and $\calL(M)$ be its lattice of flats.

\begin{itemize}
    \item \textbf{The Module $\calH(M)$}: Let $\calH = \calH(M)$ be the
    \textbf{free $\mathbb{Z}[t, t^{-1}]$-module with basis indexed by $\calL(M)$}.
    Elements of $\calH$ are formal sums of the form
    \[
    \alpha = \sum_{F \in \calL(M)} \alpha_F \cdot F, \quad \alpha_F \in \mathbb{Z}
    [t, t^{-1}].
    \]

    \item \textbf{The Abelian Subgroup $\calH_p$}: $\calH_p$ is an
    \textbf{abelian subgroup of $\calH$} consisting of all $\alpha \in \calH$
    such that for every flat $F \in \calL(M)$, the following two conditions hold:
    \begin{itemize}
        \item[i.] $\alpha_F \in \mathbb{Z}[t]$.
        \item[ii.] $\sum_{G \geq F} t^{\rank(F) - \rank(G)} \alpha_G \in Pal(0)$, where $Pal(0)$ is the set of Laurent polynomials $f(t)$ such that $f(t) = f(t^{-1})$.
    \end{itemize}

    \item \textbf{The Elements $\zeta^F$}: For any flat $F \in \calL(M)$, an element
    $\zeta^F \in \calH$ is defined as
    \[
    \zeta^F = \sum_{G \leq F} t^{\rank(F) - \rank(G)} P_{M_G^F}(t^{-2}) \cdot G.
    \]
    \item \textbf{Basis of $\calH_p$}: Proposition 2.13 of \cite{Bra19} states that the set of elements
    $\{\zeta^F\}_{F \in \calL(M)}$ forms a \textbf{$\mathbb{Z}$-basis for $\calH_p$}.
    Any element $\beta \in \calH_p$ can be uniquely expressed as a linear combination of
    the $\zeta^F$ with integer coefficients:
    \[
    \beta = \sum_{F \in \calL(M)} \beta_F(0) \zeta^F.
    \]
\end{itemize}

This algebraic framework, involving the module $\calH(M)$ and its subgroup $\calH_p$ with the basis $\{\zeta^F\}$, provides a foundation for studying the Kazhdan–Lusztig polynomials of matroids, as demonstrated by its role in the derivation of deletion formulas.

\section{The module homomorphism $\calH(M) \to \calH(M\setminus e)$}
\label{sec:module_homomorphism}
Let $M$ be a simple matroid and $e$ be an element of its ground set that is not a coloop.
There is a surjective map $\calL(M) \to \calL(M\setminus e)$ sending a flat $F \in \calL(M)$ to $F\setminus e \in \calL(M\setminus e)$.
There also exists a module homomorphism $\Delta: \calH(M) \to \calH(M\setminus e)$ which
is $\mathbb{Z}[t, t^{-1}]$-linear and is defined on the basis elements $F \in \calL(M)$ by
\[
\Delta(F) = \begin{cases}
    F & \text{if }F\not\ni e\\
    F\setminus e & \text{if } F\ni e \text{ and }F\setminus e \notin \calL(M) \\
    t^{-1}\cdot (F\setminus e) & \text{if } F\ni e \text{ and }F\setminus e \in \calL(M).
\end{cases}
\]
This definition is taken from Section 2.6 of Braden and Vysogorets \cite{Bra19}.
In \cite{Bra19}, the authors also prove the following lemma:

\begin{lemma}
\label{lem:delta_zeta_nocoloop}
Let $F \in \calL(M)$ be a flat such that $e \in F$ and $F\setminus e \notin \calL(M)$. Then
\[\Delta(\zeta^F) = \zeta^{F\setminus e} + \sum_{G\in S(M^F)}\tau(M^F_{G\cup e})\cdot\zeta^G\]
where $S(M^F) = \{G\in \calL(M^F)\mid e\notin G \text{ and } G\cup e \in \calL(M^F)\}$.
\end{lemma}

\begin{theorem}
\label{thm:delta_zeta_special_case}
Let $F \in \calL(M)$ be a flat such that $e \in F$ and $F\setminus e \in \calL(M)$, then
\[ \Delta(\zeta^F) = (t + t^{-1})\zeta^{F\setminus e} \]
where $\zeta^{F\setminus e}$ on the right-hand side is the $\zeta$-element in
$\calH(M\setminus e)$ associated with the flat $F\setminus e \in \calL(M\setminus e)$.
\end{theorem}
\begin{proof}
Let $F_0 = F\setminus e$. We are given $F_0 \in \calL(M)$.
By the definition of $\Delta(G)$:
\begin{itemize}
    \item If $G \leq F$ and $e \notin G$: Then $G \leq F_0$. Since $G \in \calL(M)$ and $e \notin G$, $G$ is also a flat in $M\setminus e$, so $G \in \calL(M\setminus e)$. In this case, $\Delta(G) = G$.
    \item If $G \leq F$ and $e \in G$: Let $G_0 = G\setminus e$. Then $G_0 \leq F_0$. Since $F_0 \in \calL(M)$, it follows that $G_0 = G \cap F_0 \in \calL(M)$. Thus $G_0 \in \calL(M\setminus e)$. In this case, $\Delta(G) = t^{-1}G_0$.
\end{itemize}
Applying this to $\Delta(\zeta^F)$:
\begin{align*}
\Delta(\zeta^F) &= \Delta \left( \sum_{G \leq F} t^{\rank_M(F) - \rank_M(G)} P_{M_G^F}(t^{-2}) \cdot G \right) \\
&= \sum_{G \leq F, G \not\ni e} t^{\rank_M(F) - \rank_M(G)} P_{M_G^F}(t^{-2}) \cdot \Delta(G) \\
&\quad + \sum_{G \leq F, G \ni e} t^{\rank_M(F) - \rank_M(G)} P_{M_G^F}(t^{-2}) \cdot \Delta(G) \\
&= \sum_{G \leq F_0} t^{\rank_M(F) - \rank_M(G)} P_{M_G^F}(t^{-2}) \cdot G \\
&\quad + \sum_{\substack{G_0 \leq F_0 \\ (G = G_0 \cup \{e\})}} t^{\rank_M(F) - \rank_M(G_0 \cup \{e\})} P_{M_{G_0 \cup \{e\}}^{F}}(t^{-2}) \cdot (t^{-1}G_0).
\end{align*}
We use the rank relations:
\begin{itemize}
    \item $\rank_M(F) = \rank_{M\setminus e}(F_0) + 1$.
    \item For $G \leq F_0$: $\rank_M(G) = \rank_{M\setminus e}(G)$.
    \item For $G_0 \leq F_0$: $\rank_M(G_0 \cup \{e\}) = \rank_{M\setminus e}(G_0) + 1$.
\end{itemize}
And the Kazhdan-Lusztig polynomial identities under these conditions:
\begin{itemize}
    \item For $G \leq F_0$: $M_G^F \cong (M\setminus e)_G^{F_0} \oplus U_{1,1}(\{e\})$, so $P_{M_G^F}(t^{-2}) = P_{(M\setminus e)_G^{F_0}}(t^{-2})$.
    \item For $G_0 \leq F_0$: $M_{G_0 \cup \{e\}}^{F} \cong (M\setminus e)_{G_0}^{F_0}$, so $P_{M_{G_0 \cup \{e\}}^{F}}(t^{-2}) = P_{(M\setminus e)_{G_0}^{F_0}}(t^{-2})$.
\end{itemize}
Substituting these into the sums:
\begin{align*}
\Delta(\zeta^F) &= \sum_{G \leq F_0} t^{(\rank_{M\setminus e}(F_0) + 1) - \rank_{M\setminus e}(G)} P_{(M\setminus e)_G^{F_0}}(t^{-2}) \cdot G \\
&\quad + t^{-1} \sum_{G_0 \leq F_0} t^{(\rank_{M\setminus e}(F_0) + 1) - (\rank_{M\setminus e}(G_0) + 1)} P_{(M\setminus e)_{G_0}^{F_0}}(t^{-2}) \cdot G_0 \\
&= t \sum_{G \leq F_0, G \in \calL(M\setminus e)} t^{\rank_{M\setminus e}(F_0) - \rank_{M\setminus e}(G)} P_{(M\setminus e)_G^{F_0}}(t^{-2}) \cdot G \\
&\quad + t^{-1} \sum_{G_0 \leq F_0, G_0 \in \calL(M\setminus e)} t^{\rank_{M\setminus e}(F_0) - \rank_{M\setminus e}(G_0)} P_{(M\setminus e)_{G_0}^{F_0}}(t^{-2}) \cdot G_0.
\end{align*}
Using the definition of $\zeta^{F_0}_{M\setminus e}$:
\begin{align*}
\Delta(\zeta^F) &= t \cdot \zeta^{F_0}_{M\setminus e} + t^{-1} \cdot \zeta^{F_0}_{M\setminus e} \\
&= (t + t^{-1})\zeta^{F_0}_{M\setminus e}.
\end{align*}
\end{proof}

\section{The Deletion Formula for $\hat{Q}_M(x)$}
\label{sec:deletion_formula}

\begin{lemma}
\label{lem:standard_basis_in_zeta}
    The standard basis $\{F\}_{F \in \calL(M)}$ of $\calH(M)$ satisfies
    \begin{equation}
    \label{cob-lemma}
        F = \sum_{G \leq F} t^{\rank(F) - \rank(G)} \hat{Q}_{M_G^F}(t^{-2}) \cdot \zeta^G.
    \end{equation}
\end{lemma}
\begin{proof}
    \lipsum[1]
\end{proof}

\begin{proof}[Proof of \ref{thm:deletion_formula_Q_hat}]
    Applying the homomorphism $\Delta$ to equation \ref{cob-lemma}
    when $F = E$, we get 
    \[[E\setminus e] = \sum_{G \leq E} t^{\rank(E) - \rank(G)} \hat{Q}_{M_G^E}(t^{-2}) \cdot \Delta(\zeta^G).\] 
    Using lemma \ref{lem:standard_basis_in_zeta}, the coefficient of $\zeta^\emptyset$ in the left-hand side is
    $t^{\rank(E\setminus e) - \rank(\emptyset)}\cdot
    \hat{Q}_{M_\emptyset^{E\setminus e}}(t^{-2}) = t^{\rank(M)}\cdot
    \hat{Q}_{M\setminus e}(t^{-2})$.
    
    By Theorem \ref{thm:delta_zeta_special_case}, the coefficient $\zeta^\emptyset$ on the right-hand side is non-zero only when $G\in \{\emptyset, e\} \cup S'$. In each of these cases, we have the following:
    \begin{itemize}
        \item For $G = \emptyset$, we have $\Delta(\zeta^\emptyset) = \zeta^\emptyset$.
        \item For $G = e$, we have $\Delta(\zeta^e) = (t+t^{-1})\zeta^\emptyset$.
        \item For $G\in S'$, we have $\Delta(\zeta^G) = \tau(M^G_e)\cdot\zeta^\emptyset
            +\text{terms not including $\zeta^\emptyset$}$.
    \end{itemize}
    
    Collecting the coefficients of
    $\zeta^\emptyset$ on the right-hand side, we have:
    \begin{align*}
        &t^{\rank(E) - \rank(\emptyset)}
        \hat{Q}_{M_\emptyset^E}(t^{-2}) + t^{\rank(E) - \rank(e)}(t+t^{-1})
        \hat{Q}_{M_e^E}(t^{-2}) + \sum_{G\in S'} t^{\rank(E) - \rank(G)}
        \hat{Q}_{M_G^E}(t^{-2})\cdot \tau(M^G_e)\\
        =\ &t^{\rank(M)}
        \hat{Q}_{M}(t^{-2}) + t^{\rank(M) - 1}(t+t^{-1})
        \hat{Q}_{M_e}(t^{-2}) + \sum_{G\in S'} t^{\rank(M) - \rank(G)}
        \hat{Q}_{M_G}(t^{-2})\cdot \tau(M^G_e).
    \end{align*}
    Equating the coefficients of $\zeta^\emptyset$ on both sides, we obtain:
    \[\hat{Q}_{M\setminus e}(t^{-2}) = \hat{Q}_{M}(t^{-2}) + (1+t^{-2})
    \hat{Q}_{M_e}(t^{-2}) + \sum_{G\in S'} t^{- \rank(G)}
    \hat{Q}_{M_G}(t^{-2})\cdot \tau(M^G_e).\]
    Finally, taking $x = t^{-2}$ and rearranging yields the desired statement: \[
    \hat{Q}_M(x) = \hat{Q}_{M\setminus e}(x) -(1 + x)\cdot \hat{Q}_{M_e}(x) -
    \sum_{G \in S'}{\tau(M^G_e)\cdot x^{\rank(G)/2}\cdot
    \hat{Q}_{M_G}(x)}.
\]
\end{proof}
The inverse Kazhdan-Lusztig polynomial $Q_M(x) = (-1)^{\rank(M)}\hat{Q}_M(x)$ then satisfies the following:
\begin{corollary}
\label{cor:deletion_formula_Q}
Let $M$ be a simple matroid and $e$ an element that is not a coloop. Then
\[
    Q_M(x) = Q_{M\setminus e}(x) + (1 + x)\cdot Q_{M_e}(x) -
    \sum_{G \in S'}{\tau(M^G_e)\cdot x^{\rank(G)/2} \cdot
    Q_{M_G}(x)}
\]
where $S' = \{F\in \calL(M)\mid e\in F \text{ and } F\setminus e \notin \calL(M)\}$.
\end{corollary}
\begin{corollary}
\label{cor:deletion_formula_Q_uniform}
Let $M:=U_{m,d}$ be the uniform matroid of rank $d$ on $m+d$ elements with $d\geq 1$. Then
\[
    Q_{U_{m,d}}(x) = Q_{U_{m-1,d}}(x) + (1 + x)\cdot Q_{U_{m,d-1}}(x) -
    \tau(U_{m,d-1})\cdot x^{d/2}.
\]
\end{corollary}
\begin{proof}
    As none of the elements in the ground set are coloops, we may apply the deletion formula for a generic element $e$ in the ground set. The set $S'$ then consists only of the top flat $E$ and the sum over $S'$ reduces to \[\tau(M_e)\cdot x^{\rank(M)/2}
    = \tau(U_{m,d-1})\cdot x^{d/2}.\]
\end{proof}

\jacob{For the previous corollary, can you now use induction to get a formula for $Q_{m,d}$ for any $m$ and $d$ that does not have any $Q$'s in it?  (Perhaps starting with the fact the inverse KL polynomials of Boolean matroids are equal to $1$?)}
\printbibliography % Ensure you have a 'references.bib' file and run biber/bibtex

\end{document}